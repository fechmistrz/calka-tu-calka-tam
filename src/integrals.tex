\documentclass[9pt, twoside, a5paper, fleqn]{extbook}
\usepackage[
	margin=10mm,
    headsep=5mm,
	twoside,
    includehead,
    % showframe,
]{geometry}

\usepackage{fancyhdr}
\fancypagestyle{plain}{
    \fancyhf{}
    \fancyfoot[C]{}
    \renewcommand{\headrulewidth}{0pt}
    \renewcommand{\footrulewidth}{0pt}
}
\pagestyle{fancy}
\fancyhead{}
\fancyfoot{}
% 341      Rozdział 7. Coś z algebry | 7.2 Pierścień adeli              342 |
\fancyhead[LE]{\thepage}
\fancyhead[RE]{\nouppercase{\leftmark}, wersja z 8 czerwca 2025}
\fancyhead[LO]{\nouppercase{\rightmark}, wersja z 8 czerwca 2025}
\fancyhead[RO]{\thepage}
\renewcommand{\headrulewidth}{0.4pt}
\renewcommand{\footrulewidth}{0.0pt}

\usepackage{parskip}
\usepackage{multicol}
\usepackage{float} % [H] for figure environment
\usepackage{hyperref} % for \url and \href

\usepackage{polski}
\usepackage[T1]{fontenc}

\usepackage{xcolor}

\usepackage{amsmath, amsfonts, amssymb, amsthm}
\DeclareMathOperator{\arsinh}{arsinh}
\DeclareMathOperator{\csch}{csch}

\newcommand{\N}{{\mathbb N}}
\newcommand{\R}{{\mathbb R}}
\newcommand{\Polylog}{\operatorname{Li}}


\newcounter{counter}
% \numberwithin{counter}{chapter}
\newtheorem{proposition}[counter]{Fakt}
\newtheorem{problem}[counter]{Problem}
\newtheorem{problem_with_solution}[counter]{Problem z rozwiązaniem}
\newtheorem{exercise}{Exercise}[chapter]

\theoremstyle{remark}
\newtheorem*{solution}{Rozwiązanie}

\author{Imię Nazwisko}
\title{Tytuł książki}

\usepackage{imakeidx}
\usepackage{makeidx}
\usepackage{etoolbox} % for patchcmd
\patchcmd{\theindex}{\MakeUppercase\indexname}{\sffamily\normalsize\bfseries\indexname}{}{}
\makeindex[title=Skorowidz]
\makeindex[name=persons,title=Indeks osób]

\usepackage{Alegreya}

\usepackage[printwatermark]{xwatermark}
\usepackage{xcolor}
\usepackage{graphicx}
\usepackage{lipsum}

\newwatermark*[allpages,color=red!50,angle=45,scale=3,xpos=0,ypos=0]{DRAFT}

\begin{document}

% strona pierwsza

\thispagestyle{empty}
{\noindent\fontsize{18pt}{18pt}\selectfont Księgozbiór matemagiczny, tom 71}

\noindent\makebox[\linewidth]{\rule{\textwidth}{1pt}}

\newpage

% koniec strony pierwszej


\input{00_intro/head_2}

% strona trzecia

\thispagestyle{empty}
{\noindent\fontsize{18pt}{18pt}\selectfont Leon z grodu Jelenia}

\noindent\makebox[\linewidth]{\rule{\textwidth}{1pt}}

\vspace{10mm}

{\noindent\fontsize{24pt}{24pt}\selectfont \textbf{Całki, całeczki i całkuńcie}}
\vspace{10mm}

{\noindent\fontsize{14pt}{14pt}\selectfont Wydanie zerowe}

\newpage

% koniec strony trzeciej



% strona czwarta

\thispagestyle{empty}
% \begin{figure}[H]
% \begin{minipage}[b]{.48\linewidth}
% {\noindent Prof. Imię Nazwisko\\
% Gdzie\\
% Gdzie dalej\\
% Gdzie kraj}
% \end{minipage}
% \end{figure}

{\noindent \textbf{Okładkę zaprojektował}\\projektant}
\vspace{5mm}

{\noindent \textbf{Zredagował}\\redaktor}
\vspace{5mm}

{\noindent \textbf{Zredagowała technicznie}\\redaktorka techniczna}
\vspace{5mm}

{\noindent \textbf{Złożyli i połamali}\\składacze z łamaczami}
\vspace{5mm}

{\noindent \textbf{Korekty dokonali}\\korektorzy}

\vfill

{\noindent Copyleft by Antykwariat Czarnoksięski, Gorzów Wielkopolski 2025.
Książka, a także każda jej część, mogą być przedrukowywane oraz w jakikolwiek inny sposób reprodukowane czy powielane mechanicznie, fotooptycznie, zapisywane elektronicznie lub magnetycznie, oraz odczytywane w środkach publicznego przekazu bez pisemnej zgody wydawcy.}

\vspace{5mm}

{\noindent Przygotowano w systemie \TeX, wydrukowano na siarczystym papierze.}

% koniec strony czwartej



% strona piąta

\tableofcontents

\chapter*{Przedmowa}
Wiele z zaprezentowanych tu problemów, ich rozwiązań (lub jednego i drugiego) zostało zaczerpnięte z popularnych zbiorów, takich jak \cite{wedrychowicz12} Banasia, Wędrychowicza.
Czasami autorzy proszą o zastosowanie wzoru na całkowanie przez części, kiedy zmyślne podstawienie prowadzi do celu krótszą drogą.
Nie przejmowaliśmy się zbytnio poleceniami z ich książki, ale żeby się w tym wszystkim nie pogubić, zamieszczamy na ostatnich stronach ,,tłumaczenie'' naszej numeracji na numerację z rozdziałów 12, 13 \cite{wedrychowicz12}.
Niektóre zadania nie zasługują na szczególną uwagę i choć ich treść została pominięta, nie brakuje ich we wspomniamym przed chwilą ,,tłumaczeniu''.

Gdzie to możliwe, staram się też proponować rozwiązania alternatywne, aby ukazać różnorodność narzędzi, które mogą być użyteczne dla osób całkujących.

TODO
$G$ oznacza...
Jeśli nie zaznaczono inaczej, $n \in \N$.

% koniec strony piątej



\raggedbottom

\chapter{Pochodne}
\input{sections/derivatives}

\chapter{Całki}
%

Czasami wystarczy zgadnąć wynik (albo znaleźć go w tablicy pochodnych) i sprawdzić, że pasuje przez zróżniczkowanie.
Na przykład wprost z faktu \ref{prp:derivative_power} wynika, że
\begin{problem}
    Niech $n \neq -1$.
    Wtedy
    \begin{equation}
        \int x^n \,\mathrm{d}x = \frac{x^{n+1}}{n+1}.
    \end{equation}
\end{problem}

%
\section{Całkowanie przez podstawianie}
% SOLUTION
\section{Całkowanie przez podstawianie}
% SOLUTION

% Banaś, Wędrychowicz, 12.18.
\begin{problem_with_solution}
    \label{banas_12_18}%
    \begin{equation}
        \int (\arcsin x)^2 \,\mathrm{d}x.
    \end{equation}
\end{problem_with_solution}

% SOLUTION
\textbf{Problem \ref{banas_12_18}}.
Podstawiamy $u = \arcsin x$ i dostajemy całkę z $u^2 \cos u$, którą rozwiązujemy przez części, tak jak w przykładzie \ref{banas_12_14}.
% SOLUTION

% Banaś, Wędrychowicz, 12.19.
\begin{problem_with_solution}
    \label{banas_12_19}%
    \begin{equation}
        \int \sin \log x \, \mathrm{d}x.
    \end{equation}
\end{problem_with_solution}

Analogicznie znajdujemy całkę funkcji $\cos \log x$.
% TODO: https://www.youtube.com/watch?v=q2043g_NogI

% SOLUTION
\textbf{Problem \ref{banas_12_19}}.
Podstawiamy $u = \log x$, $\mathrm{d} u = \mathrm{d} x / x$, $x = \exp u$ i dostajemy całkę z $e^u \sin u$, którą rozwiązujemy przez części, tak jak w przykładzie \ref{banas_12_19_auxilia}.
% SOLUTION

% Banaś, Wędrychowicz, 12.20.
% \begin{problem}
  %   \label{banas_12_20}%
    % \begin{equation}
      %   \int \cos(\log x) \, \mathrm{d}x.
    % \end{equation}
% \end{problem}

% Banaś, Wędrychowicz, 12.21.
\begin{problem_with_solution}
    \label{banas_12_21}%
    \begin{equation}
        \int \sqrt{x^2 + 1} \, \mathrm{d}x.
    \end{equation}
\end{problem_with_solution}

% SOLUTION
\textbf{Problem \ref{banas_12_21}}.
Podstawiamy $x = \tan \theta$.
Na mocy tożsamości trygonometrycznej $\tan^2 \theta + 1 = \sec^2 \theta$ nasza całka zmienia się w $\int \sec^2 \theta \cdot \sec \theta \,\mathrm{d}\theta$, czyli całkę z problemu \ref{banas_12_21_auxilia}.
Zatem
\begin{align}
    \int \sqrt{x^2 + 1} \, \mathrm{d}x & = \frac 12 \sec \theta \tan \theta + \log |\tan \theta + \sec \theta| \\
    & = \frac 1 2 x \sqrt{x^2 + 1} + \log \left|x + \sqrt{x^2+1}\right|.
\end{align}
% SOLUTION

% Banaś, Wędrychowicz, 12.30.
\begin{problem_with_solution}
    \label{banas_12_30}%
    \begin{equation}
        \int x \sqrt{a^2 - x^2} \, \mathrm{d}x.
    \end{equation}
\end{problem_with_solution}

% SOLUTION
\textbf{Problem \ref{banas_12_30}}.
Podstawić $u = a^2 - x^2$.
% SOLUTION

% Banaś, Wędrychowicz, 12.31.
\begin{problem_with_solution}
    \label{banas_12_31}%
    \begin{equation}
        \int \exp \sqrt x \, \mathrm{d}x.
    \end{equation}
\end{problem_with_solution}

% SOLUTION
\textbf{Problem \ref{banas_12_31}}.
Podstawić $u = \sqrt x$, następnie scałkować przez części: $f(u) = u$, $g'(u) = \exp u$.
% SOLUTION

% Banaś, Wędrychowicz, 12.33.
\begin{problem_with_solution}
    \label{banas_12_33}%
    \begin{equation}
        \int \tan x \, \mathrm{d}x.
    \end{equation}
\end{problem_with_solution}

Analogicznie znajdujemy całkę funkcji $\cot x$.

% SOLUTION
\textbf{Problem \ref{banas_12_33}}.
Podstawić $u = \sin x$.
% SOLUTION

% Banaś, Wędrychowicz, 12.34.
\begin{problem_with_solution}
    \label{banas_12_34}%
    \begin{equation}
        \int \frac{\mathrm{d}x}{x \log x}.
    \end{equation}
\end{problem_with_solution}

% SOLUTION
\textbf{Problem \ref{banas_12_34}}.
Podstawić $u = \log x$.
% SOLUTION

% Banaś, Wędrychowicz, 12.40.
\begin{problem_with_solution}
    \label{banas_12_40}%
    \begin{equation}
        \int \frac{x^3}{1+x^8} \, \mathrm{d}x.
    \end{equation}
\end{problem_with_solution}

% SOLUTION
\textbf{Problem \ref{banas_12_40}}.
Podstawiamy $u = x^4$ i dostajemy całkę funkcji $(1+x^2)^{-1}$, czyli ...
% TODO: dodać link jak już będzie gdzieś ta całka policzona
% SOLUTION

% Banaś, Wędrychowicz, 12.41.
\begin{problem_with_solution}
    \label{banas_12_41}%
    \begin{equation}
        \int \frac{x \,\mathrm{d}x}{\sqrt{1+x^4}}.
    \end{equation}
\end{problem_with_solution}

% SOLUTION
\textbf{Problem \ref{banas_12_41}}.
Podstawiamy $u = x^2$, a potem trygonometrycznie $\theta = \arctan u$.
% SOLUTION

% Banaś, Wędrychowicz, 12.55.
\begin{problem_with_solution}
    \label{banas_12_55}%
    \begin{equation}
        \int \frac{\mathrm{d}x}{\sqrt{1 + e^{2x}}}
    \end{equation}
\end{problem_with_solution}

% SOLUTION
\textbf{Problem \ref{banas_12_55}}.
Podstawić $u = 1 + e^{2x}$. % ?
% SOLUTION

% Banaś, Wędrychowicz, 12.57.
\begin{problem_with_solution}
    \label{banas_12_57}%
    \begin{equation}
        \int \frac{\sin x \cos x \, \mathrm{d}x}{\sqrt{(a \sin x)^2 + (b \cos x)^2}}.
    \end{equation}
\end{problem_with_solution}

% SOLUTION
\textbf{Problem \ref{banas_12_57}}.
Podstawić $u = (a \sin x)^2 + (b \cos x)^2$.
% SOLUTION

% Banaś, Wędrychowicz, 12.XX.
\begin{problem_with_solution}
    \label{banas_12_XX}%
    \begin{equation}
        \int \ldots \, \mathrm{d}x.
    \end{equation}
\end{problem_with_solution}

% SOLUTION
\textbf{Problem \ref{banas_12_XX}}.
% SOLUTION

% Banaś, Wędrychowicz, 12.XX.
\begin{problem_with_solution}
    \label{banas_12_XX}%
    \begin{equation}
        \int \ldots \, \mathrm{d}x.
    \end{equation}
\end{problem_with_solution}

% SOLUTION
\textbf{Problem \ref{banas_12_XX}}.
% SOLUTION

% Banaś, Wędrychowicz, 12.XX.
\begin{problem_with_solution}
    \label{banas_12_XX}%
    \begin{equation}
        \int \ldots \, \mathrm{d}x.
    \end{equation}
\end{problem_with_solution}

% SOLUTION
\textbf{Problem \ref{banas_12_XX}}.
% SOLUTION

% Banaś, Wędrychowicz, 12.XX.
\begin{problem_with_solution}
    \label{banas_12_XX}%
    \begin{equation}
        \int \ldots \, \mathrm{d}x.
    \end{equation}
\end{problem_with_solution}

% SOLUTION
\textbf{Problem \ref{banas_12_XX}}.
% SOLUTION

% Banaś, Wędrychowicz, 12.XX.
\begin{problem_with_solution}
    \label{banas_12_XX}%
    \begin{equation}
        \int \ldots \, \mathrm{d}x.
    \end{equation}
\end{problem_with_solution}

% SOLUTION
\textbf{Problem \ref{banas_12_XX}}.
% SOLUTION

% Banaś, Wędrychowicz, 12.XX.
\begin{problem_with_solution}
    \label{banas_12_XX}%
    \begin{equation}
        \int \ldots \, \mathrm{d}x.
    \end{equation}
\end{problem_with_solution}

% SOLUTION
\textbf{Problem \ref{banas_12_XX}}.
% SOLUTION

\begin{problem_with_solution}
    \label{nahin_1x_1x}%
    \begin{equation}
        \int_{-1}^1 \sqrt{\frac{1+x}{1-x}} \,\mathrm{d}x = \pi.
    \end{equation}
\end{problem_with_solution}

Znamy tę całkę z książki Nahina \cite{nahin15}.

% SOLUTION
\textbf{Problem \ref{nahin_1x_1x}} -- \cite[s. 115, 378]{nahin15}.
Wskazówka: podstawić $x = \cos 2 \varphi$.
% SOLUTION

% https://math.stackexchange.com/questions/170331/why-is-int-0-infty-frac-ln-x1x2-mathrmdx-0
\begin{problem_with_solution}
    \label{stack_170331}%
    \begin{equation}
        \int_0^\infty \frac{\log x}{1 + x^2} \,\mathrm{d}x.
    \end{equation}
\end{problem_with_solution}

% SOLUTION
\textbf{Problem \ref{stack_170331}} -- podstawić $\theta = \arctan x$, wtedy
\begin{equation}
    \int_0^{\pi/2} \log \tan \theta \,\mathrm{d}\theta = \int_0^{\pi/2} \log \sin \theta \,\mathrm{d}\theta - \int_0^{\pi/2} \log \cos \theta \,\mathrm{d}\theta = 0.
\end{equation}
% SOLUTION


\subsection{Podstawienia Eulera}

TODO: Banaś Wędrychowicz, 12.71 - 12.87

\begin{problem}
    Banaś-Wędrychowicz, 12.58.
\end{problem}

%
%

\section{Całkowanie przez części}

\begin{proposition}[wzór na całkowanie przez części]
\label{prp_int_by_parts}%
    Jeśli funkcje $f, g \colon I \to \R$ są różniczkowalne, to
    \begin{equation}
        \int f(x) g'(x) \,\mathrm{d}x = f(x) g(x) - \int f'(x) g(x) \,\mathrm{d} x.
    \end{equation}
\end{proposition}

\begin{proof}
    Najpierw całkujemy obie strony wzoru na pochodną iloczynu $(fg)' = fg' + f'g$, a następnie porządkujemy strony równości.
\end{proof}

% Banaś, Wędrychowicz: 12.1
\begin{problem_with_solution}
    \label{banas_12_1}%
    \begin{equation}
        \int x \sin x \,\mathrm{d} x.
    \end{equation}
\end{problem_with_solution}

% SOLUTION
\textbf{Problem \ref{banas_12_1}}.
Całkujemy przez części, $f(x) = x$, $g'(x) = \sin x$.
\begin{align}
    \int x \sin x \,\mathrm{d} x & = -x \cos x - \int - \cos x \, \mathrm{d}x \\
                                 & = -x \cos x + \sin x.
\end{align}
% SOLUTION

Analogicznie obliczamy całki funkcji $x \cos x$ albo $x \exp x$.

% Banaś, Wędrychowicz: 12.6
\begin{problem_with_solution}
    \label{banas_12_6}%
    \begin{equation}
        \int x \arctan x \,\mathrm{d} x.
    \end{equation}
\end{problem_with_solution}

% SOLUTION
\textbf{Problem \ref{banas_12_6}}.
Całkujemy przez części, $f(x) = \arctan x$, $g'(x) = x$.
\begin{align}
    \int x \arctan x \, \mathrm{d} x
    & = \frac 12 x^2 \arctan x - \int \frac{x^2 \,\mathrm{d}x}{2(x^2+1)} \\
    & = \frac 12 x^2 \arctan x - \frac 12 \left(\int 1 \,\mathrm{d}x - \int \frac{\mathrm{d}x}{x^2+1} \right) \\
    & = \frac 12 x^2 \arctan x - \frac 12 \left(x - \arctan x \right) \\
    & = \frac 12 \left((x^2+1)\arctan x - x \right).
\end{align}
% SOLUTION

% Banaś, Wędrychowicz: 12.7
\begin{problem_with_solution}
    \label{banas_12_7}%
    \begin{equation}
        \int x^n \log x \,\mathrm{d} x.
    \end{equation}
\end{problem_with_solution}

% SOLUTION
\textbf{Problem \ref{banas_12_7}}.
Całkujemy przez części, $f(x) = \log x$, $g'(x) = x^n$.
\begin{align}
    \int x^n \log x \, \mathrm{d} x & = \frac{x^{n+1} \log x}{n+1} - \int \frac{x^n \,\mathrm{d} x}{n+1} \\
                                    & = \frac{x^{n+1} \log x}{n+1} - \frac{x^{n+1}}{(n+1)^2}.
\end{align}
% SOLUTION

% Banaś, Wędrychowicz: 12.8
\begin{problem_with_solution}
    \label{banas_12_8}%
    \begin{equation}
        \int \arccos x \,\mathrm{d} x.
    \end{equation}
\end{problem_with_solution}

% SOLUTION
\textbf{Problem \ref{banas_12_8}}.
Całkujemy najpierw przez części, $f(x) = \arccos x$, $g'(x) = 1$, żeby następnie podstawić $u = 1 - x^2$, $\mathrm{d} u = -2x \mathrm{d}x$:
\begin{align}
    \int \arccos x \, \mathrm{d} x & = x \arccos x - \int  \frac{-x \,\mathrm{d}x}{\sqrt{1-x^2}} \\
    & = x \arccos x - \frac 12 \int \frac {\mathrm{d}u}{\sqrt{u}} \\
    & = x \arccos x - \sqrt{1 - x^2}.
\end{align}
% SOLUTION

Analogicznie obliczamy całkę funkcji $\arcsin x$.

% Banaś, Wędrychowicz: 12.10
\begin{problem_with_solution}
    \label{banas_12_10}%
    \begin{equation}
        \int x \cdot (\tan x)^2 \,\mathrm{d} x.
    \end{equation}
\end{problem_with_solution}

% SOLUTION
\textbf{Problem \ref{banas_12_10}}.
Całkujemy przez części, $f(x) = x$, $g'(x) = (\tan x)^2$.
\begin{align}
    \int x \cdot (\tan x)^2 x \, \mathrm{d} x & = x (\tan x - x) - \int (\tan x - x) \,\mathrm{d}x \\
    & = x (\tan x - x) - \left(-\log(\cos(x)) - \frac{x^2}{2}\right).
\end{align}
% SOLUTION

% Banaś, Wędrychowicz: 12.11
\begin{problem_with_solution}
    \label{banas_12_11}%
    \begin{equation}
        \int x \cdot (\cos x)^2 \,\mathrm{d} x.
    \end{equation}
\end{problem_with_solution}

% SOLUTION
\textbf{Problem \ref{banas_12_11}}.
Ponieważ $\cos 2x = 2 \cos^2 x - 1$, potrzebujemy znaleźć prostszą całkę 
\begin{align}
    \int x \cos 2x \, \mathrm{d} x.
\end{align}
Całkujemy przez części: $f(x) = x$, $g'(x) = \cos 2x$, co prowadzi do jeszce prostszej całki funkcji $\sin 2x$.
Ostatecznie
\begin{align}
    \int x \cos 2x \, \mathrm{d} x = \frac 1 8 \left(2x^2 + 2x \sin 2x + \cos 2x\right).
\end{align}
% SOLUTION

% Banaś, Wędrychowicz, 12.12 to całka z x log(x^2+1), ale tam wystarczy podstawić u = x^2 + 1, wtedy du = 2x dx.
% Banaś, Wędrychowicz, 12.16
% Banaś, Wędrychowicz, 12.17.
\begin{problem_with_solution}
    \label{banas_12_12}%
    \begin{equation}
        \int (\log x)^n \,\mathrm{d}x.
    \end{equation}
\end{problem_with_solution}

% SOLUTION
\textbf{Problem \ref{banas_12_12}}.
Całkujemy przez części, $f(x) = (\log x)^n$, $g'(x) = 1$.
Dostajemy początek rekurencji:
\begin{equation}
    \int (\log x)^n \, \mathrm{d}x = x (\log x)^n - n \int (\log x)^{n-1} \,\mathrm{d} x.
\end{equation}
z warunkiem brzegowym:
\begin{equation}
    \int \log x\, \mathrm{d}x = x\log x - x.
\end{equation}
% SOLUTION

% Banaś, Wędrychowicz, 12.13.
\begin{problem_with_solution}
    \label{banas_12_13}%
    \begin{equation}
        \int x^n e^x \,\mathrm{d} x.
    \end{equation}
\end{problem_with_solution}

% SOLUTION
\textbf{Problem \ref{banas_12_13}}.
Dowiedziemy tego indukcyjnie.
Dla $n = 0$, całka jest elementarna.
Jeżeli $n \ge 1$, to całkujemy przez części: $f(x) = x^n$, $g'(x) = e^x$ i dostajemy zależność rekurencyjną
\begin{equation}
    I_n = x^n e^x - nI_{n-1},
\end{equation}
której rozwiązaniem jest
\begin{equation}
    I_n = e^x \sum_{k=0}^n (-1)^{n-k} \frac{n!}{k!}x^k.
\end{equation}.
% SOLUTION

% Banaś, Wędrychowicz, 12.14.
% Banaś, Wędrychowicz, 12.15.
% Banaś, Wędrychowicz, 12.25.
\begin{problem_with_solution}
    \label{banas_12_14}%
    \begin{equation}
        \int x^3 \sin x \, \mathrm{d}x.
    \end{equation}
\end{problem_with_solution}

% SOLUTION
\textbf{Problem \ref{banas_12_14}}.
Całkujemy przez części, $f(x) = x^3$, $g'(x) = \sin x$.
Dostajemy początek rekurencji:
\begin{equation}
    \int x^3 \sin x \, \mathrm{d}x = - x^3 \cos x - \int - 3x^2 \cos x \,\mathrm{d}x
\end{equation}
rozwiązaniem której jest $3 (x^2-2) \sin x + x (6-x^2) \cos x$.
% SOLUTION

\begin{problem}
    % pomocnicza dla Banaś 12.19
    \label{banas_12_19_auxilia}%
    \begin{equation}
        \int e^x \sin x \,\mathrm{d}x = \frac {e^x} 2 (\sin x - \cos x).
    \end{equation}
\end{problem}

% Banaś, Wędrychowicz, 12.22.
% \begin{problem}
% Banaś-Wędrychowicz, 12.22. % x^2 e^x sin x
% \end{problem}

% Banaś, Wędrychowicz, 12.23.
% \begin{problem}
% Banaś-Wędrychowicz, 12.23. % x / (sin ^2 x)
% \end{problem}

% Banaś, Wędrychowicz, 12.24.
% \begin{problem}
% Banaś-Wędrychowicz, 12.24. % x arcsin x / (1 - x^2)
% \end{problem}

% Banaś, Wędrychowicz, 12.26.
\begin{problem_with_solution}
    \label{banas_12_26}%
    \begin{equation}
        \int \frac{x \log(\sqrt{x^2+1}+x)}{\sqrt{x^2+1}} \,\mathrm{d}x.
    \end{equation}
\end{problem_with_solution}

% SOLUTION
\textbf{Problem \ref{banas_12_26}}.
Zauważamy, że $\log(\sqrt{x^2+1} + x) = \arsinh x$, a następnie całkujemy przez części: $f(x) = \arsinh x$, $g'(x) = x / \sqrt{x^2+1}$, wtedy $f(x) = 1/\sqrt{x^2+1}$, $g(x) = \sqrt{x^2+1}$.
Ostateczny wynik to $\sqrt{x^2+1} \arsinh x - x$.
% SOLUTION

% Banaś, Wędrychowicz, 12.27.
\begin{problem_with_solution}
    \label{banas_12_27}%
    \begin{equation}
        \int \arctan \sqrt{x} \,\mathrm{d}x.
    \end{equation}
\end{problem_with_solution}

% SOLUTION
\textbf{Problem \ref{banas_12_27}}.
Całkujemy przez części: $f(x) = \arctan \sqrt x$, $g'(x) = 1$, potem podstawiamy $u = \sqrt{x}$, co prowadzi do klasycznej całki funkcji $(u^2 + 1)^{-1}$.
% TODO: gdzie jest ta klasyczna całka?
% SOLUTION

\begin{problem_with_solution}
    \label{banas_12_21_auxilia}%
    \begin{equation}
        \int \sec^3 x \,\mathrm{d}x.
    \end{equation}
\end{problem_with_solution}

% SOLUTION
\textbf{Problem \ref{banas_12_21_auxilia}}.
Całkujemy najpierw przez części, $f(x) = \sec x$, $g'(x) = \sec^2 x$.
Pamiętając, że $\tan^2 x = \sec^2 x - 1$, mamy
\begin{align}
    I & = \sec x \tan x - \int \sec x \tan^2 x \,\mathrm{d} x \\
        & = \sec x \tan x - I + \int \sec x \,\mathrm{d}x \\
    2I & = \sec x \tan x + \log |\tan x + \sec x|.
\end{align}
% SOLUTION

%

\section{Całkowanie funkcji wymiernych}
% SOLUTION
\section{Całkowanie funkcji wymiernych}
% SOLUTION

\begin{problem_with_solution}
    \label{banas_12_100}%
\begin{equation}
    \int \frac{\mathrm{d}x}{x^3 + x}.
\end{equation}
\end{problem_with_solution}

% SOLUTION
\textbf{Problem \ref{banas_12_100}} -- wyłączyć z mianownika $x^3$, podstawić $u = 1 + x^{-2}$ i znaleźć całkę z funkcji $1/u$.
Alternatywnie, rozłożyć na ułamki proste i dopiero potem podstawiać.
Wynik to
\begin{equation}
    I_{B100} = \log x - \frac 12 \log (x^2 + 1).
\end{equation}
% SOLUTION

\begin{problem_with_solution}
    \label{banas_12_118}%
\begin{equation}
    \int \frac{3x^3 - 8x + 5}{\sqrt{x^2 - 4x - 7}} \,\mathrm{d}x.
\end{equation}
\end{problem_with_solution}

% SOLUTION
\textbf{Problem \ref{banas_12_118}} -- podstawić $u = x - 2$, a potem $w = \sqrt{11} \sec u$.
To prowadzi do całki, która jest kombinacją liniową potęg sekansa ($\sec, \sec^2, \sec^3, \sec^4$).
Wynik po odwróceniu podstawień to
\begin{equation}
    \sqrt{x^2 - 4x - 7} (x^2 + 5x + 36) + 112 \log \left(2 - x - \sqrt{x^2 - 4x-7}\right).
\end{equation}
% SOLUTION














\begin{problem}
\label{boros_4287}%
\begin{equation}
    \int_0^\infty \frac{x^n \,\mathrm{d}x}{(ax+b)^{m+1}}  = \frac{(-1)^{n+1} (-1-m)! \cdot n!}{a^{n+1} b^{m-n} (n-m)!}
\end{equation}
\end{problem}

% SOLUTION
\textbf{Problem \ref{boros_4287}} -- patrz \cite[s. 48-60]{boros04}
% SOLUTION

\begin{problem}
\label{frac_1_x3_1}%
\begin{equation}
    \int_0^\infty \frac{\mathrm{d}x}{x^3 - 1} = - \frac{\pi}{3\sqrt{3}}
\end{equation}
\end{problem}

% SOLUTION
\textbf{Problem \ref{frac_1_x3_1}} -- patrz \cite[s. 22]{nahin15}
% SOLUTION

\begin{problem}
    \label{experimental_mathematics_p258}%
\begin{equation}
    \int_0^\infty \frac{x^8-4x^6+9x^4-5x^2+1}{x^{12}-10x^{10}+37x^8-42x^6+26x^4-8x^2+1} \,\mathrm{d}x = \frac{\pi}{2}
\end{equation}
\end{problem}

% SOLUTION
\textbf{Problem \ref{experimental_mathematics_p258}} -- patrz \cite[s. 258]{bailey07}.
% SOLUTION

Rozdział 3 Borosa i Molla (TODO), rozdział 6. Całki postaci zostawia jako teren do eksploracji dla Czytelnika, a sami rozważają całki postaci
\begin{equation}
    \int_0^\infty \frac{P(x) \,\mathrm{d}x}{(q_4 x^4 + q_2 x^2 + q_0)^{m+1}},
\end{equation}
gdzie $m \in \mathbb N$, zaś $P$ jest wielomianem stopnia co najwyżej $4m + 2$.
Przykładowe zadania: \ref{boros_moll_7_1_1}, \ref{boros_moll_7_2_3}.

\subsection{Rozkład na ułamki proste}
% SOLUTION
\subsection{Rozkład na ułamki proste}
% SOLUTION

\begin{problem_with_solution}
    \label{boros_moll_7_1_1}
\begin{equation}
    \int_0^\infty \frac{3x^3 \,\mathrm{d}x}{(x^4 + 4x^2 + 1)^5} = \frac{5}{20736} \cdot \left(60 + 7 \sqrt{3} \log \left(7 - 4 \sqrt {3}\right)\right).
\end{equation}
\end{problem_with_solution}

% SOLUTION
\textbf{Problem \ref{boros_moll_7_1_1}} -- patrz \cite[s. 138]{boros04}.
% SOLUTION

\begin{problem_with_solution}
\label{nahin_page_225}%
    \begin{equation}
		\int_1^\infty \frac{\mathrm{d}x}{x^3 + x} = \log \sqrt {2}.
    \end{equation}
\end{problem_with_solution}

% SOLUTION
\textbf{Problem \ref{nahin_page_225}}.
Rozkładamy całkowaną funkcję na sumę
\begin{equation}
	\frac 1 x - \frac{1}{2(x - i)} - \frac{1}{2(x+i)}
\end{equation}
tak jak Nahin \cite[s. 225]{nahin15}.
% SOLUTION

%
\subsection{Oszacowania liczby $\pi$}
\begin{problem}[całka Dalzella]
\label{22_7_pi}%
\begin{equation}
    \int_0^1 \frac{x^4(1-x)^4}{1 + x^4} \,\mathrm{d}x = \frac{22}{7} - \pi.
\end{equation}
\end{problem}

Donald Percy Dalzell \cite{dalzell44} jako pierwszy opublikował to cudo.
Ograniczając mianownik z dołu oraz góry przez $1$ oraz $2$ możemy dojść do wniosku, że
\begin{equation}
    \frac{22}{7} - \frac {1}{630} < \pi < \frac{22}{7} - \frac{1}{1260},
\end{equation}
a więc pomylić się o mniej niż $0.015\%$!

% TODO - na en.wiki jest łatwe w przepisaniu rozwiązanie
% SOLUTION
\textbf{Problem \ref{22_7_pi}} -- patrz \cite[s. 24]{nahin15}
% SOLUTION

% TODO: patrz też https://math.stackexchange.com/questions/1956/is-there-an-integral-that-proves-pi-333-106
Istnieje wiele uogólnień powyższego wyniku do innych przybliżeń liczby $\pi$, dobrym źródłem dalszych informacji, a może nawet inspiracji jest artykuł Lucasa \cite{lucas05}.
Na przykład:

\begin{problem}
\begin{equation}
    \int_0^1 \frac {x^8(1-x)^8 (25+816x^2)}{3164 (1+x^2)} \,\mathrm{d} x = \frac {355}{113} - \pi.
\end{equation}
\end{problem}

albo:

\begin{problem}
\begin{equation}
    \int_0^1 \frac{x^5 ( 1-x)^6 (197 + 462 x^2)}{530 (1+x^2)} \,\mathrm{d}x = \pi - \frac{333}{106}.
\end{equation}
\end{problem}
%

\section{Całkowanie funkcji trygonometrycznych}
% SOLUTION
\section{Całkowanie funkcji trygonometrycznych}
% SOLUTION

\begin{problem_with_solution}
    \label{banas_12_160}%
\begin{equation}
    \int \frac{\cos x}{\sin^3 x - \cos^3 x} \, \mathrm{d} x.
\end{equation}
\end{problem_with_solution}

% SOLUTION
\textbf{Problem \ref{banas_12_160}} -- podstawić $u = \tan x$, co prowadzi do całki z $1/(u^3-1)$ i wyniku
\begin{equation}
    -\frac {1}{\sqrt{3}} \arctan \left( \frac{1 + 2 \tan x}{\sqrt{3}} \right) 
    + 2 \log (\cos x - \sin x) 
    - \log (2 + \sin 2x)
\end{equation}
% SOLUTION

%

\section{Całki różne, różniste}
% SOLUTION
\section{Całki różne, różniste}
% SOLUTION

% https://en.wikipedia.org/wiki/Integral_of_the_secant_function#History
Gerardus Mercator\footnote{Flamandzki matematyk i geograf.} opublikował \emph{,,Nova et Aucta Orbis Terrae Descriptio ad Usum Navigantium Emendata''}, czyli mapę świata, w 1569 roku.
Do jej sporządzenia wykorzystał nowe odwzorowanie Ziemi, zwane obecnie także walcowym równokątnym.
\index{odwzorowanie Mercatora}%
Niestety nie przedstawił nigdzie swoich obliczeń!
Usterkę tę naprawił dopiero Edward Wright w roku 1599, natrafiając jednocześnie na problem:
\index[persons]{Wright, Edward}

\begin{problem}
    \begin{equation}
        \int \sec x \,\mathrm{d} x = \log {| \sec x + \tan x|}.
    \end{equation}
\end{problem}

Czas mijał (około czterdzieści lat), aż Henry Bond\footnote{Nauczyciel nawigacji, geodezji oraz innych matematycznych rzeczy.} porównał zawartość tablic logarytmicznych oraz wyniki Wrighta, co doprowadziło do postawienia hipotezy, jaka jest zwarta postać całki:
\begin{equation}
    \log \tan \left(\frac \pi 4 + \frac x 2\right).
\end{equation}
\index[persons]{Bond, Henry}%
Potwierdzenie nie przyszło od razu; pierwszy był James Gregory\footnote{Szkocki matematyk i astronom.} w pracy \emph{,,Exercitationes Geometricae''} z 1668 roku, ale było tak trudne w zrozumieniu (chociaż poprawne; podstawił $u = \sec x + \tan x$), że Isaac Barrow\footnote{Angielski teolog i matematyk.} zaproponował w \emph{,,Lectiones Geometricae''} z 1670 roku zupełnie inne podejście.
\index[persons]{Gregory, James}%
\index[persons]{Barrow, Isaac}%
Dowód Barrowa jest znany przede wszystkim dlatego, że zawiera najstarszy znany rozkład na ułamki proste podczas całkowania (!) po podstawieniu $u = \sin x$:
\index{rozkład na ułamki proste}%
\begin{align}
    \int \sec x \,\mathrm{d}x = \int \frac{\cos x}{1-\sin^2 x}\,\mathrm{d}x = \int \frac{\mathrm{d}u}{1-u^2} = \frac 1 2 \int \frac{\mathrm{d}u}{1 + u} + \frac 1 2 \int \frac{\mathrm{d}u}{1 - u} = \ldots
\end{align}
Do tego samego rozkładu ułamków prowadzi podstawienie trygonometryczne $t = \tan (x/2)$.

% TODO: https://math.stackexchange.com/questions/9286/evaluation-of-gaussian-integral-int-0-infty-mathrme-x2-dx
% https://math.stackexchange.com/a/9292/1298830
\begin{problem}[całka Gaußa, całka Eulera-Poissona]
    \label{gauss_euler_poisson}%
    \begin{equation}
        \int_{-\infty}^\infty \exp \left( -x^2 \right) \,\mathrm{d} x = \sqrt{\pi}.
    \end{equation}
\end{problem}
% TODO: https://www.youtube.com/watch?v=6hI1ij6l8e4

Wiemy, że Abraham de Moivre myślał o całkach podobnych do powyższej około 1733 roku; być może ktoś inny go ubiegł, ale nie pozostawił po sobie żadnego tropu.
\index[persons]{de Moivre, Abraham}%
Dokładny wynik podał dopiero Gauß na początku następnego wieku (rok 1809); wspomniał przy tym, że jest on zasługą Laplace'a.
\index[persons]{Gauss, Carl Friedrich}%

Oznaczmy szukaną całkę przez $I$.
Laplace pokazał przy użyciu twierdzenia Fubiniego, że:
\begin{align}
    I^2 & = 4 \int_0^\infty \int_0^\infty \exp(-x^2 -y^2)\,\mathrm{d}y \,\mathrm{d}x \\
    & = 4 \int_0^\infty \int_0^\infty x \exp [-x^2(1+s^2)]\,\mathrm{d}s \,\mathrm{d}x \\
    & = 4 \int_0^\infty \int_0^\infty x \exp [-x^2(1+s^2)]\,\mathrm{d}x \,\mathrm{d}s \\
    & = 4 \int_0^\infty \left. \frac{\exp(-x^2(1+s^2))}{-2(1+s^2)}\right|_{0}^{\infty} \,\mathrm{d}s \\
    & = 2 \int_0^\infty \frac{\mathrm{d}s}{1+s^2} \\
    & = 2 \left. \arctan s \right|_{0}^\infty = \pi.
\end{align}

Obecnie preferowana jest metoda Poissona, który zauważył, że można zrobić to samo, co wyżej (skoro całkowana funkcja $\exp (\ldots)$ jest nieujemna, to całka jest równa pierwiastkowi ze swojego kwadratu), ale wykorzystując po drodze współrzędne biegunowe:
\index[persons]{Poisson, ?}%
\begin{align}
    I^2 & = \left(\int_{-\infty}^\infty \exp \left( -x^2 \right) \,\mathrm{d}x\right)\left(\int_{-\infty}^\infty \exp \left( -y^2 \right) \,\mathrm{d}y\right) \\
    & = \int_{-\infty}^\infty \int_{-\infty}^\infty \exp \left(-x^2-y^2\right) \,\mathrm{d}x\,\mathrm{d}y \\
    & = \int_0^{2\pi} \int_0^\infty r \exp (-r^2) \,\mathrm{d}r\, \mathrm{d}\theta \\
    & = 2\pi \int_0^\infty r \exp (-r^2) \,\mathrm{d} r \\
    & = 2\pi \int^0_{-\infty} \frac 1 2 \exp s \,\mathrm{d} s \\
    & = \pi.
\end{align}
Co ciekawe, opisana technika nie działa wobec jakiejkolwiek innej całki! Patrz notka Dawsona \cite{dawson05}.

Algorytm Rischa pokazuje, że całka nieoznaczona tej samej funkcji, $\exp (-x^2)$, nie daje się wyrazić przez funkcje elementarne.
Natomiast całkę pozornie bardziej skomplikowanej funkcji $x \exp (-x^2)$ można szybko znaleźć podstawiając $u = x^2$, a jeśli w naszym arsenale jest jeszcze różniczkowanie pod znakiem całki, to wykażemy prawie tak samo szybko, że
\begin{equation}
    \int_{-\infty}^\infty x^{2n} \exp (-x^2) \,\mathrm{d}x = \frac{(2n-1)!!}{2^n} \sqrt \pi.
\end{equation}

% https://math.stackexchange.com/questions/34767/int-infty-infty-e-x2-dx-with-complex-analysis
Inne rozwiązanie zaczyna się od podstawienia $u = x^2$.
Wtedy na mocy wzoru odbiciowego Eulera $\Gamma (z) \Gamma(1-z) = \pi/\sin \pi z$ mamy:
\begin{align}
	I = \int_0^\infty u^{-1/2} e^{-u}\,\mathrm{d}u = \Gamma \left(\frac 12\right) = \sqrt{\pi}.
\end{align}
(Jeszcze więcej rozwiązań dostarcza dyskusja pod pytaniem nr 34767 w portalu Math Stackexchange).




\begin{problem}[sinus całkowy]
    \begin{equation}
        \int_0^\infty \frac {\sin x}{x} \,\mathrm{d} x = \frac \pi 2.
    \end{equation}
\end{problem}

% TODO https://www.youtube.com/watch?v=XbgiqCCWfSE
% https://www.youtube.com/watch?v=zet9sRvEazg


Sinus (i kosinus) całkowy pojawiają się także w całkach:
\begin{align}
    I_1 & = \int_0^\infty \frac{\sin mx}{x + b} \,\mathrm{d}x, \\
    I_2 & = \int_0^\infty \frac{\sin mx}{ax^2 + bx + c} \,\mathrm{d}x,
\end{align}
jak panowie Boros, Moll \cite[s. 136]{boros04} napisali.

% TODO: https://math.stackexchange.com/questions/13344/proof-of-int-0-infty-left-frac-sin-xx-right2-mathrm-dx-frac-pi2
\begin{problem}
    \begin{equation}
        \int_0^\infty \left(\frac {\sin x}{x}\right)^2 \,\mathrm{d} x = \frac \pi 2.
    \end{equation}
\end{problem}

\subsection{Funkcje Gamma i Beta}
% SOLUTION
\subsection{Funkcje Gamma i Beta}
% SOLUTION
\begin{problem}
\label{exp_x_3_gamma_4_3}%
    \begin{equation}
        I = \int_{0}^\infty \exp \left( -x^3 \right) \,\mathrm{d} x = \Gamma \left( \frac 4 3 \right).
    \end{equation}
\end{problem}

Patrz też \cite[s. 119]{nahin15}.

% SOLUTION
\textbf{Problem \ref{exp_x_3_gamma_4_3}} -- podstawiamy $y = x^3$, wtedy $\mathrm{d}x = \frac 1 3 y^{-2/3} \mathrm{d}y$
% SOLUTION

\begin{problem}
\label{wallis_x_x2_n}%
    \begin{equation}
        I_n = \int_{0}^1 (x-x^2)^n \,\mathrm{d} x = \frac{(n!)^2}{(2n+1)!}
    \end{equation}
\end{problem}

Około 1650 roku John Wallis badał tę całkę i zgadł jej wartość dla dowolnego $n \in \N$ na podstawie wartości dla małych wartości.
Patrz też \cite[s. 119-122]{nahin15}.

% SOLUTION
\textbf{Problem \ref{wallis_x_x2_n}} -- zauważamy, że całkę Wallisa można wyrazić przez funkcję Beta:
\begin{align}
	I_n = \int_{0}^1 x^n (1-x)^n \,\mathrm{d} x = B(n+1, n+1) = \frac{\Gamma(n+1) \Gamma(n+1)}{\Gamma(2n+2)}.
\end{align}
% SOLUTION

Po podstawieniu $n = 1/2$ w problemie \ref{wallis_x_x2_n} widzimy, że $I_{1/2} = \frac 1 2 \Gamma(3/2)^2$.
Ale tę samą całkę można znaleźć wykorzystując geometryczną interpretację całki jako pola powierzchni pod wykresem funkcji.
Łatwo widać, że wykres to górny półokrąg o środku w punkcie $(1/2, 0)$ i promieniu $(1/2)$, zatem szukane pole to $\frac \pi 8$, i stąd wynika już, że $\Gamma(3/2) = \sqrt{\pi}/2$.
Teraz wystarczy wstawić uzyskane wyniki do definicji funkcji Gamma i odkryć jak \cite[s. 123]{nahin15}, że
\begin{equation}
    \int_{0}^\infty \exp(-x) \sqrt{x} \,\mathrm{d} x = \frac{\sqrt \pi}{2}.
\end{equation}

\begin{problem_with_solution}
\label{sqrt_minus_log}%
    \begin{equation}
        \int_{0}^1 \sqrt{- \log x} \,\mathrm{d} x = \frac{\sqrt \pi}{2}.
    \end{equation}
\end{problem_with_solution}

% SOLUTION
\textbf{Problem \ref{sqrt_minus_log}} -- podstawiamy $y = - \log x$.
% SOLUTION

\begin{problem_with_solution}
    \label{boros_moll_p95}%
    \begin{equation}
        \int_0^\infty x^n \exp (-px) \,\mathrm{d}x = \frac{n!}{p^{n+1}}.
    \end{equation}
\end{problem_with_solution}

Rodzina wszystkich kombinacji liniowych funkcji postaci $x^n \exp (-x)$ jest zamknięta na branie całek nieoznaczonych, o~pokazanie tego proszą Boros i Moll \cite[s. 103]{boros04}.
Podobnie jest dla całek funkcji postaci $x^n \sin(x)^m$, patrz \cite[s. 135]{boros04} oraz funkcji postaci  $x^n \sin(x)^m$, patrz  \cite[s. 136]{boros04}. % TODO przepisać te całki?

% SOLUTION
\textbf{Problem \ref{boros_moll_p95}} -- zróżniczkować $n$ razy względem $p$, jak Boros, Moll \cite[s. 95]{boros04}.
% SOLUTION

\begin{problem_with_solution}
    \label{boros_moll_p97}%
    \begin{equation}
        \int_0^1 x^n \log^k x \,\mathrm{d}x = \frac{(-1)^k \cdot k!}{(n+1)^{k+1}}.
    \end{equation}
\end{problem_with_solution}

% SOLUTION
\textbf{Problem \ref{boros_moll_p97}} -- zróżniczkować całkę z $x^n$ nad $[0, 1]$ względem $n$ jak Boros, Moll \cite[s. 97]{boros04}.
% SOLUTION

Po zamianie zmiennych dostajemy:

\begin{problem}
    \begin{equation}
        \int_0^\infty x^k \exp (-x) \,\mathrm{d}x = k!.
    \end{equation}
\end{problem}

%%%%%%%%%%%%%%%%%

\begin{problem_with_solution}
    Niech $P$ będzie wielomianem stopnia $2m$. Znaleźć
    \label{boros_moll_p105}%
    \begin{equation}
        \int_0^\infty \frac{P(x) \,\mathrm{d}x}{(ax^2 + bx + c)^{m+1}}
    \end{equation}
\end{problem_with_solution}

Szczególny przypadek ($P(x) \equiv 1$, $a = c = 1$, $b = 0$) został rozprawiony się z nim w 1656 przez Wallisa.

% SOLUTION
\textbf{Problem \ref{boros_moll_p105}} -- Boros, Moll \cite[s. 105--koniecrozdziału6]{boros04}.
% SOLUTION

\textbf{Ramanujan's Master Theorem}: jeśli funkcja $F$ rozwija się w szereg Taylora, to momenty $F$ dane są wzorem... Boros, Moll, ps. 151.

%

\section{Całkowanie różniczek dwumiennych} % https://encyclopediaofmath.org/wiki/Differential_binomial
Całkowanie różniczek dwumiennych

%
% \section{Całkowanie funkcji trygonometrycznych}
% Całkowanie funkcji trygonometrycznych
	%

\section{Sztuczka Feynmana: różniczkowanie pod znakiem całki}
% SOLUTION
\section{Sztuczka Feynmana: różniczkowanie pod znakiem całki}
% SOLUTION

% https://math.stackexchange.com/questions/942263/really-advanced-techniques-of-integration-definite-or-indefinite
\begin{problem}
    \begin{equation}
        \int_0^\infty \frac{\sin x}{x} \,\mathrm{d}x = \frac \pi 2.
    \end{equation}
\end{problem}

% TODO: przepisać całkę z s. 82, Nahin

\begin{problem_with_solution}
    \label{nahin_holzweg}%
    Niech $a, b > 0$, wtedy
    \begin{equation}
        \int_{-\infty}^\infty \frac{\cos ax}{b^2 - x^4} \,\mathrm{d} x = \frac{\pi}{b} \sin (ab).
    \end{equation}
\end{problem_with_solution}

% SOLUTION
\textbf{Problem \ref{nahin_holzweg}} -- \cite[s. 115, 375, 376]{nahin15}.
% SOLUTION

\begin{problem}
    \label{nahin_datenautobahn}%
    Niech $a > b$, wtedy
    \begin{equation}
        \int_{-\infty}^\infty \frac{\cos ax}{b^4 - x^4} \,\mathrm{d} x = \frac{\pi}{2b^3} [\sin (ab) + \exp (-ab)].
    \end{equation}
\end{problem}

% SOLUTION
\textbf{Problem \ref{nahin_datenautobahn}} -- \cite[s. 115, 376]{nahin15}.
% SOLUTION

% Nahin Inside interesting... page 83
\begin{problem}
    \begin{equation}
        \int_0^\infty \frac{\sin ax}{x e^{xy}} \,\mathrm{d}x = \pm \frac \pi 2 - \arctan \frac y a.
    \end{equation}
\end{problem}

\begin{problem}
    Niech $a > 0$, wtedy
    \begin{equation}
        \int_0^\infty \frac{\sin ax}{x} \,\mathrm{d}x = \frac \pi 2.
    \end{equation}
\end{problem}
% SKAD TO


\begin{problem}[całka Frullaniego]
\index{całka Frullaniego}%
    Niech $f \colon [0, \infty) \to \R$ będzie funkcją ciągle różniczkowalną, której granica w nieskończoności istnieje.
    Wtedy dla ustalonych liczb rzeczywistych $a, b > 0$ mamy
    \begin{equation}
        \int_0^\infty \frac{f(ax) - f(bx)}{x} \,\mathrm{d} x = \left[\lim_{x \to \infty} f(x) - f(0) \right] \cdot \log \frac a b.
    \end{equation}
\end{problem}

Okazuje się, że problem rozwiązał Cauchy (około 1823 roku), ale też Giuliano Frullani\footnote{Włoski matematyk.} (zapowiedź w 1821 roku, publikacja około 1829 roku).
\index[persons]{Frullani, Giuliano}%
Nahin \cite[s. 85]{nahin15} używa tej nazwy do konkretnego wcielenia całki Frullaniego, dla $f = \arctan$.
Boros, Moll \cite[s. 98]{boros04} piszą mgliście \emph{,,under some mild conditions on the function $f$''}...

% Nahin Inside interesting... page 8x
\begin{problem}
    Niech $a, b > 0$.
    Wtedy
    \begin{equation}
        \int_0^\infty \frac{e^{-ax} - e^{-bx}}{x} \,\mathrm{d}x = \log \frac b a.
    \end{equation}
\end{problem}

% Nahin Inside interesting... page 89
\begin{problem}
    \begin{equation}
        \int_0^\infty \frac{\cos (ax) - \cos (bx)}{x^2} \,\mathrm{d}x = \frac \pi 2 (b - a).
    \end{equation}
\end{problem}

% Nahin Inside interesting... page 89
\begin{problem}
    \begin{equation}
        \int_0^\infty \frac{\cos (ax) - \cos (bx)}{x} \,\mathrm{d}x = \log \frac b a.
    \end{equation}
\end{problem}

% Nahin Inside interesting... page 89
\begin{problem}
    \label{nahin_kriegsrecht}
    \begin{equation}
        \int_0^\infty \frac{\log (a^2 x^2 + 1)}{x^2 + b^2} \,\mathrm{d}x = \frac \pi b \log (1 + ab).
    \end{equation}
\end{problem}

% SOLUTION
\textbf{Problem \ref{nahin_kriegsrecht}} -- \cite[s. 67]{nahin15} w szczególnym przypadku $a = b = 1$; \cite[s. 114, 375]{nahin15} w ogólności.
% SOLUTION

% Nahin Inside interesting... page 91
\begin{problem}
    Niech $a \ge 0$, wtedy
    \begin{equation}
        \int_0^1 \frac{x^a - 1}{\log x} \,\mathrm{d}x = \log(1+a).
    \end{equation}
\end{problem}

% Nahin Inside interesting... page 92
\begin{problem}
    Niech $a \ge 0$, wtedy
    \begin{equation}
        \int_0^1 \frac{x^a - x^b}{\log x} \,\mathrm{d}x = \log \frac{1+a}{1+b}.
    \end{equation}
\end{problem}

% Nahin Inside interesting... page 96
\begin{problem}
    Niech $a > b$, wtedy
    \begin{equation}
        \int_0^\pi \frac{\mathrm{d}x} {a + b \cos x} = \frac{\pi}{\sqrt{a^2 - b^2}}.
    \end{equation}
\end{problem}

\begin{problem}
    \label{nahin_dini}%
    Niech $a \ge 0$ będzie dowolną liczbą rzeczywistą.
    Wtedy
    \begin{equation}
        \int_0^\pi \log (1 - 2 a \cos x + a^2) \,\mathrm{d} x = \begin{cases}
            0, & \text{gdy } a^2 \le 1, \\
            2 \pi \log a & \textrm{w przeciwnym razie}.
        \end{cases}
    \end{equation}
\end{problem}

Nahin pisze, że powyższą całkę wyznaczył jako pierwszy Ulisse Dini\footnote{Włoski matematyk.} w 1878 roku i że (całka, nie Dini) ma ważne zastosowania w fizyce i inżynierii.
\index[persons]{Dini, Ulisse}%

% SOLUTION
\textbf{Problem \ref{nahin_dini}} -- \cite[s. 109-112]{nahin15}
% SOLUTION

% https://math.stackexchange.com/questions/580521/generalizing-int-01-frac-arctan-sqrtx2-2-sqrtx2-2
\begin{problem_with_solution}[całka Ahmeda]
    \label{ahmed_integral}%
    \begin{equation}
        \int_0^1 \frac{\arctan \sqrt{x^2+2}}{(x^2+1) \sqrt{x^2+2}} \,\mathrm{d}x = \frac{5\pi^2}{96}.
    \end{equation}
\end{problem_with_solution}

% TODO: https://www.youtube.com/watch?v=LkhtyxTnTZw When Ahmed and Glasser Meet...

% SOLUTION
\textbf{Problem \ref{ahmed_integral}} -- patrz praca Ahmeda Zafara \cite{ahmed02}.
\index[persons]{Zafar, Ahmed}
Rozwiązanie podaje też Nahin \cite[s. 190-194]{nahin15}.
% SOLUTION

\begin{problem_with_solution}[całka Coxetera]
    \label{coxeter_integral}%
    \begin{equation}
        \int_0^{\pi/2} \arccos \frac{\cos x}{1 + 2\cos x}  \,\mathrm{d}x = \frac{5\pi^2}{24}.
    \end{equation}
\end{problem_with_solution}

Młody Harold Coxeter zamieścił tę całkę w liście do Mathematical Gazette, rozwiązanie zostało nadesłane przez Hardy'ego.

% SOLUTION
\textbf{Problem \ref{coxeter_integral}} -- Nahin \cite[s. 190-201]{nahin15}; to prawdopodobnie najdłuższe rozwiązanie w całej jego książce.
% SOLUTION




%

\section{Funkcje Gamma oraz Beta}
	%

Funkcji $\Gamma$ jest poświęcony dziesiąty rozdział książki Borosa, Molla \cite[s. 186-???]{boros04}.

Współczesna definicja
\begin{equation}
    \Gamma(x) := \int_0^\infty \exp(-t) \cdot t^{x-1} \,\mathrm{d}t
\end{equation}
pochodzi od Legendre'a (1809), ale Euler preferował równoważne wyrażenie
\begin{equation}
    \Gamma(x) := \int_0^1 (- \log t)^{x-1} \,\mathrm{d}t.
\end{equation}

Funkcja $\Gamma$ spełnia równanie funkcyjne $\Gamma(x + 1) = x \Gamma (x)$ i stanowi uogólnienie silnii, mamy bowiem $\Gamma(k) = (k-1)!$ dla wszystkich dodatnich całkowitych $k$.
Bohr, Mollerup pokazali w 1922, że jest to jedyna funkcja $f \colon (0, \infty) \to (0, \infty)$ taka, że $f(1) = 1$, $f(x) > 0$ dla $x > 0$, spełnione jest równanie funkcyjne i $\log f$ jest wypukła.
Wielandt scharakteryzował ją jako jedyną analityczną funkcję, która spełnia równanie funkcyjne i jest ograniczona na pasku $re z \in [1, 2]$.

Mamy
\begin{equation}
    \Gamma(z) \Gamma(1-z) = \frac{\pi}{\sin \pi z},
\end{equation}

zatem $\Gamma (1/2) = \sqrt{\pi}$.

Funkcja beta (Beta?) dana jest wzorem
\begin{equation}
    B(x, y) = \int_0^1 t^{x-1} (1-t)^{y-1} \,\mathrm{d}t
\end{equation}
i wiąże ją z funkcją Gamma relacja
\begin{equation}
    B(x, y) = \frac{\Gamma(x) \Gamma(y)}{\Gamma (x+y)}.
\end{equation}

Euler pokazał, że 
\begin{equation}
    \int_0^1 \log \Gamma (z) \,\mathrm{d}z = \log \sqrt{2 \pi}.
\end{equation}

(Józef Raabe znalazł całkę na zbiorze $[a, a+1]$ w 1840 roku).
% https://www.youtube.com/watch?v=Yf57U4_8SVo = https://en.wikipedia.org/wiki/Gamma_function#Raabe's_formula

\begin{problem_with_solution}
    \label{boros_moll_10_6_1}%
    \begin{equation}
        \int_0^1 \log^2 \Gamma(x) \,\mathrm{d} x = \frac{\gamma^2}{12} + \frac{\pi^2}{48} + \frac 13 \gamma \log \sqrt{2\pi} + \frac 43 \log^2 \sqrt{2\pi} - (\gamma + 2 \log \sqrt{2 \pi}) \frac{\zeta'(2)}{\pi^2} + \frac{\zeta''(2)}{2\pi^2}.
    \end{equation}
\end{problem_with_solution}

% SOLUTION
\textbf{Problem \ref{boros_moll_10_6_1}} -- Boros, Moll \cite[s. 203]{boros04} piszą, że całkę znaleźli Espinosa, Moll (2002).
% SOLUTION

% definicja funkcji psi jako Gamma'/Gamma

\begin{problem_with_solution}
    \label{boros_moll_10_12_1}%
    \begin{equation}
        \psi(x) = \int_0^\infty \left(\frac{\exp(-t)}{t} - \frac{\exp(-xt)}{1 - \exp(-t)}\right) \,\mathrm{d}t
    \end{equation}
\end{problem_with_solution}

% SOLUTION
\textbf{Problem \ref{boros_moll_10_12_1}} -- Boros, Moll \cite[s. 216]{boros04}.
% SOLUTION

%

\section{Zadania z turniejów całkowania}
	%

% Harvard:
% TODO: https://www.youtube.com/watch?v=hxAUEat_04o

% 2025
% TODO: https://www.youtube.com/watch?v=rGolSnrWc7s


\subsection{MIT Integration BEE 2024}
% kwalifikacje, problem 2: \frac{(x-1)^{\log (x+1)}}{(x+1)^{\log (x-1)}} = 1
% https://www.youtube.com/watch?v=p75k1DbuS0o

\begin{problem_with_solution}[ćwierćfinał 2, problem 2]
    \label{bee_mit_2024_q2_p2}%
    \begin{equation}
        I = \int_0^1 \frac 1 x \log (1 + x^2 + x^3 + x^4 + x^5 + x^6 + x^7 + x^9) \,\mathrm{d}x
    \end{equation}
\end{problem_with_solution}

% SOLUTION
% https://math.stackexchange.com/questions/1617081/proving-an-integration-equality
\textbf{Problem \ref{bee_mit_2024_q2_p2}} -- łatwo widać, że szukana całka jest równa $I_2 + I_3 + I_4$, gdzie
\begin{align}
    I_n & = \int_0^1 \frac {\log (1 + x^n)}{x} \,\mathrm{d}x \\
        & = \frac 1 n \int_0^1 \frac {\log (1 + x)}{x} \,\mathrm{d}x \\
        & = \frac 1 n \int_0^1 \frac 1 x \sum_{k=1}^\infty \frac{(-1)^{k-1}x^k}{k} \,\mathrm{d}x \\
        & = \frac 1 n \sum_{k=1}^\infty \frac{(-1)^{k-1}}{k^2} \\
        & = \frac {\pi^2}{12n}.
\end{align}
% SOLUTION

% qualifier q7 https://www.youtube.com/watch?v=YI-c9b7vNEs
% q9 https://www.youtube.com/watch?v=hzWtJ-qadws
% q15 https://www.youtube.com/watch?v=mxoyNN5g3hE
% q19 https://www.youtube.com/watch?v=7JwGegZDz4Y

% 2023
% https://math.stackexchange.com/questions/4642139/question-from-mit-integration-bee-2023-final-evaluate-int1-0-sum-infty-n
% https://www.youtube.com/watch?v=gn6LOe_bgw4

% 2022
% regular q8 https://www.youtube.com/watch?v=MlyvsB7PBlI

% 2020
% TODO: https://www.youtube.com/watch?v=oZWqG4IIHc0

% 2012
% q10, qualifier https://www.youtube.com/watch?v=HV489m57f2s

%

%

\section{Teoretyczna teoria}
\section{Teoretyczna teoria} % SOLUTION

\begin{problem}[problem B4 na egzaminie Putnam 1968]
    \label{putnam_1968_b4}%
    Niech $f \colon \R \to \R$ będzie ciągłą funkcją taką, że całka $\int_\R f(x)\,\mathrm{d}x$ istnieje.
    Pokazać, że całka
    \begin{equation}
        \int_\R f\left(x - \frac 1 x\right)\,\mathrm{d}x
    \end{equation}
    też istnieje i przyjmuje tę samą wartość.
\end{problem}

% SOLUTION
\begin{solution}[do problemu \ref{putnam_1968_b4}]
    Będziemy całkować przez podstawienie, $x = \exp \theta$ (i potem $x = - \exp -\theta$):
    \begin{align}
        \int_{-\infty}^{\infty}f\left(x-x^{-1}\right)dx&=\int_{0}^{\infty}f\left(x-x^{-1}\right)dx+\int_{-\infty}^{0}f\left(x-x^{-1}\right)dx=\\
        &=\int_{-\infty}^{\infty}f(2\sinh\theta)\,e^{\theta}d\theta+\int_{-\infty}^{\infty}f(2\sinh\theta)\,e^{-\theta}d\theta=\\
        &=\int_{-\infty}^{\infty}f(2\sinh\theta)\,2\cosh\theta\,d\theta=\\
        &=\int_{-\infty}^{\infty}f(x)\,dx.
    \end{align}
\end{solution}
% SOLUTION

%

\section{Trudne całki}
	%

\begin{problem_with_solution}
    \label{reuleaux_tetrahedron}%
    Czworościan Reuleaux to bryła będąca częścią wspólną czterech kul, których środki leżą w wierzchołkach czworościanu foremnego, a promienie są tej samej długości, co krawędzie tego czworościanu.
    Znaleźć objętość tej bryły,
    \begin{equation}
        V = \int_0^1
        \frac{
            8\sqrt{3}
        }{
            1 + 3t^2
        } - \frac{
            16 \sqrt{2} (3t+1) (4t^2 +t+1)^{3/2}
        }{
            (3t^2+1)(11t^2 + 2t + 3)^2
        } - \frac{
            \sqrt{2} (249 t^2 + 54t + 65)
        }{
            (11t^2 + 2t +3)^2
        } \,\mathrm{d} t.
    \end{equation}
\end{problem_with_solution}

% SOLUTION
\textbf{Problem \ref{reuleaux_tetrahedron}} -- patrz \url{https://mathworld.wolfram.com/ReuleauxTetrahedron.html}.
% SOLUTION

% TODO: https://mathworld.wolfram.com/images/gifs/FoxTrotMathTest.jpg
% TODO https://mathworld.wolfram.com/DefiniteIntegral.html




\begin{problem_with_solution}
    \label{schuster_integral}%
    Niech $S, C$ oznaczają całki Fresnela.
    % TODO: https://en.wikipedia.org/wiki/Fresnel_integral
    \begin{equation}
        \int_0^\infty (S(x)^2 + C(x)^2) \,\mathrm{d}x = \sqrt{\frac{\pi}{8}} \approx  0.62665\,70686\ldots
    \end{equation}
\end{problem_with_solution}

W 1925 roku brytyjski fizyk Arthur Schuster opublikował pracę na temat teorii światła, w której natknął się na powyższą całkę.
Sam nie potrafił jej wyznaczyć, ale zajął się tym Hardy -- dostał ten samą wartość, którą Schuster przypuszczał.

% SOLUTION
\textbf{Problem \ref{schuster_integral}} -- patrz Nahin \cite[s. 201-205]{nahin15}.
% SOLUTION

\begin{problem_with_solution}
    \label{watson_integrals1}%
    \begin{equation}
        I_1 = \frac{1}{\pi^3} \int_0^\pi\int_0^\pi\int_0^\pi \frac{\mathrm{d}u \, \mathrm{d}v \, \mathrm{d}w}{1 - \cos u \cos v \cos w} = \frac{\Gamma(1/4)^4}{4 \pi^3} = 1.39320\,39296\ldots
    \end{equation}
\end{problem_with_solution}

\begin{problem_with_solution}
    \label{watson_integrals2}%
    \begin{equation}
        I_2 = \frac{1}{\pi^3} \int_0^\pi\int_0^\pi\int_0^\pi \frac{\mathrm{d}u \, \mathrm{d}v \, \mathrm{d}w}{3 - \cos u \cos v - \cos u \cos w - \cos v \cos w} = \frac{3 \Gamma(1/3)^6}{2^{14/3} \pi^4} = 0.44822\,03943\ldots
    \end{equation}
\end{problem_with_solution}

\begin{problem_with_solution}
    \label{watson_integrals3}%
    \begin{equation}
        I_3 = \frac{1}{\pi^3} \int_0^\pi\int_0^\pi\int_0^\pi \frac{\mathrm{d}u \, \mathrm{d}v \, \mathrm{d}w}{3 - \cos u - \cos v - \cos w} = \frac{\Gamma(1/24) \Gamma(5/24) \Gamma(7/24) \Gamma(11/24)}{16 \sqrt{6} \pi^3} = 0.50546\,20197\ldots
    \end{equation}
\end{problem_with_solution}

W 1938 roku van Peype, student holenderskiego fizyka Kramnersa, napisał pracę, gdzie pojawiły się te trzy całki.
Chociaż van Peype znał wartość $I_1$, nie mógł sobie poradzić z $I_2$ oraz $I_3$, więc wysłał je do brytyjskiego fizyka Ralpha Fowlera, który przekazał je Hardy'emu, który nie poradził sobie z nimi.
Wynik jako pierwszy uzyskał George Watson, matematyk angielski.

% SOLUTION
\textbf{Problemy \ref{watson_integrals1}, \ref{watson_integrals2}, \ref{watson_integrals3}} -- patrz Nahin \cite[s. 206-212]{nahin15}.
% SOLUTION


%
	\subsection{Prawie niemożliwe całki}
% SOLUTION
\subsection{Prawie niemożliwe całki}
% SOLUTION
Wszystkie poniższe całki pojawiają się w książce Valeana \cite{valean19}.

\begin{problem_with_solution}
    \label{valean_grundpreis}%
    Niech $y \in (-1, 1)$.
    Wtedy
    \begin{equation}
        \int_0^1 \frac{\mathrm{d}x}{(1+yx) \sqrt{1-x^2}} = \frac{\arccos y}{\sqrt{1-y^2}}.
    \end{equation}
\end{problem_with_solution}

% SOLUTION
\textbf{Problem \ref{valean_grundpreis}} -- 
patrz \cite[s. 1]{valean19}.
% SOLUTION

\begin{problem_with_solution}
    \label{valean_zeugenstand}%
    Niech $m, n$ będą liczbami naturalnymi.
    Wtedy
    \begin{equation}
        \int_0^1 x^m \log^n x \,\mathrm{d} x = \frac{(-1)^n \cdot n!}{(m+1)^{n+1}}.
    \end{equation}
\end{problem_with_solution}

% SOLUTION
\begin{solution}[do problemu \ref{valean_zeugenstand}]
    Patrz \cite[s. 1]{valean19}.
\end{solution}
% SOLUTION


Niech $H_{n}^{(m)} = 1 + 1/2^m + \ldots + 1/n^m$ oznacza $n$-tą uogólnioną liczbę harmoniczną.

\begin{problem_with_solution}
    \label{valean_1_3}%
    Rozpatrujemy rodzinę całek
    \begin{equation}
        I_{k,n} := \int_0^1 x^{n-1} \log^k (1-x) \,\mathrm{d} x.
    \end{equation}
    Mamy:
    \begin{align}
        I_{1,n} & = - \frac{H_n}{n} \\
        I_{2,n} & = \frac{H_n^2 + H_n^{(2)}}{n} \\
        I_{3,n} & = - \frac{H_n^3 + 3H_nH_n^{(2)} + 2H_n^{(3)}}{n} \\
        I_{4,n} & = \frac{H_n^4 + 6H_n^2 H_n^{(2)} + 8H_nH_n^{(3)} + 3(H_n^{(2)})^2 + 6H_n^{(4)}}{n}.
    \end{align}
\end{problem_with_solution}

% (Valean nazywa to ,,four logarithmic integrals strongly connected with the league of harmonic series'').

% SOLUTION
\begin{solution}[do problemu \ref{valean_1_3}]
    Patrz \cite[s. 2]{valean19}.
\end{solution}
% SOLUTION

\begin{problem_with_solution}
    \label{valean_1_5}%
    Niech $s > 0$ będzie liczbą rzeczywistą, zaś $\psi$ oznacza funkcję digamma.
    Wtedy
    \begin{align}
        \int_0^1 \frac{x^{s-1}}{x+1} \,\mathrm{d} x & = \psi(s) - \psi\left(\frac s2\right) - \log 2 \\
        \int_0^\infty e^{-sx} \tanh x \,\mathrm{d} x & = \frac 1 2 \left[\psi\left(\frac{s+2}{4}\right) - \psi \left(\frac s4 \right) - \frac 2 s\right]. 
    \end{align}
\end{problem_with_solution}

% (Valean nazywa to ,,a couple of practical definite integrals expressed in terms of the digamma function'').

% SOLUTION
\begin{solution}[do problemu \ref{valean_1_5}]
    Patrz \cite[s. 3]{valean19}.
\end{solution}
% SOLUTION

\begin{problem_with_solution}
    \label{valean_1_7}%
    \begin{align}
        \int_0^1 \frac{1}{x} \log^2 (1+x) \,\mathrm{d}x & = \frac{1}{4} \zeta(3) \\
        \int_0^1 \frac{1}{x} \log (1+x) \log (1-x) \,\mathrm{d}x & = -\frac{5}{8} \zeta(3)
    \end{align}
\end{problem_with_solution}

% two little tricky classical logarithmic integrals

% SOLUTION
\begin{solution}[do problemu \ref{valean_1_7}]
    Patrz \cite[s. 4]{valean19}.
\end{solution}
% SOLUTION

\begin{problem_with_solution}[]
    \label{valean_1_8}%
    \begin{align}
        \int_0^1 [\log(1+x) \log(1-x)]^2 \,\mathrm{d} x & =
        24 - 8 \zeta(2)- 8 \zeta(3) - \zeta(4) \\
        & + 8 \log(2)\zeta(2) + 8 \log(2)\zeta(3) \\
        & - 4 \log^2(2)\zeta(2) \\
        & - 24 \log(2) + 12 \log^2(2)- 4 \log^3(2) + \log^4(2); 
    \end{align}
\end{problem_with_solution}

% a special trio of integrals

% SOLUTION
\begin{solution}[do problemu \ref{valean_1_8}]
    Patrz \cite[s. 4, 5]{valean19}.
\end{solution}
% SOLUTION

% TODO: https://math.stackexchange.com/questions/3413586/conjectural-closed-form-of-int-01-frac-logn-1-x-logn-1-1x1x-d

\begin{problem_with_solution}
    \label{valean_1_10}%
    Niech $n \ge 1$ będzie liczbą naturalną.
    Znaleźć
    \begin{equation}
        I_n = \int_0^1 \frac 1 x \log(1-x) \log^{2n} x \log (1+x) \,\mathrm{d}x.
    \end{equation}
    Jeśli jest to za trudne, pokazać, że
    \begin{align}
        I_1 & = \frac 3 4 \zeta (2) \zeta (3) - \frac {27}{16} \zeta(5), \\
        I_2 & = \frac 9 4 \zeta (3) \zeta (4) + \frac{45}{4} \zeta(2) \zeta(5) - \frac{363}{16} \zeta (7), \\
        I_3 & = \frac{2835}{8} \zeta(2) \zeta (7) + \frac {135}{8} \zeta (3) \zeta (6) + \frac {675}{8} \zeta (4) \zeta (5) - \frac {22635}{32} \zeta (9).
    \end{align} 
\end{problem_with_solution}

% the evaluation of a class of logarithmic integrals using a slightly modified result from ,,Table of Integrals, Series and Products'' by I. S. Gradshteyn and I. M. Ryzhik together with a series result elementarily proved by Guy Bastien

% SOLUTION
\begin{solution}[do problemu \ref{valean_1_10}]
    Patrz \cite[s. 6, 7]{valean19}.
\end{solution}
% SOLUTION

\begin{problem_with_solution}
    \label{valean_1_13}%
    \begin{equation}
        \int_0^1 \frac{x \log (1 \pm x)}{1 + x^2} \, \mathrm{d} x = \frac 1 8 \left(\log^2 (2) + \frac{\pm 3 - 2}{2} \zeta(2)\right).
    \end{equation} 
\end{problem_with_solution}

% A Special Pair of Logarithmic Integrals with Connections in the Area of the Alternating Harmonic Series

% SOLUTION
\begin{solution}[do problemu \ref{valean_1_13}]
    Patrz \cite[s. 8]{valean19}.
\end{solution}
% SOLUTION

% Another Special Pair of Logarithmic Integrals with Connections in the Area of the Alternating Harmonic Series

\begin{problem_with_solution}
    \label{valean_1_14}%
    \begin{align}
        16 \int_0^1 \frac{x}{1+ x^2} \log (1 - x) \log x \,\mathrm{d}x & = \frac{41}{4} \zeta(3) - 9 \log(2) \zeta(2) \\
        16 \int_0^1 \frac{x}{1+ x^2} \log (1 + x) \log x \,\mathrm{d}x & = -\frac{15}{4} \zeta(3) + 3 \log(2) \zeta(2)
    \end{align} 
\end{problem_with_solution}

% A Special Pair of Logarithmic Integrals with Connections in the Area of the Alternating Harmonic Series

% SOLUTION
\begin{solution}[do problemu \ref{valean_1_14}]
    Patrz \cite[s. 8]{valean19}.
\end{solution}
% SOLUTION

\begin{problem_with_solution}
    \label{valean_1_17}%
    \begin{align}
        \int_0^1 \int_0^1 \frac{\log y - \log x}{\log (- \log x) - \log(- \log y)} \,\mathrm{d}x \,\mathrm{d}y = \frac{7 \zeta(3)}{6 \zeta (2)}.
    \end{align} 
\end{problem_with_solution}

% Let’s Take Two Double Logarithmic Integrals with Beautiful Values Expressed in Terms of the Riemann Zeta Function

% SOLUTION
\begin{solution}[do problemu \ref{valean_1_17}]
    Patrz \cite[s. 10]{valean19}.
\end{solution}
% SOLUTION

\begin{problem_with_solution}
    \label{valean_1_18}%
    Niech $G$ oznacza stałą Catalana.
    \begin{align}
        8 \int_0^1 \log (1 - x) \arctan x \,\mathrm{d}x & = 4 \log (2) - \log^2 (2) + \frac 5 2 \zeta(2) - 2 \pi + \pi \log 2 - 8 G \\
        8 \int_0^1 \log (1 + x) \arctan x \,\mathrm{d}x & = 4 \log (2) - \log^2 (2) - \frac 1 2 \zeta(2) - 2 \pi + 3 \pi \log 2.
    \end{align} 
\end{problem_with_solution}

% Interesting Integrals Containing the Inverse Tangent Function and the Logarithmic Function

% SOLUTION
\begin{solution}[do problemu \ref{valean_1_18}]
    Patrz \cite[s. 10, 11]{valean19}.
\end{solution}
% SOLUTION

\begin{problem_with_solution}
    \label{valean_1_20}%
    Niech $G$ oznacza stałą Catalana.
    \begin{align}
        8 \int_0^1 \frac{\log (1 - x) \arctan x}{1+x^2} \,\mathrm{d}x & = \frac 3 4 \log (2) \zeta(2) - \frac 7 8 \zeta(3) - \pi G, \\
        8 \int_0^1 \frac{\log (1 + x) \arctan x}{1+x^2} \,\mathrm{d}x & = \frac 3 4 \log (2) \zeta(2) + \frac {21} 8 \zeta(3) - \pi G,
    \end{align} 
\end{problem_with_solution}

% More Interesting Integrals Involving the Inverse Tangent Function and the Logarithmic Function: The First Part

% SOLUTION
\begin{solution}[do problemu \ref{valean_1_20}]
    Patrz \cite[s. 12]{valean19}.
\end{solution}
% SOLUTION

\begin{problem_with_solution}
    \label{valean_1_21}%
    Niech $G$ oznacza stałą Catalana.
    \begin{align}
        \int_0^1 \frac{\arctan^2 x \log (1 + x)}{1 + x^2} \,\mathrm{d} x = \log 2 \frac {\pi^3}{384} + \frac {21}{256} \pi \zeta(3) - \frac{3}{16} \zeta (2) G.
    \end{align} 
\end{problem_with_solution}

% More Interesting Integrals Involving the Inverse Tangent Function and the Logarithmic Function: The Second Part

% SOLUTION
\begin{solution}[do problemu \ref{valean_1_21}]
    Patrz \cite[s. 12]{valean19}.
\end{solution}
% SOLUTION

\begin{problem_with_solution}
    \label{valean_1_22}%
    Niech $G$ oznacza stałą Catalana.
    \begin{align}
        I & = \int_0^1 \arctan x \log x \left(\log (1-x) - \frac {x}{1-x}\right) \,\mathrm{d} x \\
        & = G - \frac{41}{64} \zeta (3) + \frac{9 \log 2 - 5}{96} \pi^2 + \frac{2 - \log 2}{8} \pi - \frac {\log 2}{2} + \frac{\log^2 (2)}{8}.
    \end{align} 
\end{problem_with_solution}

% Challenging Integrals Involving arctan(x), log(x), log(1−x)

% SOLUTION
\begin{solution}[do problemu \ref{valean_1_22}]
    Patrz \cite[s. 13]{valean19}.
\end{solution}
% SOLUTION


\begin{problem_with_solution}
    \label{valean_1_23}%
    \begin{align}
        I & = \int_0^1 \arctan x \log x \log (1 + x) \,\mathrm{d}x \\
        & = \frac{\log 2}{2} G - \frac{\pi^3}{64} + \frac{15}{64} \zeta(3) - \frac{\pi^2}{96} (3 \log 2 + 1) + \frac{\pi} {8} (4-3 \log 2) + \frac{\log^2(2)}{8} - \log 2.
    \end{align} 
\end{problem_with_solution}

% Challenging Integrals Involving arctan(x), log(x), log(1−x)

% SOLUTION
\begin{solution}[do problemu \ref{valean_1_23}]
    Patrz \cite[s. 13, 14]{valean19}.
\end{solution}
% SOLUTION



\begin{problem_with_solution}
    \label{valean_1_24}%
    Niech $n \ge 1$ będzie naturalne.
    Znaleźć wartość
    \begin{align}
        I_{2n} & := \int_0^1 \frac {\arctan x \log^{2n} (x)}{1 + x } \,\mathrm{d}x
    \end{align} 
    lub, jeśli jest to za trudne, pokazać, że 
    \begin{align}
        I_{1} = \frac{\log 2}{2}G - \frac{\pi^3}{64}.
    \end{align} 
\end{problem_with_solution}

% Challenging Integrals Involving arctan(x), log(x), log(1−x)

% SOLUTION
\begin{solution}[do problemu \ref{valean_1_24}]
    Patrz \cite[s. 14, 15]{valean19}.
\end{solution}
% SOLUTION


\begin{problem_with_solution}
    \label{valean_1_26}%
    \begin{align}
        \int_0^1 \frac{\arctan x}{x} \log \frac{1+x^2}{(1-x)^2} \,\mathrm{d}x = \frac{\pi^3}{16}.
    \end{align} 
\end{problem_with_solution}

% Challenging Integrals Involving arctan(x), log(x), log(1−x)

% SOLUTION
\begin{solution}[do problemu \ref{valean_1_26}]
    Patrz \cite[s. 17]{valean19}.
\end{solution}
% SOLUTION

\begin{problem_with_solution}
    \label{valean_1_32}%
    Znaleźć rekurencję, jaką spełnia
    \begin{align}
        I_n = \int_0^1 \frac{x^n}{(1+x)(1+x^2)^n} \,\mathrm{d}x.
    \end{align} 
\end{problem_with_solution}

% Challenging Integrals Involving arctan(x), log(x), log(1−x)

% SOLUTION
\begin{solution}[do problemu \ref{valean_1_32}]
    Patrz \cite[s. 21, 22]{valean19}.
\end{solution}
% SOLUTION

\begin{problem_with_solution}
    \label{valean_1_37}%
    \begin{align}
        \int_0^\infty \int_0^\infty \frac {e^{-x}-e^{-y}}{x-y} \frac{1-e^{-x}}{x} \frac{1-e^{-y}}{y} \,\mathrm{d}x \,\mathrm{d}y.
    \end{align} 
\end{problem_with_solution}

% SOLUTION
\begin{solution}[do problemu \ref{valean_1_37}]
    Patrz \cite[s. ?????]{valean19}.
\end{solution}
% SOLUTION


\begin{problem_with_solution}
    \label{valean_1_38}%
    Znaleźć
    \begin{align}
        \lim_{n\to\infty} \left(\frac{1}{n!} \int_0^\infty \int_0^\infty \frac{x^n - y^n}{e^x - e^y} \,\mathrm{d}x \,\mathrm{d}y - 2n\right)
    \end{align} 
    po znalezieniu wewnętrznej całki (dla całkowitych liczb $n \ge 1$).
\end{problem_with_solution}

% SOLUTION
\begin{solution}[do problemu \ref{valean_1_38}]
    Patrz \cite[s. ?????]{valean19}.
\end{solution}
% SOLUTION

\begin{problem_with_solution}
    \label{valean_1_40}%
    \begin{align}
        I_4 & = \int_0^\infty \frac{\pi^2}{x^3} \tanh (\pi x)  + \frac{3}{x^5} \tanh(\pi x) - \frac{3\pi}{x^4} \,\mathrm{d}x = 93 \zeta(5) - 42 \zeta(2) \zeta(3), \\
        I_5 & = \int_0^\infty \frac{2\pi^2}{x^3} \tanh(\pi x) + \frac{12}{x^5} \tanh (\pi x) - \frac{6 \pi^2}{x^3} \csch (2 \pi x) - \frac{9 \pi}{x^4} \,\mathrm{d} x \\
        & = 372 \zeta(5) - 192 \zeta(2)\zeta(3).
    \end{align} 
\end{problem_with_solution}

% SOLUTION
\begin{solution}[do problemu \ref{valean_1_40}]
    Patrz \cite[s. ?????]{valean19}.
\end{solution}
% SOLUTION


\begin{problem_with_solution}
    \label{valean_1_41}%
    \begin{align}
        I_1 & = \int_0^\infty \frac{\sin (\sin x)}{x} e^{\cos x} \,\mathrm{d} x = \frac{\pi} 2 (e - 1), \\
        I_2 & = \int_0^\infty \frac{\sin x \cdot \sin (\sin x)}{x^2} e^{\cos x} \,\mathrm{d} x = \frac{\pi} 2 (e - 1).
    \end{align} 
\end{problem_with_solution}

% SOLUTION
\begin{solution}[do problemu \ref{valean_1_41}]
    Patrz \cite[s. ?????]{valean19}.
\end{solution}
% SOLUTION


%
	%

\subsection{Znalezione na math.stackexchange.com}
% SOLUTION
\subsection{Znalezione na math.stackexchange.com}
% SOLUTION
Wszystkie poniższe całki pojawiają się na stronie na math.stackexchange.com.

%%

% https://math.stackexchange.com/q/1653979
\begin{problem}[pytanie 9402]
    \label{stack_9402}%
    % TODO: https://math.stackexchange.com/questions/9402/calculating-the-integral-int-0-infty-frac-cos-x1x2-mathrmdx-with
    \begin{equation}
        \int_0^\infty \frac {\cos b x}{1+x^2} \,\mathrm{d}x = \frac{\pi} 2 \exp(-b).
    \end{equation}
\end{problem}

%%

\begin{problem}[pytanie 15719]
    \label{stack_15719}%
    \begin{equation}
        \int \frac{1 + x^2}{(1 - x^2) \sqrt{1 + x^4}} \,\mathrm{d}x = \frac{1}{\sqrt 2} \log \frac{2x + \sqrt{2x^4 + 2}}{x^2 - 1}.
    \end{equation}
\end{problem}

%%

% https://math.stackexchange.com/questions/110457/closed-form-for-int-0-infty-fracxn1-xmdx
\begin{problem_with_solution}[pytanie 110457]
    \label{stack_110457}%
    Niech $0 < n < m$, wtedy
    \begin{equation}
        \int_0^\infty \frac{n x^{n-1}}{1 + x^m} \,\mathrm{d} x = \frac {\pi n} m \operatorname{csc} \frac {\pi n}{m}.
    \end{equation}
\end{problem_with_solution}

% SOLUTION
\textbf{Problem \ref{stack_110457}} -- podstawiamy $x = \tan^{2/m} \theta$, co prowadzi do całki
\begin{align}
    I & = \int_0^\infty \frac{x^{n-1}}{1 + x^m} \,\mathrm{d} x \\
      & = \int_0^{\pi/2} \frac 2 m \tan^{2n/m - 1} \theta \,\mathrm{d}\theta \\
      & = \frac 1 m \beta\left( \frac nm, 1 - \frac nm \right) \\
      & = \frac 1 m \Gamma \left(\frac nm\right) \Gamma \left(1 - \frac nm\right) \\
      & = \frac \pi m \operatorname{csc} \frac {\pi n}{m}.
\end{align}
% SOLUTION

%%

% https://math.stackexchange.com/questions/155941/evaluate-the-integral-int-01-frac-lnx1x21-mathrm-dx
\begin{problem_with_solution}[pytanie 155941]
    \label{stack_155941}%
    \begin{equation}
        \int_0^1 \frac{\log (1+x)}{1 + x^2} \,\mathrm{d}x = \frac \pi 8  \log 2.
    \end{equation}
\end{problem_with_solution}

% SOLUTION
\textbf{Problem \ref{stack_155941}} -- podstawiamy $x = \tan \theta$.
% SOLUTION

%%

% https://math.stackexchange.com/q/178790
\begin{problem}[pytanie 178790]
    \label{stack_178790}%
    \begin{equation}
        \int_0^{\pi/2} \frac{x^2}{x^2 + [\log (2 \cos x)]^2} \,\mathrm{d}x = \frac{\pi}{8} (1 - \gamma + \log (2 \pi)).
    \end{equation}
\end{problem}

%%

% https://math.stackexchange.com/questions/187729/evaluating-int-0-infty-sin-x2-dx-with-real-methods
\begin{problem_with_solution}[pytanie 187729, całka Fresnela]
    \label{stack_187729}%
    \begin{equation}
        I = \int_0^\infty \sin (x^2) \,\mathrm{d} x = \sqrt{\frac \pi 8}.
    \end{equation}
\end{problem_with_solution}

Całki Fresnela mają praktyczne zastosowanie, historycznie pierwszym było obliczenie natężenia pola elektromagnetycznego w środowisku, gdzie światło ugina się wokół nieprzezroczystych obiektów.
\index{całka Fresnela}%

% SOLUTION
\textbf{Problem \ref{stack_187729}} -- znajdziemy ogólniejszą całkę $I_\lambda$ funkcji $\sin(x^2) e^{-\lambda x^2}$ nad zbiorem $[0, \infty)$.
\begin{align}
    I_\lambda^2 & = \left(\int_0^\infty \sin(x^2) e^{-\lambda x^2} \,\mathrm{d}x \right)^2 \\
    & = \int_0^\infty \int_0^\infty \sin(x^2)\sin(y^2) e^{- \lambda(x^2+y^2)}\,\mathrm{d}y\,\mathrm{d}x \\
    & = \frac12 \int_0^\infty \int_0^\infty \left(\cos(x^2-y^2)-\cos(x^2+y^2)\right) e^{- \lambda(x^2+y^2)}\,\mathrm{d}y\,\mathrm{d}x \\
    & = \frac12 \int_0^{\pi/2} \int_0^\infty \left(\cos(r^2\cos(2\phi))-\cos(r^2)\right)e^{- \lambda r^2} \,r\,\mathrm{d}r\,\mathrm{d}\phi \\
    & = \frac14 \int_0^{\pi/2} \int_0^\infty \left(\cos(s\cos(2\phi))-\cos(s)\right) e^{- \lambda s} \,\mathrm{d}s\,\mathrm{d}\phi \\
    & = \frac14 \int_0^{\pi/2} \left( \frac{ \lambda}{\cos^2(2\phi)+ \lambda^2} - \frac{ \lambda}{1+ \lambda^2}\right)\,\mathrm{d}\phi \\
    & = \frac12 \int_0^{\pi/4} \frac{ \lambda\,\mathrm{d}\phi}{\cos^2(2\phi)+ \lambda^2} - \frac{ \lambda\pi/8}{1+ \lambda^2} \\
    & = \frac14 \int_0^{\pi/4} \frac{ \lambda\,\mathrm{d} \tan(2\phi)} {1+ \lambda^2+ \lambda^2 \tan^2(2\phi)} - \frac{ \lambda\pi/8}{1+ \lambda^2} \\
    & = \frac14 \int_0^\infty \frac1{1+ \lambda^2+t^2}\,\mathrm{d}t - \frac{ \lambda\pi/8}{1+ \lambda^2} \\
    & = \frac{\pi/8}{\sqrt{1+ \lambda^2}} - \frac{ \lambda\pi/8}{1+ \lambda^2}
\end{align}
% SOLUTION

%%

% https://math.stackexchange.com/questions/426325/evaluate-int-01-frac-log-left-1x2-sqrt3-right1x-mathrm-dx
\begin{problem}[pytanie 426325]
    \label{stack_426325}%
    \begin{equation}
        \int_0^1 \frac{\log \left(1 + x^{2 + \sqrt 3}\right)}{1 + x} \,\mathrm{d} x = \frac{\pi^2}{12} (1 - \sqrt 3) + \log (2) \log(1 + \sqrt 3).
    \end{equation}
\end{problem}

%%

% https://math.stackexchange.com/questions/464769/how-to-prove-int-01-tan-1-left-frac-tanh-1x-tan-1x-pi-tanh-1
\begin{problem}[pytanie 464769]
    \label{stack_464769}%
    \begin{equation}
        \int_0^1 \arctan \frac { \operatorname{artanh} x - \arctan x} {\pi + \operatorname{artanh} x - \arctan x}  \, \frac{\mathrm{d}x}{x} = \frac \pi 4 \log \frac{\pi}{2 \sqrt{2}}.
    \end{equation}
\end{problem}

%%

% https://math.stackexchange.com/questions/507425/an-integral-involving-airy-functions-int-0-infty-fracxp-operatornameai
\begin{problem_with_solution}[pytanie 507425]
    \label{stack_507425}%
    Niech
    \begin{align}
        \operatorname{Ai} (x) & = \frac 1 \pi \int_0^\infty \cos \left( x z + \frac {z^3} 3 \right) \,\mathrm{d}z, \\
        \operatorname{Bi} (x) & = \frac 1 \pi \int_0^\infty \sin \left( x z + \frac {z^3} 3 \right) + \exp \left( x z - \frac {z^3} 3 \right) \,\mathrm{d}z
    \end{align}
    będą funkcjami Airy'ego, zaś $n \in \mathbb N$ parametrem.
    Znaleźć
    \begin{equation}
        I_n = \int_0^\infty \frac{x^{3n} \,\mathrm{d} x}{(\operatorname{Ai} x)^2 + (\operatorname{Bi} x)^2}.
    \end{equation}
\end{problem_with_solution}

% SOLUTION
\textbf{Problem \ref{stack_507425}} -- $I_n = \pi^2 a_{2n} / (6 \cdot 2^{7n})$, gdzie $a_0 = 1$, $a_{n+1} = (6n+4)a_n \sum_{k=0}^n a_k a_{n-k}$.
% SOLUTION


%%

% https://math.stackexchange.com/questions/523027/a-math-contest-problem-int-01-ln-left1-frac-ln2x4-pi2-right-frac
\begin{problem}[pytanie 523027]
    \label{stack_523027}%
    \begin{equation}
        \int_0^1 \log\left(1+\left(\frac{\log x}{2\pi}\right)^2 \right)\frac{\log(1-x)}x \,\mathrm{d} x=-\pi^2\left(4\zeta'(-1)+\frac23\right).
    \end{equation}
\end{problem}

%%

% https://math.stackexchange.com/questions/541751/how-prove-this-i-int-0-infty-frac1x-ln-left-frac1x1-x-right2/541861#541861
\begin{problem}[pytanie 541751]
    \label{stack_541751}%
    \begin{equation}
        I = \int_0^\infty \frac{1}{x} \log \left(\frac{1+x}{1-x}\right)^2 \,\mathrm{d}x = \pi^2.
    \end{equation}
\end{problem}

%%

% https://math.stackexchange.com/q/562694
\begin{problem}[pytanie 562694]
    \label{stack_562694}%
    \begin{equation}
        \int_{-1}^1 \frac{1}{x} \sqrt{\frac{1+x}{1-x}} \log \frac{2x^2+2x+1}{2x^2-2x+1} \,\mathrm{d}x = 4 \pi \operatorname{arccot} \sqrt{\phi}.
    \end{equation}
\end{problem}

%%

\begin{problem}[pytanie 570997]
    \label{stack_570997}%
    \begin{equation}
        \int_0^1 \frac{\log (x + \sqrt 2)}{\sqrt{x(1-x)(2-x)}} \,\mathrm{d}x = \frac{\pi^{3/2}}{8\sqrt{2} \Gamma(3/4)^2 } \left(7 \log 2 + 4 \log (1 + \sqrt 2) - \pi \right).
    \end{equation}
\end{problem}

%%

% https://math.stackexchange.com/questions/815863/compute-int-0-pi-4-frac1-x2-ln1x21x2-1-x2-ln1-x21-x4
\begin{problem}[pytanie 815863]
    \label{stack_815863}%
    \begin{align}
        I & = \int_0^{\pi/4} \frac{ (1-x^2) [ \log(1+x^2) - \log(1 - x^2)] + 1 + x^2}{(1-x^4)(1+x^2)} x \exp \frac {x^2 - 1}{x^2 + 1} \,\mathrm{d} x \\
        & =  - \frac 1 4  \exp \frac{\pi^2 - 16}{\pi^2 + 16} \log \frac {16 - \pi^2}{16 + \pi^2}.
    \end{align}
\end{problem}

%%

% TODO: https://math.stackexchange.com/a/942440

%%

\begin{problem_with_solution}[pytanie 1582943]
    \label{stack_1582943}%
    \begin{equation}
        \int_0^\infty \left(\frac{\tanh x}{x}\right)^2 \,\mathrm{d}x = \frac{14 \zeta (3)}{\pi^2}.
    \end{equation}
\end{problem_with_solution}

% SOLUTION
\textbf{Problem \ref{stack_1582943}} -- dowodzimy najpierw, że poniższe całki są równe:
\begin{equation}
    \frac{\pi^2}{4} \int_0^\infty \frac{\tanh x \cdot \tanh xs}{x^2} \,\mathrm{d}x = s \int_0^1 \log \frac{1-x}{1+x} \log \frac{1-x^s}{1+x^s} \frac{\mathrm{d}x}{x}.
\end{equation}
% SOLUTION

%%

% https://math.stackexchange.com/q/1653979
\begin{problem}[pytanie 1653979]
    \label{stack_1653979}%
    % Niech $\phi = \frac 1 2 (1 + \sqrt 5)$ oznacza złotą liczbę.
    \begin{equation}
        \int_0^\infty \frac{5x^2}{1  + x^{10}} \,\mathrm{d}x = \frac{\pi}{\phi}.
    \end{equation}
\end{problem}

%%

% https://math.stackexchange.com/q/2529614
\begin{problem}[pytanie 2529614]
    Korzystając z technik analizy zespolonej pokazać, że
    \label{stack_2529614}%
    \begin{equation}
        \int_0^\infty \frac{x^{-\mathrm{i}a}}{x^2+bx+1} \,\mathrm{d}x = \frac{2\pi}{\sqrt{4-b^2}} \cdot \frac{\sinh (a \arccos (b/2))}{\sinh (a \pi)}.
    \end{equation}
\end{problem}

%%

% https://math.stackexchange.com/q/2826571
\begin{problem}[pytanie 2826571]
    \label{stack_2826571}%
    Niech $(m, n)$ oznacza największy wspólny dzielnik liczb $m, n$.
    Wtedy
    \begin{equation}
        \int_0^{\pi/2} \log \lvert \sin(mx) \rvert \cdot \log \lvert\sin(nx)\rvert \, dx = \frac{\pi^3}{24} \frac{(m,n)^2}{mn}+\frac{\pi\log (2)^2}{2}.
    \end{equation}
\end{problem}

%%

% https://math.stackexchange.com/questions/3490404/
\begin{problem}[pytanie 3490404]
    \label{stack_3490404}%
    Niech $f(x) = x \log (1 - \sin x) / \sin x$.
    Udowodnić, że
    \begin{align}
        2\int_0^{\pi/2} f(x) \,\mathrm{d}x =
        \int_{\pi/2}^\pi\ f(x) \,\mathrm{d}x,
    \end{align}
    bez uprzedniego znajdowania funkcji pierwotnej $f$.
\end{problem}

%%

% https://math.stackexchange.com/q/3981861
\begin{problem}[pytanie 3981861]
    Udowodnić lub podać kontrprzykład: niech $f \colon [0, 2\pi] \to [0, 2\pi]$ będzie różniczkowalną funkcją taką, że $f(0) = f(2\pi)$. Wtedy
    \label{stack_3981861}%
    \begin{equation}
        \left(\int_0^{2 \pi} \cos f(x) \,d x\right)^2
        +
        \left(\int_0^{2 \pi} \sqrt{(f'(x))^2+\sin^2 f(x)} \, \mathrm{d}x\right)^2 \ge (2 \pi)^2.
    \end{equation}
\end{problem}

%%

% https://math.stackexchange.com/q/4568778
\begin{problem}[pytanie 4568778]
    \label{stack_4568778}%
    \begin{equation}
        \int_0^1 \frac{\log (x+1) - \log(2x^2)}{\sqrt{x^2 + 2x}}\,\mathrm{d}x = \frac{\pi^2}{2}.
    \end{equation}
\end{problem}

%

\chapter{Rozwiązania}
% %

Czasami wystarczy zgadnąć wynik (albo znaleźć go w tablicy pochodnych) i sprawdzić, że pasuje przez zróżniczkowanie.
Na przykład wprost z faktu \ref{prp:derivative_power} wynika, że
\begin{problem}
    Niech $n \neq -1$.
    Wtedy
    \begin{equation}
        \int x^n \,\mathrm{d}x = \frac{x^{n+1}}{n+1}.
    \end{equation}
\end{problem}

%
\section{Całkowanie przez podstawianie}
% SOLUTION
\section{Całkowanie przez podstawianie}
% SOLUTION

% Banaś, Wędrychowicz, 12.18.
\begin{problem_with_solution}
    \label{banas_12_18}%
    \begin{equation}
        \int (\arcsin x)^2 \,\mathrm{d}x.
    \end{equation}
\end{problem_with_solution}

% SOLUTION
\textbf{Problem \ref{banas_12_18}}.
Podstawiamy $u = \arcsin x$ i dostajemy całkę z $u^2 \cos u$, którą rozwiązujemy przez części, tak jak w przykładzie \ref{banas_12_14}.
% SOLUTION

% Banaś, Wędrychowicz, 12.19.
\begin{problem_with_solution}
    \label{banas_12_19}%
    \begin{equation}
        \int \sin \log x \, \mathrm{d}x.
    \end{equation}
\end{problem_with_solution}

Analogicznie znajdujemy całkę funkcji $\cos \log x$.
% TODO: https://www.youtube.com/watch?v=q2043g_NogI

% SOLUTION
\textbf{Problem \ref{banas_12_19}}.
Podstawiamy $u = \log x$, $\mathrm{d} u = \mathrm{d} x / x$, $x = \exp u$ i dostajemy całkę z $e^u \sin u$, którą rozwiązujemy przez części, tak jak w przykładzie \ref{banas_12_19_auxilia}.
% SOLUTION

% Banaś, Wędrychowicz, 12.20.
% \begin{problem}
  %   \label{banas_12_20}%
    % \begin{equation}
      %   \int \cos(\log x) \, \mathrm{d}x.
    % \end{equation}
% \end{problem}

% Banaś, Wędrychowicz, 12.21.
\begin{problem_with_solution}
    \label{banas_12_21}%
    \begin{equation}
        \int \sqrt{x^2 + 1} \, \mathrm{d}x.
    \end{equation}
\end{problem_with_solution}

% SOLUTION
\textbf{Problem \ref{banas_12_21}}.
Podstawiamy $x = \tan \theta$.
Na mocy tożsamości trygonometrycznej $\tan^2 \theta + 1 = \sec^2 \theta$ nasza całka zmienia się w $\int \sec^2 \theta \cdot \sec \theta \,\mathrm{d}\theta$, czyli całkę z problemu \ref{banas_12_21_auxilia}.
Zatem
\begin{align}
    \int \sqrt{x^2 + 1} \, \mathrm{d}x & = \frac 12 \sec \theta \tan \theta + \log |\tan \theta + \sec \theta| \\
    & = \frac 1 2 x \sqrt{x^2 + 1} + \log \left|x + \sqrt{x^2+1}\right|.
\end{align}
% SOLUTION

% Banaś, Wędrychowicz, 12.30.
\begin{problem_with_solution}
    \label{banas_12_30}%
    \begin{equation}
        \int x \sqrt{a^2 - x^2} \, \mathrm{d}x.
    \end{equation}
\end{problem_with_solution}

% SOLUTION
\textbf{Problem \ref{banas_12_30}}.
Podstawić $u = a^2 - x^2$.
% SOLUTION

% Banaś, Wędrychowicz, 12.31.
\begin{problem_with_solution}
    \label{banas_12_31}%
    \begin{equation}
        \int \exp \sqrt x \, \mathrm{d}x.
    \end{equation}
\end{problem_with_solution}

% SOLUTION
\textbf{Problem \ref{banas_12_31}}.
Podstawić $u = \sqrt x$, następnie scałkować przez części: $f(u) = u$, $g'(u) = \exp u$.
% SOLUTION

% Banaś, Wędrychowicz, 12.33.
\begin{problem_with_solution}
    \label{banas_12_33}%
    \begin{equation}
        \int \tan x \, \mathrm{d}x.
    \end{equation}
\end{problem_with_solution}

Analogicznie znajdujemy całkę funkcji $\cot x$.

% SOLUTION
\textbf{Problem \ref{banas_12_33}}.
Podstawić $u = \sin x$.
% SOLUTION

% Banaś, Wędrychowicz, 12.34.
\begin{problem_with_solution}
    \label{banas_12_34}%
    \begin{equation}
        \int \frac{\mathrm{d}x}{x \log x}.
    \end{equation}
\end{problem_with_solution}

% SOLUTION
\textbf{Problem \ref{banas_12_34}}.
Podstawić $u = \log x$.
% SOLUTION

% Banaś, Wędrychowicz, 12.40.
\begin{problem_with_solution}
    \label{banas_12_40}%
    \begin{equation}
        \int \frac{x^3}{1+x^8} \, \mathrm{d}x.
    \end{equation}
\end{problem_with_solution}

% SOLUTION
\textbf{Problem \ref{banas_12_40}}.
Podstawiamy $u = x^4$ i dostajemy całkę funkcji $(1+x^2)^{-1}$, czyli ...
% TODO: dodać link jak już będzie gdzieś ta całka policzona
% SOLUTION

% Banaś, Wędrychowicz, 12.41.
\begin{problem_with_solution}
    \label{banas_12_41}%
    \begin{equation}
        \int \frac{x \,\mathrm{d}x}{\sqrt{1+x^4}}.
    \end{equation}
\end{problem_with_solution}

% SOLUTION
\textbf{Problem \ref{banas_12_41}}.
Podstawiamy $u = x^2$, a potem trygonometrycznie $\theta = \arctan u$.
% SOLUTION

% Banaś, Wędrychowicz, 12.55.
\begin{problem_with_solution}
    \label{banas_12_55}%
    \begin{equation}
        \int \frac{\mathrm{d}x}{\sqrt{1 + e^{2x}}}
    \end{equation}
\end{problem_with_solution}

% SOLUTION
\textbf{Problem \ref{banas_12_55}}.
Podstawić $u = 1 + e^{2x}$. % ?
% SOLUTION

% Banaś, Wędrychowicz, 12.57.
\begin{problem_with_solution}
    \label{banas_12_57}%
    \begin{equation}
        \int \frac{\sin x \cos x \, \mathrm{d}x}{\sqrt{(a \sin x)^2 + (b \cos x)^2}}.
    \end{equation}
\end{problem_with_solution}

% SOLUTION
\textbf{Problem \ref{banas_12_57}}.
Podstawić $u = (a \sin x)^2 + (b \cos x)^2$.
% SOLUTION

% Banaś, Wędrychowicz, 12.XX.
\begin{problem_with_solution}
    \label{banas_12_XX}%
    \begin{equation}
        \int \ldots \, \mathrm{d}x.
    \end{equation}
\end{problem_with_solution}

% SOLUTION
\textbf{Problem \ref{banas_12_XX}}.
% SOLUTION

% Banaś, Wędrychowicz, 12.XX.
\begin{problem_with_solution}
    \label{banas_12_XX}%
    \begin{equation}
        \int \ldots \, \mathrm{d}x.
    \end{equation}
\end{problem_with_solution}

% SOLUTION
\textbf{Problem \ref{banas_12_XX}}.
% SOLUTION

% Banaś, Wędrychowicz, 12.XX.
\begin{problem_with_solution}
    \label{banas_12_XX}%
    \begin{equation}
        \int \ldots \, \mathrm{d}x.
    \end{equation}
\end{problem_with_solution}

% SOLUTION
\textbf{Problem \ref{banas_12_XX}}.
% SOLUTION

% Banaś, Wędrychowicz, 12.XX.
\begin{problem_with_solution}
    \label{banas_12_XX}%
    \begin{equation}
        \int \ldots \, \mathrm{d}x.
    \end{equation}
\end{problem_with_solution}

% SOLUTION
\textbf{Problem \ref{banas_12_XX}}.
% SOLUTION

% Banaś, Wędrychowicz, 12.XX.
\begin{problem_with_solution}
    \label{banas_12_XX}%
    \begin{equation}
        \int \ldots \, \mathrm{d}x.
    \end{equation}
\end{problem_with_solution}

% SOLUTION
\textbf{Problem \ref{banas_12_XX}}.
% SOLUTION

% Banaś, Wędrychowicz, 12.XX.
\begin{problem_with_solution}
    \label{banas_12_XX}%
    \begin{equation}
        \int \ldots \, \mathrm{d}x.
    \end{equation}
\end{problem_with_solution}

% SOLUTION
\textbf{Problem \ref{banas_12_XX}}.
% SOLUTION

\begin{problem_with_solution}
    \label{nahin_1x_1x}%
    \begin{equation}
        \int_{-1}^1 \sqrt{\frac{1+x}{1-x}} \,\mathrm{d}x = \pi.
    \end{equation}
\end{problem_with_solution}

Znamy tę całkę z książki Nahina \cite{nahin15}.

% SOLUTION
\textbf{Problem \ref{nahin_1x_1x}} -- \cite[s. 115, 378]{nahin15}.
Wskazówka: podstawić $x = \cos 2 \varphi$.
% SOLUTION

% https://math.stackexchange.com/questions/170331/why-is-int-0-infty-frac-ln-x1x2-mathrmdx-0
\begin{problem_with_solution}
    \label{stack_170331}%
    \begin{equation}
        \int_0^\infty \frac{\log x}{1 + x^2} \,\mathrm{d}x.
    \end{equation}
\end{problem_with_solution}

% SOLUTION
\textbf{Problem \ref{stack_170331}} -- podstawić $\theta = \arctan x$, wtedy
\begin{equation}
    \int_0^{\pi/2} \log \tan \theta \,\mathrm{d}\theta = \int_0^{\pi/2} \log \sin \theta \,\mathrm{d}\theta - \int_0^{\pi/2} \log \cos \theta \,\mathrm{d}\theta = 0.
\end{equation}
% SOLUTION


\subsection{Podstawienia Eulera}

TODO: Banaś Wędrychowicz, 12.71 - 12.87

\begin{problem}
    Banaś-Wędrychowicz, 12.58.
\end{problem}

%
%

\section{Całkowanie przez części}

\begin{proposition}[wzór na całkowanie przez części]
\label{prp_int_by_parts}%
    Jeśli funkcje $f, g \colon I \to \R$ są różniczkowalne, to
    \begin{equation}
        \int f(x) g'(x) \,\mathrm{d}x = f(x) g(x) - \int f'(x) g(x) \,\mathrm{d} x.
    \end{equation}
\end{proposition}

\begin{proof}
    Najpierw całkujemy obie strony wzoru na pochodną iloczynu $(fg)' = fg' + f'g$, a następnie porządkujemy strony równości.
\end{proof}

% Banaś, Wędrychowicz: 12.1
\begin{problem_with_solution}
    \label{banas_12_1}%
    \begin{equation}
        \int x \sin x \,\mathrm{d} x.
    \end{equation}
\end{problem_with_solution}

% SOLUTION
\textbf{Problem \ref{banas_12_1}}.
Całkujemy przez części, $f(x) = x$, $g'(x) = \sin x$.
\begin{align}
    \int x \sin x \,\mathrm{d} x & = -x \cos x - \int - \cos x \, \mathrm{d}x \\
                                 & = -x \cos x + \sin x.
\end{align}
% SOLUTION

Analogicznie obliczamy całki funkcji $x \cos x$ albo $x \exp x$.

% Banaś, Wędrychowicz: 12.6
\begin{problem_with_solution}
    \label{banas_12_6}%
    \begin{equation}
        \int x \arctan x \,\mathrm{d} x.
    \end{equation}
\end{problem_with_solution}

% SOLUTION
\textbf{Problem \ref{banas_12_6}}.
Całkujemy przez części, $f(x) = \arctan x$, $g'(x) = x$.
\begin{align}
    \int x \arctan x \, \mathrm{d} x
    & = \frac 12 x^2 \arctan x - \int \frac{x^2 \,\mathrm{d}x}{2(x^2+1)} \\
    & = \frac 12 x^2 \arctan x - \frac 12 \left(\int 1 \,\mathrm{d}x - \int \frac{\mathrm{d}x}{x^2+1} \right) \\
    & = \frac 12 x^2 \arctan x - \frac 12 \left(x - \arctan x \right) \\
    & = \frac 12 \left((x^2+1)\arctan x - x \right).
\end{align}
% SOLUTION

% Banaś, Wędrychowicz: 12.7
\begin{problem_with_solution}
    \label{banas_12_7}%
    \begin{equation}
        \int x^n \log x \,\mathrm{d} x.
    \end{equation}
\end{problem_with_solution}

% SOLUTION
\textbf{Problem \ref{banas_12_7}}.
Całkujemy przez części, $f(x) = \log x$, $g'(x) = x^n$.
\begin{align}
    \int x^n \log x \, \mathrm{d} x & = \frac{x^{n+1} \log x}{n+1} - \int \frac{x^n \,\mathrm{d} x}{n+1} \\
                                    & = \frac{x^{n+1} \log x}{n+1} - \frac{x^{n+1}}{(n+1)^2}.
\end{align}
% SOLUTION

% Banaś, Wędrychowicz: 12.8
\begin{problem_with_solution}
    \label{banas_12_8}%
    \begin{equation}
        \int \arccos x \,\mathrm{d} x.
    \end{equation}
\end{problem_with_solution}

% SOLUTION
\textbf{Problem \ref{banas_12_8}}.
Całkujemy najpierw przez części, $f(x) = \arccos x$, $g'(x) = 1$, żeby następnie podstawić $u = 1 - x^2$, $\mathrm{d} u = -2x \mathrm{d}x$:
\begin{align}
    \int \arccos x \, \mathrm{d} x & = x \arccos x - \int  \frac{-x \,\mathrm{d}x}{\sqrt{1-x^2}} \\
    & = x \arccos x - \frac 12 \int \frac {\mathrm{d}u}{\sqrt{u}} \\
    & = x \arccos x - \sqrt{1 - x^2}.
\end{align}
% SOLUTION

Analogicznie obliczamy całkę funkcji $\arcsin x$.

% Banaś, Wędrychowicz: 12.10
\begin{problem_with_solution}
    \label{banas_12_10}%
    \begin{equation}
        \int x \cdot (\tan x)^2 \,\mathrm{d} x.
    \end{equation}
\end{problem_with_solution}

% SOLUTION
\textbf{Problem \ref{banas_12_10}}.
Całkujemy przez części, $f(x) = x$, $g'(x) = (\tan x)^2$.
\begin{align}
    \int x \cdot (\tan x)^2 x \, \mathrm{d} x & = x (\tan x - x) - \int (\tan x - x) \,\mathrm{d}x \\
    & = x (\tan x - x) - \left(-\log(\cos(x)) - \frac{x^2}{2}\right).
\end{align}
% SOLUTION

% Banaś, Wędrychowicz: 12.11
\begin{problem_with_solution}
    \label{banas_12_11}%
    \begin{equation}
        \int x \cdot (\cos x)^2 \,\mathrm{d} x.
    \end{equation}
\end{problem_with_solution}

% SOLUTION
\textbf{Problem \ref{banas_12_11}}.
Ponieważ $\cos 2x = 2 \cos^2 x - 1$, potrzebujemy znaleźć prostszą całkę 
\begin{align}
    \int x \cos 2x \, \mathrm{d} x.
\end{align}
Całkujemy przez części: $f(x) = x$, $g'(x) = \cos 2x$, co prowadzi do jeszce prostszej całki funkcji $\sin 2x$.
Ostatecznie
\begin{align}
    \int x \cos 2x \, \mathrm{d} x = \frac 1 8 \left(2x^2 + 2x \sin 2x + \cos 2x\right).
\end{align}
% SOLUTION

% Banaś, Wędrychowicz, 12.12 to całka z x log(x^2+1), ale tam wystarczy podstawić u = x^2 + 1, wtedy du = 2x dx.
% Banaś, Wędrychowicz, 12.16
% Banaś, Wędrychowicz, 12.17.
\begin{problem_with_solution}
    \label{banas_12_12}%
    \begin{equation}
        \int (\log x)^n \,\mathrm{d}x.
    \end{equation}
\end{problem_with_solution}

% SOLUTION
\textbf{Problem \ref{banas_12_12}}.
Całkujemy przez części, $f(x) = (\log x)^n$, $g'(x) = 1$.
Dostajemy początek rekurencji:
\begin{equation}
    \int (\log x)^n \, \mathrm{d}x = x (\log x)^n - n \int (\log x)^{n-1} \,\mathrm{d} x.
\end{equation}
z warunkiem brzegowym:
\begin{equation}
    \int \log x\, \mathrm{d}x = x\log x - x.
\end{equation}
% SOLUTION

% Banaś, Wędrychowicz, 12.13.
\begin{problem_with_solution}
    \label{banas_12_13}%
    \begin{equation}
        \int x^n e^x \,\mathrm{d} x.
    \end{equation}
\end{problem_with_solution}

% SOLUTION
\textbf{Problem \ref{banas_12_13}}.
Dowiedziemy tego indukcyjnie.
Dla $n = 0$, całka jest elementarna.
Jeżeli $n \ge 1$, to całkujemy przez części: $f(x) = x^n$, $g'(x) = e^x$ i dostajemy zależność rekurencyjną
\begin{equation}
    I_n = x^n e^x - nI_{n-1},
\end{equation}
której rozwiązaniem jest
\begin{equation}
    I_n = e^x \sum_{k=0}^n (-1)^{n-k} \frac{n!}{k!}x^k.
\end{equation}.
% SOLUTION

% Banaś, Wędrychowicz, 12.14.
% Banaś, Wędrychowicz, 12.15.
% Banaś, Wędrychowicz, 12.25.
\begin{problem_with_solution}
    \label{banas_12_14}%
    \begin{equation}
        \int x^3 \sin x \, \mathrm{d}x.
    \end{equation}
\end{problem_with_solution}

% SOLUTION
\textbf{Problem \ref{banas_12_14}}.
Całkujemy przez części, $f(x) = x^3$, $g'(x) = \sin x$.
Dostajemy początek rekurencji:
\begin{equation}
    \int x^3 \sin x \, \mathrm{d}x = - x^3 \cos x - \int - 3x^2 \cos x \,\mathrm{d}x
\end{equation}
rozwiązaniem której jest $3 (x^2-2) \sin x + x (6-x^2) \cos x$.
% SOLUTION

\begin{problem}
    % pomocnicza dla Banaś 12.19
    \label{banas_12_19_auxilia}%
    \begin{equation}
        \int e^x \sin x \,\mathrm{d}x = \frac {e^x} 2 (\sin x - \cos x).
    \end{equation}
\end{problem}

% Banaś, Wędrychowicz, 12.22.
% \begin{problem}
% Banaś-Wędrychowicz, 12.22. % x^2 e^x sin x
% \end{problem}

% Banaś, Wędrychowicz, 12.23.
% \begin{problem}
% Banaś-Wędrychowicz, 12.23. % x / (sin ^2 x)
% \end{problem}

% Banaś, Wędrychowicz, 12.24.
% \begin{problem}
% Banaś-Wędrychowicz, 12.24. % x arcsin x / (1 - x^2)
% \end{problem}

% Banaś, Wędrychowicz, 12.26.
\begin{problem_with_solution}
    \label{banas_12_26}%
    \begin{equation}
        \int \frac{x \log(\sqrt{x^2+1}+x)}{\sqrt{x^2+1}} \,\mathrm{d}x.
    \end{equation}
\end{problem_with_solution}

% SOLUTION
\textbf{Problem \ref{banas_12_26}}.
Zauważamy, że $\log(\sqrt{x^2+1} + x) = \arsinh x$, a następnie całkujemy przez części: $f(x) = \arsinh x$, $g'(x) = x / \sqrt{x^2+1}$, wtedy $f(x) = 1/\sqrt{x^2+1}$, $g(x) = \sqrt{x^2+1}$.
Ostateczny wynik to $\sqrt{x^2+1} \arsinh x - x$.
% SOLUTION

% Banaś, Wędrychowicz, 12.27.
\begin{problem_with_solution}
    \label{banas_12_27}%
    \begin{equation}
        \int \arctan \sqrt{x} \,\mathrm{d}x.
    \end{equation}
\end{problem_with_solution}

% SOLUTION
\textbf{Problem \ref{banas_12_27}}.
Całkujemy przez części: $f(x) = \arctan \sqrt x$, $g'(x) = 1$, potem podstawiamy $u = \sqrt{x}$, co prowadzi do klasycznej całki funkcji $(u^2 + 1)^{-1}$.
% TODO: gdzie jest ta klasyczna całka?
% SOLUTION

\begin{problem_with_solution}
    \label{banas_12_21_auxilia}%
    \begin{equation}
        \int \sec^3 x \,\mathrm{d}x.
    \end{equation}
\end{problem_with_solution}

% SOLUTION
\textbf{Problem \ref{banas_12_21_auxilia}}.
Całkujemy najpierw przez części, $f(x) = \sec x$, $g'(x) = \sec^2 x$.
Pamiętając, że $\tan^2 x = \sec^2 x - 1$, mamy
\begin{align}
    I & = \sec x \tan x - \int \sec x \tan^2 x \,\mathrm{d} x \\
        & = \sec x \tan x - I + \int \sec x \,\mathrm{d}x \\
    2I & = \sec x \tan x + \log |\tan x + \sec x|.
\end{align}
% SOLUTION

%

\section{Całkowanie funkcji wymiernych}
% SOLUTION
\section{Całkowanie funkcji wymiernych}
% SOLUTION

\begin{problem_with_solution}
    \label{banas_12_100}%
\begin{equation}
    \int \frac{\mathrm{d}x}{x^3 + x}.
\end{equation}
\end{problem_with_solution}

% SOLUTION
\textbf{Problem \ref{banas_12_100}} -- wyłączyć z mianownika $x^3$, podstawić $u = 1 + x^{-2}$ i znaleźć całkę z funkcji $1/u$.
Alternatywnie, rozłożyć na ułamki proste i dopiero potem podstawiać.
Wynik to
\begin{equation}
    I_{B100} = \log x - \frac 12 \log (x^2 + 1).
\end{equation}
% SOLUTION

\begin{problem_with_solution}
    \label{banas_12_118}%
\begin{equation}
    \int \frac{3x^3 - 8x + 5}{\sqrt{x^2 - 4x - 7}} \,\mathrm{d}x.
\end{equation}
\end{problem_with_solution}

% SOLUTION
\textbf{Problem \ref{banas_12_118}} -- podstawić $u = x - 2$, a potem $w = \sqrt{11} \sec u$.
To prowadzi do całki, która jest kombinacją liniową potęg sekansa ($\sec, \sec^2, \sec^3, \sec^4$).
Wynik po odwróceniu podstawień to
\begin{equation}
    \sqrt{x^2 - 4x - 7} (x^2 + 5x + 36) + 112 \log \left(2 - x - \sqrt{x^2 - 4x-7}\right).
\end{equation}
% SOLUTION














\begin{problem}
\label{boros_4287}%
\begin{equation}
    \int_0^\infty \frac{x^n \,\mathrm{d}x}{(ax+b)^{m+1}}  = \frac{(-1)^{n+1} (-1-m)! \cdot n!}{a^{n+1} b^{m-n} (n-m)!}
\end{equation}
\end{problem}

% SOLUTION
\textbf{Problem \ref{boros_4287}} -- patrz \cite[s. 48-60]{boros04}
% SOLUTION

\begin{problem}
\label{frac_1_x3_1}%
\begin{equation}
    \int_0^\infty \frac{\mathrm{d}x}{x^3 - 1} = - \frac{\pi}{3\sqrt{3}}
\end{equation}
\end{problem}

% SOLUTION
\textbf{Problem \ref{frac_1_x3_1}} -- patrz \cite[s. 22]{nahin15}
% SOLUTION

\begin{problem}
    \label{experimental_mathematics_p258}%
\begin{equation}
    \int_0^\infty \frac{x^8-4x^6+9x^4-5x^2+1}{x^{12}-10x^{10}+37x^8-42x^6+26x^4-8x^2+1} \,\mathrm{d}x = \frac{\pi}{2}
\end{equation}
\end{problem}

% SOLUTION
\textbf{Problem \ref{experimental_mathematics_p258}} -- patrz \cite[s. 258]{bailey07}.
% SOLUTION

Rozdział 3 Borosa i Molla (TODO), rozdział 6. Całki postaci zostawia jako teren do eksploracji dla Czytelnika, a sami rozważają całki postaci
\begin{equation}
    \int_0^\infty \frac{P(x) \,\mathrm{d}x}{(q_4 x^4 + q_2 x^2 + q_0)^{m+1}},
\end{equation}
gdzie $m \in \mathbb N$, zaś $P$ jest wielomianem stopnia co najwyżej $4m + 2$.
Przykładowe zadania: \ref{boros_moll_7_1_1}, \ref{boros_moll_7_2_3}.

\subsection{Rozkład na ułamki proste}
% SOLUTION
\subsection{Rozkład na ułamki proste}
% SOLUTION

\begin{problem_with_solution}
    \label{boros_moll_7_1_1}
\begin{equation}
    \int_0^\infty \frac{3x^3 \,\mathrm{d}x}{(x^4 + 4x^2 + 1)^5} = \frac{5}{20736} \cdot \left(60 + 7 \sqrt{3} \log \left(7 - 4 \sqrt {3}\right)\right).
\end{equation}
\end{problem_with_solution}

% SOLUTION
\textbf{Problem \ref{boros_moll_7_1_1}} -- patrz \cite[s. 138]{boros04}.
% SOLUTION

\begin{problem_with_solution}
\label{nahin_page_225}%
    \begin{equation}
		\int_1^\infty \frac{\mathrm{d}x}{x^3 + x} = \log \sqrt {2}.
    \end{equation}
\end{problem_with_solution}

% SOLUTION
\textbf{Problem \ref{nahin_page_225}}.
Rozkładamy całkowaną funkcję na sumę
\begin{equation}
	\frac 1 x - \frac{1}{2(x - i)} - \frac{1}{2(x+i)}
\end{equation}
tak jak Nahin \cite[s. 225]{nahin15}.
% SOLUTION

%
%

\section{Całkowanie funkcji trygonometrycznych}
% SOLUTION
\section{Całkowanie funkcji trygonometrycznych}
% SOLUTION

\begin{problem_with_solution}
    \label{banas_12_160}%
\begin{equation}
    \int \frac{\cos x}{\sin^3 x - \cos^3 x} \, \mathrm{d} x.
\end{equation}
\end{problem_with_solution}

% SOLUTION
\textbf{Problem \ref{banas_12_160}} -- podstawić $u = \tan x$, co prowadzi do całki z $1/(u^3-1)$ i wyniku
\begin{equation}
    -\frac {1}{\sqrt{3}} \arctan \left( \frac{1 + 2 \tan x}{\sqrt{3}} \right) 
    + 2 \log (\cos x - \sin x) 
    - \log (2 + \sin 2x)
\end{equation}
% SOLUTION

%

\section{Całki różne, różniste}
% SOLUTION
\section{Całki różne, różniste}
% SOLUTION

% https://en.wikipedia.org/wiki/Integral_of_the_secant_function#History
Gerardus Mercator\footnote{Flamandzki matematyk i geograf.} opublikował \emph{,,Nova et Aucta Orbis Terrae Descriptio ad Usum Navigantium Emendata''}, czyli mapę świata, w 1569 roku.
Do jej sporządzenia wykorzystał nowe odwzorowanie Ziemi, zwane obecnie także walcowym równokątnym.
\index{odwzorowanie Mercatora}%
Niestety nie przedstawił nigdzie swoich obliczeń!
Usterkę tę naprawił dopiero Edward Wright w roku 1599, natrafiając jednocześnie na problem:
\index[persons]{Wright, Edward}

\begin{problem}
    \begin{equation}
        \int \sec x \,\mathrm{d} x = \log {| \sec x + \tan x|}.
    \end{equation}
\end{problem}

Czas mijał (około czterdzieści lat), aż Henry Bond\footnote{Nauczyciel nawigacji, geodezji oraz innych matematycznych rzeczy.} porównał zawartość tablic logarytmicznych oraz wyniki Wrighta, co doprowadziło do postawienia hipotezy, jaka jest zwarta postać całki:
\begin{equation}
    \log \tan \left(\frac \pi 4 + \frac x 2\right).
\end{equation}
\index[persons]{Bond, Henry}%
Potwierdzenie nie przyszło od razu; pierwszy był James Gregory\footnote{Szkocki matematyk i astronom.} w pracy \emph{,,Exercitationes Geometricae''} z 1668 roku, ale było tak trudne w zrozumieniu (chociaż poprawne; podstawił $u = \sec x + \tan x$), że Isaac Barrow\footnote{Angielski teolog i matematyk.} zaproponował w \emph{,,Lectiones Geometricae''} z 1670 roku zupełnie inne podejście.
\index[persons]{Gregory, James}%
\index[persons]{Barrow, Isaac}%
Dowód Barrowa jest znany przede wszystkim dlatego, że zawiera najstarszy znany rozkład na ułamki proste podczas całkowania (!) po podstawieniu $u = \sin x$:
\index{rozkład na ułamki proste}%
\begin{align}
    \int \sec x \,\mathrm{d}x = \int \frac{\cos x}{1-\sin^2 x}\,\mathrm{d}x = \int \frac{\mathrm{d}u}{1-u^2} = \frac 1 2 \int \frac{\mathrm{d}u}{1 + u} + \frac 1 2 \int \frac{\mathrm{d}u}{1 - u} = \ldots
\end{align}
Do tego samego rozkładu ułamków prowadzi podstawienie trygonometryczne $t = \tan (x/2)$.

% TODO: https://math.stackexchange.com/questions/9286/evaluation-of-gaussian-integral-int-0-infty-mathrme-x2-dx
% https://math.stackexchange.com/a/9292/1298830
\begin{problem}[całka Gaußa, całka Eulera-Poissona]
    \label{gauss_euler_poisson}%
    \begin{equation}
        \int_{-\infty}^\infty \exp \left( -x^2 \right) \,\mathrm{d} x = \sqrt{\pi}.
    \end{equation}
\end{problem}
% TODO: https://www.youtube.com/watch?v=6hI1ij6l8e4

Wiemy, że Abraham de Moivre myślał o całkach podobnych do powyższej około 1733 roku; być może ktoś inny go ubiegł, ale nie pozostawił po sobie żadnego tropu.
\index[persons]{de Moivre, Abraham}%
Dokładny wynik podał dopiero Gauß na początku następnego wieku (rok 1809); wspomniał przy tym, że jest on zasługą Laplace'a.
\index[persons]{Gauss, Carl Friedrich}%

Oznaczmy szukaną całkę przez $I$.
Laplace pokazał przy użyciu twierdzenia Fubiniego, że:
\begin{align}
    I^2 & = 4 \int_0^\infty \int_0^\infty \exp(-x^2 -y^2)\,\mathrm{d}y \,\mathrm{d}x \\
    & = 4 \int_0^\infty \int_0^\infty x \exp [-x^2(1+s^2)]\,\mathrm{d}s \,\mathrm{d}x \\
    & = 4 \int_0^\infty \int_0^\infty x \exp [-x^2(1+s^2)]\,\mathrm{d}x \,\mathrm{d}s \\
    & = 4 \int_0^\infty \left. \frac{\exp(-x^2(1+s^2))}{-2(1+s^2)}\right|_{0}^{\infty} \,\mathrm{d}s \\
    & = 2 \int_0^\infty \frac{\mathrm{d}s}{1+s^2} \\
    & = 2 \left. \arctan s \right|_{0}^\infty = \pi.
\end{align}

Obecnie preferowana jest metoda Poissona, który zauważył, że można zrobić to samo, co wyżej (skoro całkowana funkcja $\exp (\ldots)$ jest nieujemna, to całka jest równa pierwiastkowi ze swojego kwadratu), ale wykorzystując po drodze współrzędne biegunowe:
\index[persons]{Poisson, ?}%
\begin{align}
    I^2 & = \left(\int_{-\infty}^\infty \exp \left( -x^2 \right) \,\mathrm{d}x\right)\left(\int_{-\infty}^\infty \exp \left( -y^2 \right) \,\mathrm{d}y\right) \\
    & = \int_{-\infty}^\infty \int_{-\infty}^\infty \exp \left(-x^2-y^2\right) \,\mathrm{d}x\,\mathrm{d}y \\
    & = \int_0^{2\pi} \int_0^\infty r \exp (-r^2) \,\mathrm{d}r\, \mathrm{d}\theta \\
    & = 2\pi \int_0^\infty r \exp (-r^2) \,\mathrm{d} r \\
    & = 2\pi \int^0_{-\infty} \frac 1 2 \exp s \,\mathrm{d} s \\
    & = \pi.
\end{align}
Co ciekawe, opisana technika nie działa wobec jakiejkolwiek innej całki! Patrz notka Dawsona \cite{dawson05}.

Algorytm Rischa pokazuje, że całka nieoznaczona tej samej funkcji, $\exp (-x^2)$, nie daje się wyrazić przez funkcje elementarne.
Natomiast całkę pozornie bardziej skomplikowanej funkcji $x \exp (-x^2)$ można szybko znaleźć podstawiając $u = x^2$, a jeśli w naszym arsenale jest jeszcze różniczkowanie pod znakiem całki, to wykażemy prawie tak samo szybko, że
\begin{equation}
    \int_{-\infty}^\infty x^{2n} \exp (-x^2) \,\mathrm{d}x = \frac{(2n-1)!!}{2^n} \sqrt \pi.
\end{equation}

% https://math.stackexchange.com/questions/34767/int-infty-infty-e-x2-dx-with-complex-analysis
Inne rozwiązanie zaczyna się od podstawienia $u = x^2$.
Wtedy na mocy wzoru odbiciowego Eulera $\Gamma (z) \Gamma(1-z) = \pi/\sin \pi z$ mamy:
\begin{align}
	I = \int_0^\infty u^{-1/2} e^{-u}\,\mathrm{d}u = \Gamma \left(\frac 12\right) = \sqrt{\pi}.
\end{align}
(Jeszcze więcej rozwiązań dostarcza dyskusja pod pytaniem nr 34767 w portalu Math Stackexchange).




\begin{problem}[sinus całkowy]
    \begin{equation}
        \int_0^\infty \frac {\sin x}{x} \,\mathrm{d} x = \frac \pi 2.
    \end{equation}
\end{problem}

% TODO https://www.youtube.com/watch?v=XbgiqCCWfSE
% https://www.youtube.com/watch?v=zet9sRvEazg


Sinus (i kosinus) całkowy pojawiają się także w całkach:
\begin{align}
    I_1 & = \int_0^\infty \frac{\sin mx}{x + b} \,\mathrm{d}x, \\
    I_2 & = \int_0^\infty \frac{\sin mx}{ax^2 + bx + c} \,\mathrm{d}x,
\end{align}
jak panowie Boros, Moll \cite[s. 136]{boros04} napisali.

% TODO: https://math.stackexchange.com/questions/13344/proof-of-int-0-infty-left-frac-sin-xx-right2-mathrm-dx-frac-pi2
\begin{problem}
    \begin{equation}
        \int_0^\infty \left(\frac {\sin x}{x}\right)^2 \,\mathrm{d} x = \frac \pi 2.
    \end{equation}
\end{problem}

\subsection{Funkcje Gamma i Beta}
% SOLUTION
\subsection{Funkcje Gamma i Beta}
% SOLUTION
\begin{problem}
\label{exp_x_3_gamma_4_3}%
    \begin{equation}
        I = \int_{0}^\infty \exp \left( -x^3 \right) \,\mathrm{d} x = \Gamma \left( \frac 4 3 \right).
    \end{equation}
\end{problem}

Patrz też \cite[s. 119]{nahin15}.

% SOLUTION
\textbf{Problem \ref{exp_x_3_gamma_4_3}} -- podstawiamy $y = x^3$, wtedy $\mathrm{d}x = \frac 1 3 y^{-2/3} \mathrm{d}y$
% SOLUTION

\begin{problem}
\label{wallis_x_x2_n}%
    \begin{equation}
        I_n = \int_{0}^1 (x-x^2)^n \,\mathrm{d} x = \frac{(n!)^2}{(2n+1)!}
    \end{equation}
\end{problem}

Około 1650 roku John Wallis badał tę całkę i zgadł jej wartość dla dowolnego $n \in \N$ na podstawie wartości dla małych wartości.
Patrz też \cite[s. 119-122]{nahin15}.

% SOLUTION
\textbf{Problem \ref{wallis_x_x2_n}} -- zauważamy, że całkę Wallisa można wyrazić przez funkcję Beta:
\begin{align}
	I_n = \int_{0}^1 x^n (1-x)^n \,\mathrm{d} x = B(n+1, n+1) = \frac{\Gamma(n+1) \Gamma(n+1)}{\Gamma(2n+2)}.
\end{align}
% SOLUTION

Po podstawieniu $n = 1/2$ w problemie \ref{wallis_x_x2_n} widzimy, że $I_{1/2} = \frac 1 2 \Gamma(3/2)^2$.
Ale tę samą całkę można znaleźć wykorzystując geometryczną interpretację całki jako pola powierzchni pod wykresem funkcji.
Łatwo widać, że wykres to górny półokrąg o środku w punkcie $(1/2, 0)$ i promieniu $(1/2)$, zatem szukane pole to $\frac \pi 8$, i stąd wynika już, że $\Gamma(3/2) = \sqrt{\pi}/2$.
Teraz wystarczy wstawić uzyskane wyniki do definicji funkcji Gamma i odkryć jak \cite[s. 123]{nahin15}, że
\begin{equation}
    \int_{0}^\infty \exp(-x) \sqrt{x} \,\mathrm{d} x = \frac{\sqrt \pi}{2}.
\end{equation}

\begin{problem_with_solution}
\label{sqrt_minus_log}%
    \begin{equation}
        \int_{0}^1 \sqrt{- \log x} \,\mathrm{d} x = \frac{\sqrt \pi}{2}.
    \end{equation}
\end{problem_with_solution}

% SOLUTION
\textbf{Problem \ref{sqrt_minus_log}} -- podstawiamy $y = - \log x$.
% SOLUTION

\begin{problem_with_solution}
    \label{boros_moll_p95}%
    \begin{equation}
        \int_0^\infty x^n \exp (-px) \,\mathrm{d}x = \frac{n!}{p^{n+1}}.
    \end{equation}
\end{problem_with_solution}

Rodzina wszystkich kombinacji liniowych funkcji postaci $x^n \exp (-x)$ jest zamknięta na branie całek nieoznaczonych, o~pokazanie tego proszą Boros i Moll \cite[s. 103]{boros04}.
Podobnie jest dla całek funkcji postaci $x^n \sin(x)^m$, patrz \cite[s. 135]{boros04} oraz funkcji postaci  $x^n \sin(x)^m$, patrz  \cite[s. 136]{boros04}. % TODO przepisać te całki?

% SOLUTION
\textbf{Problem \ref{boros_moll_p95}} -- zróżniczkować $n$ razy względem $p$, jak Boros, Moll \cite[s. 95]{boros04}.
% SOLUTION

\begin{problem_with_solution}
    \label{boros_moll_p97}%
    \begin{equation}
        \int_0^1 x^n \log^k x \,\mathrm{d}x = \frac{(-1)^k \cdot k!}{(n+1)^{k+1}}.
    \end{equation}
\end{problem_with_solution}

% SOLUTION
\textbf{Problem \ref{boros_moll_p97}} -- zróżniczkować całkę z $x^n$ nad $[0, 1]$ względem $n$ jak Boros, Moll \cite[s. 97]{boros04}.
% SOLUTION

Po zamianie zmiennych dostajemy:

\begin{problem}
    \begin{equation}
        \int_0^\infty x^k \exp (-x) \,\mathrm{d}x = k!.
    \end{equation}
\end{problem}

%%%%%%%%%%%%%%%%%

\begin{problem_with_solution}
    Niech $P$ będzie wielomianem stopnia $2m$. Znaleźć
    \label{boros_moll_p105}%
    \begin{equation}
        \int_0^\infty \frac{P(x) \,\mathrm{d}x}{(ax^2 + bx + c)^{m+1}}
    \end{equation}
\end{problem_with_solution}

Szczególny przypadek ($P(x) \equiv 1$, $a = c = 1$, $b = 0$) został rozprawiony się z nim w 1656 przez Wallisa.

% SOLUTION
\textbf{Problem \ref{boros_moll_p105}} -- Boros, Moll \cite[s. 105--koniecrozdziału6]{boros04}.
% SOLUTION
% %

\section{Całkowanie różniczek dwumiennych} % https://encyclopediaofmath.org/wiki/Differential_binomial
Całkowanie różniczek dwumiennych

%
% \section{Całkowanie funkcji trygonometrycznych}
% Całkowanie funkcji trygonometrycznych
%

\section{Sztuczka Feynmana: różniczkowanie pod znakiem całki}
% SOLUTION
\section{Sztuczka Feynmana: różniczkowanie pod znakiem całki}
% SOLUTION

% https://math.stackexchange.com/questions/942263/really-advanced-techniques-of-integration-definite-or-indefinite
\begin{problem}
    \begin{equation}
        \int_0^\infty \frac{\sin x}{x} \,\mathrm{d}x = \frac \pi 2.
    \end{equation}
\end{problem}

% TODO: przepisać całkę z s. 82, Nahin

\begin{problem_with_solution}
    \label{nahin_holzweg}%
    Niech $a, b > 0$, wtedy
    \begin{equation}
        \int_{-\infty}^\infty \frac{\cos ax}{b^2 - x^4} \,\mathrm{d} x = \frac{\pi}{b} \sin (ab).
    \end{equation}
\end{problem_with_solution}

% SOLUTION
\textbf{Problem \ref{nahin_holzweg}} -- \cite[s. 115, 375, 376]{nahin15}.
% SOLUTION

\begin{problem}
    \label{nahin_datenautobahn}%
    Niech $a > b$, wtedy
    \begin{equation}
        \int_{-\infty}^\infty \frac{\cos ax}{b^4 - x^4} \,\mathrm{d} x = \frac{\pi}{2b^3} [\sin (ab) + \exp (-ab)].
    \end{equation}
\end{problem}

% SOLUTION
\textbf{Problem \ref{nahin_datenautobahn}} -- \cite[s. 115, 376]{nahin15}.
% SOLUTION

% Nahin Inside interesting... page 83
\begin{problem}
    \begin{equation}
        \int_0^\infty \frac{\sin ax}{x e^{xy}} \,\mathrm{d}x = \pm \frac \pi 2 - \arctan \frac y a.
    \end{equation}
\end{problem}

\begin{problem}
    Niech $a > 0$, wtedy
    \begin{equation}
        \int_0^\infty \frac{\sin ax}{x} \,\mathrm{d}x = \frac \pi 2.
    \end{equation}
\end{problem}
% SKAD TO


\begin{problem}[całka Frullaniego]
\index{całka Frullaniego}%
    Niech $f \colon [0, \infty) \to \R$ będzie funkcją ciągle różniczkowalną, której granica w nieskończoności istnieje.
    Wtedy dla ustalonych liczb rzeczywistych $a, b > 0$ mamy
    \begin{equation}
        \int_0^\infty \frac{f(ax) - f(bx)}{x} \,\mathrm{d} x = \left[\lim_{x \to \infty} f(x) - f(0) \right] \cdot \log \frac a b.
    \end{equation}
\end{problem}

Okazuje się, że problem rozwiązał Cauchy (około 1823 roku), ale też Giuliano Frullani\footnote{Włoski matematyk.} (zapowiedź w 1821 roku, publikacja około 1829 roku).
\index[persons]{Frullani, Giuliano}%
Nahin \cite[s. 85]{nahin15} używa tej nazwy do konkretnego wcielenia całki Frullaniego, dla $f = \arctan$.
Boros, Moll \cite[s. 98]{boros04} piszą mgliście \emph{,,under some mild conditions on the function $f$''}...

% Nahin Inside interesting... page 8x
\begin{problem}
    Niech $a, b > 0$.
    Wtedy
    \begin{equation}
        \int_0^\infty \frac{e^{-ax} - e^{-bx}}{x} \,\mathrm{d}x = \log \frac b a.
    \end{equation}
\end{problem}

% Nahin Inside interesting... page 89
\begin{problem}
    \begin{equation}
        \int_0^\infty \frac{\cos (ax) - \cos (bx)}{x^2} \,\mathrm{d}x = \frac \pi 2 (b - a).
    \end{equation}
\end{problem}

% Nahin Inside interesting... page 89
\begin{problem}
    \begin{equation}
        \int_0^\infty \frac{\cos (ax) - \cos (bx)}{x} \,\mathrm{d}x = \log \frac b a.
    \end{equation}
\end{problem}

% Nahin Inside interesting... page 89
\begin{problem}
    \label{nahin_kriegsrecht}
    \begin{equation}
        \int_0^\infty \frac{\log (a^2 x^2 + 1)}{x^2 + b^2} \,\mathrm{d}x = \frac \pi b \log (1 + ab).
    \end{equation}
\end{problem}

% SOLUTION
\textbf{Problem \ref{nahin_kriegsrecht}} -- \cite[s. 67]{nahin15} w szczególnym przypadku $a = b = 1$; \cite[s. 114, 375]{nahin15} w ogólności.
% SOLUTION

% Nahin Inside interesting... page 91
\begin{problem}
    Niech $a \ge 0$, wtedy
    \begin{equation}
        \int_0^1 \frac{x^a - 1}{\log x} \,\mathrm{d}x = \log(1+a).
    \end{equation}
\end{problem}

% Nahin Inside interesting... page 92
\begin{problem}
    Niech $a \ge 0$, wtedy
    \begin{equation}
        \int_0^1 \frac{x^a - x^b}{\log x} \,\mathrm{d}x = \log \frac{1+a}{1+b}.
    \end{equation}
\end{problem}

% Nahin Inside interesting... page 96
\begin{problem}
    Niech $a > b$, wtedy
    \begin{equation}
        \int_0^\pi \frac{\mathrm{d}x} {a + b \cos x} = \frac{\pi}{\sqrt{a^2 - b^2}}.
    \end{equation}
\end{problem}

\begin{problem}
    \label{nahin_dini}%
    Niech $a \ge 0$ będzie dowolną liczbą rzeczywistą.
    Wtedy
    \begin{equation}
        \int_0^\pi \log (1 - 2 a \cos x + a^2) \,\mathrm{d} x = \begin{cases}
            0, & \text{gdy } a^2 \le 1, \\
            2 \pi \log a & \textrm{w przeciwnym razie}.
        \end{cases}
    \end{equation}
\end{problem}

Nahin pisze, że powyższą całkę wyznaczył jako pierwszy Ulisse Dini\footnote{Włoski matematyk.} w 1878 roku i że (całka, nie Dini) ma ważne zastosowania w fizyce i inżynierii.
\index[persons]{Dini, Ulisse}%

% SOLUTION
\textbf{Problem \ref{nahin_dini}} -- \cite[s. 109-112]{nahin15}
% SOLUTION

% https://math.stackexchange.com/questions/580521/generalizing-int-01-frac-arctan-sqrtx2-2-sqrtx2-2
\begin{problem_with_solution}[całka Ahmeda]
    \label{ahmed_integral}%
    \begin{equation}
        \int_0^1 \frac{\arctan \sqrt{x^2+2}}{(x^2+1) \sqrt{x^2+2}} \,\mathrm{d}x = \frac{5\pi^2}{96}.
    \end{equation}
\end{problem_with_solution}

% TODO: https://www.youtube.com/watch?v=LkhtyxTnTZw When Ahmed and Glasser Meet...

% SOLUTION
\textbf{Problem \ref{ahmed_integral}} -- patrz praca Ahmeda Zafara \cite{ahmed02}.
\index[persons]{Zafar, Ahmed}
Rozwiązanie podaje też Nahin \cite[s. 190-194]{nahin15}.
% SOLUTION

\begin{problem_with_solution}[całka Coxetera]
    \label{coxeter_integral}%
    \begin{equation}
        \int_0^{\pi/2} \arccos \frac{\cos x}{1 + 2\cos x}  \,\mathrm{d}x = \frac{5\pi^2}{24}.
    \end{equation}
\end{problem_with_solution}

Młody Harold Coxeter zamieścił tę całkę w liście do Mathematical Gazette, rozwiązanie zostało nadesłane przez Hardy'ego.

% SOLUTION
\textbf{Problem \ref{coxeter_integral}} -- Nahin \cite[s. 190-201]{nahin15}; to prawdopodobnie najdłuższe rozwiązanie w całej jego książce.
% SOLUTION




%

\section{Zadania z turniejów całkowania}
	%

% Harvard:
% TODO: https://www.youtube.com/watch?v=hxAUEat_04o

% 2025
% TODO: https://www.youtube.com/watch?v=rGolSnrWc7s


\subsection{MIT Integration BEE 2024}
% kwalifikacje, problem 2: \frac{(x-1)^{\log (x+1)}}{(x+1)^{\log (x-1)}} = 1
% https://www.youtube.com/watch?v=p75k1DbuS0o

\begin{problem_with_solution}[ćwierćfinał 2, problem 2]
    \label{bee_mit_2024_q2_p2}%
    \begin{equation}
        I = \int_0^1 \frac 1 x \log (1 + x^2 + x^3 + x^4 + x^5 + x^6 + x^7 + x^9) \,\mathrm{d}x
    \end{equation}
\end{problem_with_solution}

% SOLUTION
% https://math.stackexchange.com/questions/1617081/proving-an-integration-equality
\textbf{Problem \ref{bee_mit_2024_q2_p2}} -- łatwo widać, że szukana całka jest równa $I_2 + I_3 + I_4$, gdzie
\begin{align}
    I_n & = \int_0^1 \frac {\log (1 + x^n)}{x} \,\mathrm{d}x \\
        & = \frac 1 n \int_0^1 \frac {\log (1 + x)}{x} \,\mathrm{d}x \\
        & = \frac 1 n \int_0^1 \frac 1 x \sum_{k=1}^\infty \frac{(-1)^{k-1}x^k}{k} \,\mathrm{d}x \\
        & = \frac 1 n \sum_{k=1}^\infty \frac{(-1)^{k-1}}{k^2} \\
        & = \frac {\pi^2}{12n}.
\end{align}
% SOLUTION

% qualifier q7 https://www.youtube.com/watch?v=YI-c9b7vNEs
% q9 https://www.youtube.com/watch?v=hzWtJ-qadws
% q15 https://www.youtube.com/watch?v=mxoyNN5g3hE
% q19 https://www.youtube.com/watch?v=7JwGegZDz4Y

% 2023
% https://math.stackexchange.com/questions/4642139/question-from-mit-integration-bee-2023-final-evaluate-int1-0-sum-infty-n
% https://www.youtube.com/watch?v=gn6LOe_bgw4

% 2022
% regular q8 https://www.youtube.com/watch?v=MlyvsB7PBlI

% 2020
% TODO: https://www.youtube.com/watch?v=oZWqG4IIHc0

% 2012
% q10, qualifier https://www.youtube.com/watch?v=HV489m57f2s

%

%

\section{Teoretyczna teoria}
\section{Teoretyczna teoria} % SOLUTION

\begin{problem}[problem B4 na egzaminie Putnam 1968]
    \label{putnam_1968_b4}%
    Niech $f \colon \R \to \R$ będzie ciągłą funkcją taką, że całka $\int_\R f(x)\,\mathrm{d}x$ istnieje.
    Pokazać, że całka
    \begin{equation}
        \int_\R f\left(x - \frac 1 x\right)\,\mathrm{d}x
    \end{equation}
    też istnieje i przyjmuje tę samą wartość.
\end{problem}

% SOLUTION
\begin{solution}[do problemu \ref{putnam_1968_b4}]
    Będziemy całkować przez podstawienie, $x = \exp \theta$ (i potem $x = - \exp -\theta$):
    \begin{align}
        \int_{-\infty}^{\infty}f\left(x-x^{-1}\right)dx&=\int_{0}^{\infty}f\left(x-x^{-1}\right)dx+\int_{-\infty}^{0}f\left(x-x^{-1}\right)dx=\\
        &=\int_{-\infty}^{\infty}f(2\sinh\theta)\,e^{\theta}d\theta+\int_{-\infty}^{\infty}f(2\sinh\theta)\,e^{-\theta}d\theta=\\
        &=\int_{-\infty}^{\infty}f(2\sinh\theta)\,2\cosh\theta\,d\theta=\\
        &=\int_{-\infty}^{\infty}f(x)\,dx.
    \end{align}
\end{solution}
% SOLUTION

%

\section{Trudne całki}
	%

\begin{problem_with_solution}
    \label{reuleaux_tetrahedron}%
    Czworościan Reuleaux to bryła będąca częścią wspólną czterech kul, których środki leżą w wierzchołkach czworościanu foremnego, a promienie są tej samej długości, co krawędzie tego czworościanu.
    Znaleźć objętość tej bryły,
    \begin{equation}
        V = \int_0^1
        \frac{
            8\sqrt{3}
        }{
            1 + 3t^2
        } - \frac{
            16 \sqrt{2} (3t+1) (4t^2 +t+1)^{3/2}
        }{
            (3t^2+1)(11t^2 + 2t + 3)^2
        } - \frac{
            \sqrt{2} (249 t^2 + 54t + 65)
        }{
            (11t^2 + 2t +3)^2
        } \,\mathrm{d} t.
    \end{equation}
\end{problem_with_solution}

% SOLUTION
\textbf{Problem \ref{reuleaux_tetrahedron}} -- patrz \url{https://mathworld.wolfram.com/ReuleauxTetrahedron.html}.
% SOLUTION

% TODO: https://mathworld.wolfram.com/images/gifs/FoxTrotMathTest.jpg
% TODO https://mathworld.wolfram.com/DefiniteIntegral.html




\begin{problem_with_solution}
    \label{schuster_integral}%
    Niech $S, C$ oznaczają całki Fresnela.
    % TODO: https://en.wikipedia.org/wiki/Fresnel_integral
    \begin{equation}
        \int_0^\infty (S(x)^2 + C(x)^2) \,\mathrm{d}x = \sqrt{\frac{\pi}{8}} \approx  0.62665\,70686\ldots
    \end{equation}
\end{problem_with_solution}

W 1925 roku brytyjski fizyk Arthur Schuster opublikował pracę na temat teorii światła, w której natknął się na powyższą całkę.
Sam nie potrafił jej wyznaczyć, ale zajął się tym Hardy -- dostał ten samą wartość, którą Schuster przypuszczał.

% SOLUTION
\textbf{Problem \ref{schuster_integral}} -- patrz Nahin \cite[s. 201-205]{nahin15}.
% SOLUTION

\begin{problem_with_solution}
    \label{watson_integrals1}%
    \begin{equation}
        I_1 = \frac{1}{\pi^3} \int_0^\pi\int_0^\pi\int_0^\pi \frac{\mathrm{d}u \, \mathrm{d}v \, \mathrm{d}w}{1 - \cos u \cos v \cos w} = \frac{\Gamma(1/4)^4}{4 \pi^3} = 1.39320\,39296\ldots
    \end{equation}
\end{problem_with_solution}

\begin{problem_with_solution}
    \label{watson_integrals2}%
    \begin{equation}
        I_2 = \frac{1}{\pi^3} \int_0^\pi\int_0^\pi\int_0^\pi \frac{\mathrm{d}u \, \mathrm{d}v \, \mathrm{d}w}{3 - \cos u \cos v - \cos u \cos w - \cos v \cos w} = \frac{3 \Gamma(1/3)^6}{2^{14/3} \pi^4} = 0.44822\,03943\ldots
    \end{equation}
\end{problem_with_solution}

\begin{problem_with_solution}
    \label{watson_integrals3}%
    \begin{equation}
        I_3 = \frac{1}{\pi^3} \int_0^\pi\int_0^\pi\int_0^\pi \frac{\mathrm{d}u \, \mathrm{d}v \, \mathrm{d}w}{3 - \cos u - \cos v - \cos w} = \frac{\Gamma(1/24) \Gamma(5/24) \Gamma(7/24) \Gamma(11/24)}{16 \sqrt{6} \pi^3} = 0.50546\,20197\ldots
    \end{equation}
\end{problem_with_solution}

W 1938 roku van Peype, student holenderskiego fizyka Kramnersa, napisał pracę, gdzie pojawiły się te trzy całki.
Chociaż van Peype znał wartość $I_1$, nie mógł sobie poradzić z $I_2$ oraz $I_3$, więc wysłał je do brytyjskiego fizyka Ralpha Fowlera, który przekazał je Hardy'emu, który nie poradził sobie z nimi.
Wynik jako pierwszy uzyskał George Watson, matematyk angielski.

% SOLUTION
\textbf{Problemy \ref{watson_integrals1}, \ref{watson_integrals2}, \ref{watson_integrals3}} -- patrz Nahin \cite[s. 206-212]{nahin15}.
% SOLUTION


%
	\subsection{Prawie niemożliwe całki}
% SOLUTION
\subsection{Prawie niemożliwe całki}
% SOLUTION
Wszystkie poniższe całki pojawiają się w książce Valeana \cite{valean19}.

\begin{problem_with_solution}
    \label{valean_grundpreis}%
    Niech $y \in (-1, 1)$.
    Wtedy
    \begin{equation}
        \int_0^1 \frac{\mathrm{d}x}{(1+yx) \sqrt{1-x^2}} = \frac{\arccos y}{\sqrt{1-y^2}}.
    \end{equation}
\end{problem_with_solution}

% SOLUTION
\textbf{Problem \ref{valean_grundpreis}} -- 
patrz \cite[s. 1]{valean19}.
% SOLUTION

\begin{problem_with_solution}
    \label{valean_zeugenstand}%
    Niech $m, n$ będą liczbami naturalnymi.
    Wtedy
    \begin{equation}
        \int_0^1 x^m \log^n x \,\mathrm{d} x = \frac{(-1)^n \cdot n!}{(m+1)^{n+1}}.
    \end{equation}
\end{problem_with_solution}

% SOLUTION
\begin{solution}[do problemu \ref{valean_zeugenstand}]
    Patrz \cite[s. 1]{valean19}.
\end{solution}
% SOLUTION


Niech $H_{n}^{(m)} = 1 + 1/2^m + \ldots + 1/n^m$ oznacza $n$-tą uogólnioną liczbę harmoniczną.

\begin{problem_with_solution}
    \label{valean_1_3}%
    Rozpatrujemy rodzinę całek
    \begin{equation}
        I_{k,n} := \int_0^1 x^{n-1} \log^k (1-x) \,\mathrm{d} x.
    \end{equation}
    Mamy:
    \begin{align}
        I_{1,n} & = - \frac{H_n}{n} \\
        I_{2,n} & = \frac{H_n^2 + H_n^{(2)}}{n} \\
        I_{3,n} & = - \frac{H_n^3 + 3H_nH_n^{(2)} + 2H_n^{(3)}}{n} \\
        I_{4,n} & = \frac{H_n^4 + 6H_n^2 H_n^{(2)} + 8H_nH_n^{(3)} + 3(H_n^{(2)})^2 + 6H_n^{(4)}}{n}.
    \end{align}
\end{problem_with_solution}

% (Valean nazywa to ,,four logarithmic integrals strongly connected with the league of harmonic series'').

% SOLUTION
\begin{solution}[do problemu \ref{valean_1_3}]
    Patrz \cite[s. 2]{valean19}.
\end{solution}
% SOLUTION

\begin{problem_with_solution}
    \label{valean_1_5}%
    Niech $s > 0$ będzie liczbą rzeczywistą, zaś $\psi$ oznacza funkcję digamma.
    Wtedy
    \begin{align}
        \int_0^1 \frac{x^{s-1}}{x+1} \,\mathrm{d} x & = \psi(s) - \psi\left(\frac s2\right) - \log 2 \\
        \int_0^\infty e^{-sx} \tanh x \,\mathrm{d} x & = \frac 1 2 \left[\psi\left(\frac{s+2}{4}\right) - \psi \left(\frac s4 \right) - \frac 2 s\right]. 
    \end{align}
\end{problem_with_solution}

% (Valean nazywa to ,,a couple of practical definite integrals expressed in terms of the digamma function'').

% SOLUTION
\begin{solution}[do problemu \ref{valean_1_5}]
    Patrz \cite[s. 3]{valean19}.
\end{solution}
% SOLUTION

\begin{problem_with_solution}
    \label{valean_1_7}%
    \begin{align}
        \int_0^1 \frac{1}{x} \log^2 (1+x) \,\mathrm{d}x & = \frac{1}{4} \zeta(3) \\
        \int_0^1 \frac{1}{x} \log (1+x) \log (1-x) \,\mathrm{d}x & = -\frac{5}{8} \zeta(3)
    \end{align}
\end{problem_with_solution}

% two little tricky classical logarithmic integrals

% SOLUTION
\begin{solution}[do problemu \ref{valean_1_7}]
    Patrz \cite[s. 4]{valean19}.
\end{solution}
% SOLUTION

\begin{problem_with_solution}[]
    \label{valean_1_8}%
    \begin{align}
        \int_0^1 [\log(1+x) \log(1-x)]^2 \,\mathrm{d} x & =
        24 - 8 \zeta(2)- 8 \zeta(3) - \zeta(4) \\
        & + 8 \log(2)\zeta(2) + 8 \log(2)\zeta(3) \\
        & - 4 \log^2(2)\zeta(2) \\
        & - 24 \log(2) + 12 \log^2(2)- 4 \log^3(2) + \log^4(2); 
    \end{align}
\end{problem_with_solution}

% a special trio of integrals

% SOLUTION
\begin{solution}[do problemu \ref{valean_1_8}]
    Patrz \cite[s. 4, 5]{valean19}.
\end{solution}
% SOLUTION

% TODO: https://math.stackexchange.com/questions/3413586/conjectural-closed-form-of-int-01-frac-logn-1-x-logn-1-1x1x-d

\begin{problem_with_solution}
    \label{valean_1_10}%
    Niech $n \ge 1$ będzie liczbą naturalną.
    Znaleźć
    \begin{equation}
        I_n = \int_0^1 \frac 1 x \log(1-x) \log^{2n} x \log (1+x) \,\mathrm{d}x.
    \end{equation}
    Jeśli jest to za trudne, pokazać, że
    \begin{align}
        I_1 & = \frac 3 4 \zeta (2) \zeta (3) - \frac {27}{16} \zeta(5), \\
        I_2 & = \frac 9 4 \zeta (3) \zeta (4) + \frac{45}{4} \zeta(2) \zeta(5) - \frac{363}{16} \zeta (7), \\
        I_3 & = \frac{2835}{8} \zeta(2) \zeta (7) + \frac {135}{8} \zeta (3) \zeta (6) + \frac {675}{8} \zeta (4) \zeta (5) - \frac {22635}{32} \zeta (9).
    \end{align} 
\end{problem_with_solution}

% the evaluation of a class of logarithmic integrals using a slightly modified result from ,,Table of Integrals, Series and Products'' by I. S. Gradshteyn and I. M. Ryzhik together with a series result elementarily proved by Guy Bastien

% SOLUTION
\begin{solution}[do problemu \ref{valean_1_10}]
    Patrz \cite[s. 6, 7]{valean19}.
\end{solution}
% SOLUTION

\begin{problem_with_solution}
    \label{valean_1_13}%
    \begin{equation}
        \int_0^1 \frac{x \log (1 \pm x)}{1 + x^2} \, \mathrm{d} x = \frac 1 8 \left(\log^2 (2) + \frac{\pm 3 - 2}{2} \zeta(2)\right).
    \end{equation} 
\end{problem_with_solution}

% A Special Pair of Logarithmic Integrals with Connections in the Area of the Alternating Harmonic Series

% SOLUTION
\begin{solution}[do problemu \ref{valean_1_13}]
    Patrz \cite[s. 8]{valean19}.
\end{solution}
% SOLUTION

% Another Special Pair of Logarithmic Integrals with Connections in the Area of the Alternating Harmonic Series

\begin{problem_with_solution}
    \label{valean_1_14}%
    \begin{align}
        16 \int_0^1 \frac{x}{1+ x^2} \log (1 - x) \log x \,\mathrm{d}x & = \frac{41}{4} \zeta(3) - 9 \log(2) \zeta(2) \\
        16 \int_0^1 \frac{x}{1+ x^2} \log (1 + x) \log x \,\mathrm{d}x & = -\frac{15}{4} \zeta(3) + 3 \log(2) \zeta(2)
    \end{align} 
\end{problem_with_solution}

% A Special Pair of Logarithmic Integrals with Connections in the Area of the Alternating Harmonic Series

% SOLUTION
\begin{solution}[do problemu \ref{valean_1_14}]
    Patrz \cite[s. 8]{valean19}.
\end{solution}
% SOLUTION

\begin{problem_with_solution}
    \label{valean_1_17}%
    \begin{align}
        \int_0^1 \int_0^1 \frac{\log y - \log x}{\log (- \log x) - \log(- \log y)} \,\mathrm{d}x \,\mathrm{d}y = \frac{7 \zeta(3)}{6 \zeta (2)}.
    \end{align} 
\end{problem_with_solution}

% Let’s Take Two Double Logarithmic Integrals with Beautiful Values Expressed in Terms of the Riemann Zeta Function

% SOLUTION
\begin{solution}[do problemu \ref{valean_1_17}]
    Patrz \cite[s. 10]{valean19}.
\end{solution}
% SOLUTION

\begin{problem_with_solution}
    \label{valean_1_18}%
    Niech $G$ oznacza stałą Catalana.
    \begin{align}
        8 \int_0^1 \log (1 - x) \arctan x \,\mathrm{d}x & = 4 \log (2) - \log^2 (2) + \frac 5 2 \zeta(2) - 2 \pi + \pi \log 2 - 8 G \\
        8 \int_0^1 \log (1 + x) \arctan x \,\mathrm{d}x & = 4 \log (2) - \log^2 (2) - \frac 1 2 \zeta(2) - 2 \pi + 3 \pi \log 2.
    \end{align} 
\end{problem_with_solution}

% Interesting Integrals Containing the Inverse Tangent Function and the Logarithmic Function

% SOLUTION
\begin{solution}[do problemu \ref{valean_1_18}]
    Patrz \cite[s. 10, 11]{valean19}.
\end{solution}
% SOLUTION

\begin{problem_with_solution}
    \label{valean_1_20}%
    Niech $G$ oznacza stałą Catalana.
    \begin{align}
        8 \int_0^1 \frac{\log (1 - x) \arctan x}{1+x^2} \,\mathrm{d}x & = \frac 3 4 \log (2) \zeta(2) - \frac 7 8 \zeta(3) - \pi G, \\
        8 \int_0^1 \frac{\log (1 + x) \arctan x}{1+x^2} \,\mathrm{d}x & = \frac 3 4 \log (2) \zeta(2) + \frac {21} 8 \zeta(3) - \pi G,
    \end{align} 
\end{problem_with_solution}

% More Interesting Integrals Involving the Inverse Tangent Function and the Logarithmic Function: The First Part

% SOLUTION
\begin{solution}[do problemu \ref{valean_1_20}]
    Patrz \cite[s. 12]{valean19}.
\end{solution}
% SOLUTION

\begin{problem_with_solution}
    \label{valean_1_21}%
    Niech $G$ oznacza stałą Catalana.
    \begin{align}
        \int_0^1 \frac{\arctan^2 x \log (1 + x)}{1 + x^2} \,\mathrm{d} x = \log 2 \frac {\pi^3}{384} + \frac {21}{256} \pi \zeta(3) - \frac{3}{16} \zeta (2) G.
    \end{align} 
\end{problem_with_solution}

% More Interesting Integrals Involving the Inverse Tangent Function and the Logarithmic Function: The Second Part

% SOLUTION
\begin{solution}[do problemu \ref{valean_1_21}]
    Patrz \cite[s. 12]{valean19}.
\end{solution}
% SOLUTION

\begin{problem_with_solution}
    \label{valean_1_22}%
    Niech $G$ oznacza stałą Catalana.
    \begin{align}
        I & = \int_0^1 \arctan x \log x \left(\log (1-x) - \frac {x}{1-x}\right) \,\mathrm{d} x \\
        & = G - \frac{41}{64} \zeta (3) + \frac{9 \log 2 - 5}{96} \pi^2 + \frac{2 - \log 2}{8} \pi - \frac {\log 2}{2} + \frac{\log^2 (2)}{8}.
    \end{align} 
\end{problem_with_solution}

% Challenging Integrals Involving arctan(x), log(x), log(1−x)

% SOLUTION
\begin{solution}[do problemu \ref{valean_1_22}]
    Patrz \cite[s. 13]{valean19}.
\end{solution}
% SOLUTION


\begin{problem_with_solution}
    \label{valean_1_23}%
    \begin{align}
        I & = \int_0^1 \arctan x \log x \log (1 + x) \,\mathrm{d}x \\
        & = \frac{\log 2}{2} G - \frac{\pi^3}{64} + \frac{15}{64} \zeta(3) - \frac{\pi^2}{96} (3 \log 2 + 1) + \frac{\pi} {8} (4-3 \log 2) + \frac{\log^2(2)}{8} - \log 2.
    \end{align} 
\end{problem_with_solution}

% Challenging Integrals Involving arctan(x), log(x), log(1−x)

% SOLUTION
\begin{solution}[do problemu \ref{valean_1_23}]
    Patrz \cite[s. 13, 14]{valean19}.
\end{solution}
% SOLUTION



\begin{problem_with_solution}
    \label{valean_1_24}%
    Niech $n \ge 1$ będzie naturalne.
    Znaleźć wartość
    \begin{align}
        I_{2n} & := \int_0^1 \frac {\arctan x \log^{2n} (x)}{1 + x } \,\mathrm{d}x
    \end{align} 
    lub, jeśli jest to za trudne, pokazać, że 
    \begin{align}
        I_{1} = \frac{\log 2}{2}G - \frac{\pi^3}{64}.
    \end{align} 
\end{problem_with_solution}

% Challenging Integrals Involving arctan(x), log(x), log(1−x)

% SOLUTION
\begin{solution}[do problemu \ref{valean_1_24}]
    Patrz \cite[s. 14, 15]{valean19}.
\end{solution}
% SOLUTION


\begin{problem_with_solution}
    \label{valean_1_26}%
    \begin{align}
        \int_0^1 \frac{\arctan x}{x} \log \frac{1+x^2}{(1-x)^2} \,\mathrm{d}x = \frac{\pi^3}{16}.
    \end{align} 
\end{problem_with_solution}

% Challenging Integrals Involving arctan(x), log(x), log(1−x)

% SOLUTION
\begin{solution}[do problemu \ref{valean_1_26}]
    Patrz \cite[s. 17]{valean19}.
\end{solution}
% SOLUTION

\begin{problem_with_solution}
    \label{valean_1_32}%
    Znaleźć rekurencję, jaką spełnia
    \begin{align}
        I_n = \int_0^1 \frac{x^n}{(1+x)(1+x^2)^n} \,\mathrm{d}x.
    \end{align} 
\end{problem_with_solution}

% Challenging Integrals Involving arctan(x), log(x), log(1−x)

% SOLUTION
\begin{solution}[do problemu \ref{valean_1_32}]
    Patrz \cite[s. 21, 22]{valean19}.
\end{solution}
% SOLUTION

\begin{problem_with_solution}
    \label{valean_1_37}%
    \begin{align}
        \int_0^\infty \int_0^\infty \frac {e^{-x}-e^{-y}}{x-y} \frac{1-e^{-x}}{x} \frac{1-e^{-y}}{y} \,\mathrm{d}x \,\mathrm{d}y.
    \end{align} 
\end{problem_with_solution}

% SOLUTION
\begin{solution}[do problemu \ref{valean_1_37}]
    Patrz \cite[s. ?????]{valean19}.
\end{solution}
% SOLUTION


\begin{problem_with_solution}
    \label{valean_1_38}%
    Znaleźć
    \begin{align}
        \lim_{n\to\infty} \left(\frac{1}{n!} \int_0^\infty \int_0^\infty \frac{x^n - y^n}{e^x - e^y} \,\mathrm{d}x \,\mathrm{d}y - 2n\right)
    \end{align} 
    po znalezieniu wewnętrznej całki (dla całkowitych liczb $n \ge 1$).
\end{problem_with_solution}

% SOLUTION
\begin{solution}[do problemu \ref{valean_1_38}]
    Patrz \cite[s. ?????]{valean19}.
\end{solution}
% SOLUTION

\begin{problem_with_solution}
    \label{valean_1_40}%
    \begin{align}
        I_4 & = \int_0^\infty \frac{\pi^2}{x^3} \tanh (\pi x)  + \frac{3}{x^5} \tanh(\pi x) - \frac{3\pi}{x^4} \,\mathrm{d}x = 93 \zeta(5) - 42 \zeta(2) \zeta(3), \\
        I_5 & = \int_0^\infty \frac{2\pi^2}{x^3} \tanh(\pi x) + \frac{12}{x^5} \tanh (\pi x) - \frac{6 \pi^2}{x^3} \csch (2 \pi x) - \frac{9 \pi}{x^4} \,\mathrm{d} x \\
        & = 372 \zeta(5) - 192 \zeta(2)\zeta(3).
    \end{align} 
\end{problem_with_solution}

% SOLUTION
\begin{solution}[do problemu \ref{valean_1_40}]
    Patrz \cite[s. ?????]{valean19}.
\end{solution}
% SOLUTION


\begin{problem_with_solution}
    \label{valean_1_41}%
    \begin{align}
        I_1 & = \int_0^\infty \frac{\sin (\sin x)}{x} e^{\cos x} \,\mathrm{d} x = \frac{\pi} 2 (e - 1), \\
        I_2 & = \int_0^\infty \frac{\sin x \cdot \sin (\sin x)}{x^2} e^{\cos x} \,\mathrm{d} x = \frac{\pi} 2 (e - 1).
    \end{align} 
\end{problem_with_solution}

% SOLUTION
\begin{solution}[do problemu \ref{valean_1_41}]
    Patrz \cite[s. ?????]{valean19}.
\end{solution}
% SOLUTION


%
	%

\subsection{Znalezione na math.stackexchange.com}
% SOLUTION
\subsection{Znalezione na math.stackexchange.com}
% SOLUTION
Wszystkie poniższe całki pojawiają się na stronie na math.stackexchange.com.

%%

% https://math.stackexchange.com/q/1653979
\begin{problem}[pytanie 9402]
    \label{stack_9402}%
    % TODO: https://math.stackexchange.com/questions/9402/calculating-the-integral-int-0-infty-frac-cos-x1x2-mathrmdx-with
    \begin{equation}
        \int_0^\infty \frac {\cos b x}{1+x^2} \,\mathrm{d}x = \frac{\pi} 2 \exp(-b).
    \end{equation}
\end{problem}

%%

\begin{problem}[pytanie 15719]
    \label{stack_15719}%
    \begin{equation}
        \int \frac{1 + x^2}{(1 - x^2) \sqrt{1 + x^4}} \,\mathrm{d}x = \frac{1}{\sqrt 2} \log \frac{2x + \sqrt{2x^4 + 2}}{x^2 - 1}.
    \end{equation}
\end{problem}

%%

% https://math.stackexchange.com/questions/110457/closed-form-for-int-0-infty-fracxn1-xmdx
\begin{problem_with_solution}[pytanie 110457]
    \label{stack_110457}%
    Niech $0 < n < m$, wtedy
    \begin{equation}
        \int_0^\infty \frac{n x^{n-1}}{1 + x^m} \,\mathrm{d} x = \frac {\pi n} m \operatorname{csc} \frac {\pi n}{m}.
    \end{equation}
\end{problem_with_solution}

% SOLUTION
\textbf{Problem \ref{stack_110457}} -- podstawiamy $x = \tan^{2/m} \theta$, co prowadzi do całki
\begin{align}
    I & = \int_0^\infty \frac{x^{n-1}}{1 + x^m} \,\mathrm{d} x \\
      & = \int_0^{\pi/2} \frac 2 m \tan^{2n/m - 1} \theta \,\mathrm{d}\theta \\
      & = \frac 1 m \beta\left( \frac nm, 1 - \frac nm \right) \\
      & = \frac 1 m \Gamma \left(\frac nm\right) \Gamma \left(1 - \frac nm\right) \\
      & = \frac \pi m \operatorname{csc} \frac {\pi n}{m}.
\end{align}
% SOLUTION

%%

% https://math.stackexchange.com/questions/155941/evaluate-the-integral-int-01-frac-lnx1x21-mathrm-dx
\begin{problem_with_solution}[pytanie 155941]
    \label{stack_155941}%
    \begin{equation}
        \int_0^1 \frac{\log (1+x)}{1 + x^2} \,\mathrm{d}x = \frac \pi 8  \log 2.
    \end{equation}
\end{problem_with_solution}

% SOLUTION
\textbf{Problem \ref{stack_155941}} -- podstawiamy $x = \tan \theta$.
% SOLUTION

%%

% https://math.stackexchange.com/q/178790
\begin{problem}[pytanie 178790]
    \label{stack_178790}%
    \begin{equation}
        \int_0^{\pi/2} \frac{x^2}{x^2 + [\log (2 \cos x)]^2} \,\mathrm{d}x = \frac{\pi}{8} (1 - \gamma + \log (2 \pi)).
    \end{equation}
\end{problem}

%%

% https://math.stackexchange.com/questions/187729/evaluating-int-0-infty-sin-x2-dx-with-real-methods
\begin{problem_with_solution}[pytanie 187729, całka Fresnela]
    \label{stack_187729}%
    \begin{equation}
        I = \int_0^\infty \sin (x^2) \,\mathrm{d} x = \sqrt{\frac \pi 8}.
    \end{equation}
\end{problem_with_solution}

Całki Fresnela mają praktyczne zastosowanie, historycznie pierwszym było obliczenie natężenia pola elektromagnetycznego w środowisku, gdzie światło ugina się wokół nieprzezroczystych obiektów.
\index{całka Fresnela}%

% SOLUTION
\textbf{Problem \ref{stack_187729}} -- znajdziemy ogólniejszą całkę $I_\lambda$ funkcji $\sin(x^2) e^{-\lambda x^2}$ nad zbiorem $[0, \infty)$.
\begin{align}
    I_\lambda^2 & = \left(\int_0^\infty \sin(x^2) e^{-\lambda x^2} \,\mathrm{d}x \right)^2 \\
    & = \int_0^\infty \int_0^\infty \sin(x^2)\sin(y^2) e^{- \lambda(x^2+y^2)}\,\mathrm{d}y\,\mathrm{d}x \\
    & = \frac12 \int_0^\infty \int_0^\infty \left(\cos(x^2-y^2)-\cos(x^2+y^2)\right) e^{- \lambda(x^2+y^2)}\,\mathrm{d}y\,\mathrm{d}x \\
    & = \frac12 \int_0^{\pi/2} \int_0^\infty \left(\cos(r^2\cos(2\phi))-\cos(r^2)\right)e^{- \lambda r^2} \,r\,\mathrm{d}r\,\mathrm{d}\phi \\
    & = \frac14 \int_0^{\pi/2} \int_0^\infty \left(\cos(s\cos(2\phi))-\cos(s)\right) e^{- \lambda s} \,\mathrm{d}s\,\mathrm{d}\phi \\
    & = \frac14 \int_0^{\pi/2} \left( \frac{ \lambda}{\cos^2(2\phi)+ \lambda^2} - \frac{ \lambda}{1+ \lambda^2}\right)\,\mathrm{d}\phi \\
    & = \frac12 \int_0^{\pi/4} \frac{ \lambda\,\mathrm{d}\phi}{\cos^2(2\phi)+ \lambda^2} - \frac{ \lambda\pi/8}{1+ \lambda^2} \\
    & = \frac14 \int_0^{\pi/4} \frac{ \lambda\,\mathrm{d} \tan(2\phi)} {1+ \lambda^2+ \lambda^2 \tan^2(2\phi)} - \frac{ \lambda\pi/8}{1+ \lambda^2} \\
    & = \frac14 \int_0^\infty \frac1{1+ \lambda^2+t^2}\,\mathrm{d}t - \frac{ \lambda\pi/8}{1+ \lambda^2} \\
    & = \frac{\pi/8}{\sqrt{1+ \lambda^2}} - \frac{ \lambda\pi/8}{1+ \lambda^2}
\end{align}
% SOLUTION

%%

% https://math.stackexchange.com/questions/426325/evaluate-int-01-frac-log-left-1x2-sqrt3-right1x-mathrm-dx
\begin{problem}[pytanie 426325]
    \label{stack_426325}%
    \begin{equation}
        \int_0^1 \frac{\log \left(1 + x^{2 + \sqrt 3}\right)}{1 + x} \,\mathrm{d} x = \frac{\pi^2}{12} (1 - \sqrt 3) + \log (2) \log(1 + \sqrt 3).
    \end{equation}
\end{problem}

%%

% https://math.stackexchange.com/questions/464769/how-to-prove-int-01-tan-1-left-frac-tanh-1x-tan-1x-pi-tanh-1
\begin{problem}[pytanie 464769]
    \label{stack_464769}%
    \begin{equation}
        \int_0^1 \arctan \frac { \operatorname{artanh} x - \arctan x} {\pi + \operatorname{artanh} x - \arctan x}  \, \frac{\mathrm{d}x}{x} = \frac \pi 4 \log \frac{\pi}{2 \sqrt{2}}.
    \end{equation}
\end{problem}

%%

% https://math.stackexchange.com/questions/507425/an-integral-involving-airy-functions-int-0-infty-fracxp-operatornameai
\begin{problem_with_solution}[pytanie 507425]
    \label{stack_507425}%
    Niech
    \begin{align}
        \operatorname{Ai} (x) & = \frac 1 \pi \int_0^\infty \cos \left( x z + \frac {z^3} 3 \right) \,\mathrm{d}z, \\
        \operatorname{Bi} (x) & = \frac 1 \pi \int_0^\infty \sin \left( x z + \frac {z^3} 3 \right) + \exp \left( x z - \frac {z^3} 3 \right) \,\mathrm{d}z
    \end{align}
    będą funkcjami Airy'ego, zaś $n \in \mathbb N$ parametrem.
    Znaleźć
    \begin{equation}
        I_n = \int_0^\infty \frac{x^{3n} \,\mathrm{d} x}{(\operatorname{Ai} x)^2 + (\operatorname{Bi} x)^2}.
    \end{equation}
\end{problem_with_solution}

% SOLUTION
\textbf{Problem \ref{stack_507425}} -- $I_n = \pi^2 a_{2n} / (6 \cdot 2^{7n})$, gdzie $a_0 = 1$, $a_{n+1} = (6n+4)a_n \sum_{k=0}^n a_k a_{n-k}$.
% SOLUTION


%%

% https://math.stackexchange.com/questions/523027/a-math-contest-problem-int-01-ln-left1-frac-ln2x4-pi2-right-frac
\begin{problem}[pytanie 523027]
    \label{stack_523027}%
    \begin{equation}
        \int_0^1 \log\left(1+\left(\frac{\log x}{2\pi}\right)^2 \right)\frac{\log(1-x)}x \,\mathrm{d} x=-\pi^2\left(4\zeta'(-1)+\frac23\right).
    \end{equation}
\end{problem}

%%

% https://math.stackexchange.com/questions/541751/how-prove-this-i-int-0-infty-frac1x-ln-left-frac1x1-x-right2/541861#541861
\begin{problem}[pytanie 541751]
    \label{stack_541751}%
    \begin{equation}
        I = \int_0^\infty \frac{1}{x} \log \left(\frac{1+x}{1-x}\right)^2 \,\mathrm{d}x = \pi^2.
    \end{equation}
\end{problem}

%%

% https://math.stackexchange.com/q/562694
\begin{problem}[pytanie 562694]
    \label{stack_562694}%
    \begin{equation}
        \int_{-1}^1 \frac{1}{x} \sqrt{\frac{1+x}{1-x}} \log \frac{2x^2+2x+1}{2x^2-2x+1} \,\mathrm{d}x = 4 \pi \operatorname{arccot} \sqrt{\phi}.
    \end{equation}
\end{problem}

%%

\begin{problem}[pytanie 570997]
    \label{stack_570997}%
    \begin{equation}
        \int_0^1 \frac{\log (x + \sqrt 2)}{\sqrt{x(1-x)(2-x)}} \,\mathrm{d}x = \frac{\pi^{3/2}}{8\sqrt{2} \Gamma(3/4)^2 } \left(7 \log 2 + 4 \log (1 + \sqrt 2) - \pi \right).
    \end{equation}
\end{problem}

%%

% https://math.stackexchange.com/questions/815863/compute-int-0-pi-4-frac1-x2-ln1x21x2-1-x2-ln1-x21-x4
\begin{problem}[pytanie 815863]
    \label{stack_815863}%
    \begin{align}
        I & = \int_0^{\pi/4} \frac{ (1-x^2) [ \log(1+x^2) - \log(1 - x^2)] + 1 + x^2}{(1-x^4)(1+x^2)} x \exp \frac {x^2 - 1}{x^2 + 1} \,\mathrm{d} x \\
        & =  - \frac 1 4  \exp \frac{\pi^2 - 16}{\pi^2 + 16} \log \frac {16 - \pi^2}{16 + \pi^2}.
    \end{align}
\end{problem}

%%

% TODO: https://math.stackexchange.com/a/942440

%%

\begin{problem_with_solution}[pytanie 1582943]
    \label{stack_1582943}%
    \begin{equation}
        \int_0^\infty \left(\frac{\tanh x}{x}\right)^2 \,\mathrm{d}x = \frac{14 \zeta (3)}{\pi^2}.
    \end{equation}
\end{problem_with_solution}

% SOLUTION
\textbf{Problem \ref{stack_1582943}} -- dowodzimy najpierw, że poniższe całki są równe:
\begin{equation}
    \frac{\pi^2}{4} \int_0^\infty \frac{\tanh x \cdot \tanh xs}{x^2} \,\mathrm{d}x = s \int_0^1 \log \frac{1-x}{1+x} \log \frac{1-x^s}{1+x^s} \frac{\mathrm{d}x}{x}.
\end{equation}
% SOLUTION

%%

% https://math.stackexchange.com/q/1653979
\begin{problem}[pytanie 1653979]
    \label{stack_1653979}%
    % Niech $\phi = \frac 1 2 (1 + \sqrt 5)$ oznacza złotą liczbę.
    \begin{equation}
        \int_0^\infty \frac{5x^2}{1  + x^{10}} \,\mathrm{d}x = \frac{\pi}{\phi}.
    \end{equation}
\end{problem}

%%

% https://math.stackexchange.com/q/2529614
\begin{problem}[pytanie 2529614]
    Korzystając z technik analizy zespolonej pokazać, że
    \label{stack_2529614}%
    \begin{equation}
        \int_0^\infty \frac{x^{-\mathrm{i}a}}{x^2+bx+1} \,\mathrm{d}x = \frac{2\pi}{\sqrt{4-b^2}} \cdot \frac{\sinh (a \arccos (b/2))}{\sinh (a \pi)}.
    \end{equation}
\end{problem}

%%

% https://math.stackexchange.com/q/2826571
\begin{problem}[pytanie 2826571]
    \label{stack_2826571}%
    Niech $(m, n)$ oznacza największy wspólny dzielnik liczb $m, n$.
    Wtedy
    \begin{equation}
        \int_0^{\pi/2} \log \lvert \sin(mx) \rvert \cdot \log \lvert\sin(nx)\rvert \, dx = \frac{\pi^3}{24} \frac{(m,n)^2}{mn}+\frac{\pi\log (2)^2}{2}.
    \end{equation}
\end{problem}

%%

% https://math.stackexchange.com/questions/3490404/
\begin{problem}[pytanie 3490404]
    \label{stack_3490404}%
    Niech $f(x) = x \log (1 - \sin x) / \sin x$.
    Udowodnić, że
    \begin{align}
        2\int_0^{\pi/2} f(x) \,\mathrm{d}x =
        \int_{\pi/2}^\pi\ f(x) \,\mathrm{d}x,
    \end{align}
    bez uprzedniego znajdowania funkcji pierwotnej $f$.
\end{problem}

%%

% https://math.stackexchange.com/q/3981861
\begin{problem}[pytanie 3981861]
    Udowodnić lub podać kontrprzykład: niech $f \colon [0, 2\pi] \to [0, 2\pi]$ będzie różniczkowalną funkcją taką, że $f(0) = f(2\pi)$. Wtedy
    \label{stack_3981861}%
    \begin{equation}
        \left(\int_0^{2 \pi} \cos f(x) \,d x\right)^2
        +
        \left(\int_0^{2 \pi} \sqrt{(f'(x))^2+\sin^2 f(x)} \, \mathrm{d}x\right)^2 \ge (2 \pi)^2.
    \end{equation}
\end{problem}

%%

% https://math.stackexchange.com/q/4568778
\begin{problem}[pytanie 4568778]
    \label{stack_4568778}%
    \begin{equation}
        \int_0^1 \frac{\log (x+1) - \log(2x^2)}{\sqrt{x^2 + 2x}}\,\mathrm{d}x = \frac{\pi^2}{2}.
    \end{equation}
\end{problem}

%

\begin{problem_with_solution}
\label{nahin_page_225}%
    \begin{equation}
		\int_1^\infty \frac{\mathrm{d}x}{x^3 + x} = \log \sqrt {2}.
    \end{equation}
\end{problem_with_solution}

% SOLUTION
\textbf{Problem \ref{nahin_page_225}}.
Rozkładamy całkowaną funkcję na sumę
\begin{equation}
	\frac 1 x - \frac{1}{2(x - i)} - \frac{1}{2(x+i)}
\end{equation}
tak jak Nahin \cite[s. 225]{nahin15}.
% SOLUTION



\chapter{Tłumaczenie numeracji}
Po lewej stronie numer zadania w \cite{wedrychowicz12}, po prawej numer tego samego zadaina u nas.

\begin{multicols}{3}
\begin{itemize}
    \item 12.1 -- problem \ref{banas_12_1}
    \item 12.2 -- problem \ref{banas_12_2}
    \item 12.3 -- problem \ref{banas_12_3}
    \item 12.4 -- tak samo jak 12.3
    \item 12.5 -- tak samo jak 12.3
    \item 12.6 -- problem \ref{banas_12_6}
    \item 12.7 -- problem \ref{banas_12_7}
    \item 12.8 -- problem \ref{banas_12_8}
    \item 12.9 -- problem \ref{banas_12_9}
    \item 12.10 -- problem \ref{banas_12_10}
    \item 12.11 -- problem \ref{banas_12_11}
    \item 12.12 -- problem \ref{banas_12_12}
    \item 12.13 -- problem \ref{banas_12_13}
    \item 12.14 -- problem \ref{banas_12_14}
    \item 12.15 -- tak samo jak 12.14
    \item 12.16 -- tak samo jak 12.12
    \item 12.17 -- tak samo jak 12.12
    \item 12.18 -- problem \ref{banas_12_18}
    \item 12.19 -- problem \ref{banas_12_19}
    \item 12.20 -- problem \ref{banas_12_20}
    \item 12.21 -- problem \ref{banas_12_21}
    \item 12.22 -- pominięta
    \item 12.23 -- pominięta
    \item 12.24 -- pominięta
    \item 12.25 -- tak samo jak 12.14
    \item 12.26 -- problem \ref{banas_12_26}
    \item 12.27 -- problem \ref{banas_12_27}
    \item 12.28 -- pominięta
    \item 12.29 -- patrz akapit za \ref{gauss_euler_poisson}
    \item 12.30 -- problem \ref{banas_12_30}
    \item 12.31 -- problem \ref{banas_12_31}
    \item 12.32 -- tak samo jak 12.33
    \item 12.33 -- problem \ref{banas_12_33}
    \item 12.34 -- problem \ref{banas_12_34}
    \item 12.35 -- pominięta
    \item 12.36 -- pominięta
    \item 12.37 -- pominięta
    \item 12.38 -- pominięta
    \item 12.39 -- pominięta
    \item 12.40 -- problem \ref{banas_12_40}
    \item 12.41 -- problem \ref{banas_12_41}
    \item 12.42 -- pominięta
    \item 12.43 -- pominięta
    \item 12.44 -- pominięta
    \item 12.45 -- pominięta
    \item 12.46 -- pominięta
    \item 12.47 -- pominięta
    \item 12.48 -- pominięta
    \item 12.49 -- pominięta
    \item 12.50 -- pominięta
    % \item 12.51 -- problem \ref{banas_12_51}
    % \item 12.52 -- problem \ref{banas_12_52}
    % \item 12.53 -- problem \ref{banas_12_53}
    % \item 12.54 -- problem \ref{banas_12_54}
    % \item 12.55 -- problem \ref{banas_12_55}
    % \item 12.56 -- problem \ref{banas_12_56}
    % \item 12.57 -- problem \ref{banas_12_57}
    % \item 12.58 -- problem \ref{banas_12_58}
    \item 12.59 -- pominięta
    \item 12.60 -- pominięta
    \item 12.61 -- pominięta
    \item 12.62 -- pominięta
    \item 12.63 -- pominięta
    \item 12.64 -- pominięta
    \item 12.65 -- pominięta
    \item 12.66 -- pominięta
    \item 12.67 -- pominięta
    \item 12.68 -- pominięta
    \item 12.69 -- pominięta
    \item 12.70 -- pominięta
    % \item 12.71 -- problem \ref{banas_12_71}
    % \item 12.72 -- problem \ref{banas_12_72}
    % \item 12.73 -- problem \ref{banas_12_73}
    % \item 12.74 -- problem \ref{banas_12_74}
    % \item 12.75 -- problem \ref{banas_12_75}
    % \item 12.76 -- problem \ref{banas_12_76}
    % \item 12.77 -- problem \ref{banas_12_77}
    % \item 12.78 -- problem \ref{banas_12_78}
    % \item 12.79 -- problem \ref{banas_12_79}
    % \item 12.80 -- problem \ref{banas_12_80}
    % \item 12.81 -- problem \ref{banas_12_81}
    % \item 12.82 -- problem \ref{banas_12_82}
    % \item 12.83 -- problem \ref{banas_12_83}
    % \item 12.84 -- problem \ref{banas_12_84}
    % \item 12.85 -- problem \ref{banas_12_85}
    % \item 12.86 -- problem \ref{banas_12_86}
    % \item 12.87 -- problem \ref{banas_12_87}
    % \item 12.88 -- problem \ref{banas_12_88}
    % \item 12.89 -- problem \ref{banas_12_89}
    % \item 12.90 -- problem \ref{banas_12_90}
    % \item 12.91 -- problem \ref{banas_12_91}
    % \item 12.92 -- problem \ref{banas_12_92}
    % \item 12.93 -- problem \ref{banas_12_93}
    % \item 12.94 -- problem \ref{banas_12_94}
    % \item 12.95 -- problem \ref{banas_12_95}
    % \item 12.96 -- problem \ref{banas_12_96}
    % \item 12.97 -- problem \ref{banas_12_97}
    % \item 12.98 -- problem \ref{banas_12_98}
    % \item 12.99 -- problem \ref{banas_12_99}
    \item 12.100 -- problem \ref{banas_12_100}
    % \item 12.101 -- problem \ref{banas_12_101}
    % \item 12.102 -- problem \ref{banas_12_102}
    % \item 12.103 -- problem \ref{banas_12_103}
    % \item 12.104 -- problem \ref{banas_12_104}
    % \item 12.105 -- problem \ref{banas_12_105}
    % \item 12.106 -- problem \ref{banas_12_106}
    % \item 12.107 -- problem \ref{banas_12_107}
    % \item 12.108 -- problem \ref{banas_12_108}
    % \item 12.109 -- problem \ref{banas_12_109}
    % \item 12.110 -- problem \ref{banas_12_110}
    % \item 12.111 -- problem \ref{banas_12_111}
    % \item 12.112 -- problem \ref{banas_12_112}
    % \item 12.113 -- problem \ref{banas_12_113}
    % \item 12.114 -- problem \ref{banas_12_114}
    % \item 12.115 -- problem \ref{banas_12_115}
    % \item 12.116 -- problem \ref{banas_12_116}
    % \item 12.117 -- problem \ref{banas_12_117}
    \item 12.118 -- problem \ref{banas_12_118}
    % \item 12.119 -- problem \ref{banas_12_119}
    % \item 12.120 -- problem \ref{banas_12_120}
    % \item 12.121 -- problem \ref{banas_12_121}
    % \item 12.122 -- problem \ref{banas_12_122}
    % \item 12.123 -- problem \ref{banas_12_123}
    % \item 12.124 -- problem \ref{banas_12_124}
    % \item 12.125 -- problem \ref{banas_12_125}
    % \item 12.126 -- problem \ref{banas_12_126}
    % \item 12.127 -- problem \ref{banas_12_127}
    % \item 12.128 -- problem \ref{banas_12_128}
    % \item 12.129 -- problem \ref{banas_12_129}
    % \item 12.130 -- problem \ref{banas_12_130}
    % \item 12.131 -- problem \ref{banas_12_131}
    % \item 12.132 -- problem \ref{banas_12_132}
    % \item 12.133 -- problem \ref{banas_12_133}
    % \item 12.134 -- problem \ref{banas_12_134}
    % \item 12.135 -- problem \ref{banas_12_135}
    % \item 12.136 -- problem \ref{banas_12_136}
    % \item 12.137 -- problem \ref{banas_12_137}
    % \item 12.138 -- problem \ref{banas_12_138}
    % \item 12.139 -- problem \ref{banas_12_139}
    % \item 12.140 -- problem \ref{banas_12_140}
    % \item 12.141 -- problem \ref{banas_12_141}
    % \item 12.142 -- problem \ref{banas_12_142}
    % \item 12.143 -- problem \ref{banas_12_143}
    % \item 12.144 -- problem \ref{banas_12_144}
    % \item 12.145 -- problem \ref{banas_12_145}
    % \item 12.146 -- problem \ref{banas_12_146}
    % \item 12.147 -- problem \ref{banas_12_147}
    % \item 12.148 -- problem \ref{banas_12_148}
    % \item 12.149 -- problem \ref{banas_12_149}
    % \item 12.150 -- problem \ref{banas_12_150}
    % \item 12.151 -- problem \ref{banas_12_151}
    % \item 12.152 -- problem \ref{banas_12_152}
    % \item 12.153 -- problem \ref{banas_12_153}
    % \item 12.154 -- problem \ref{banas_12_154}
    % \item 12.155 -- problem \ref{banas_12_155}
    % \item 12.156 -- problem \ref{banas_12_156}
    % \item 12.157 -- problem \ref{banas_12_157}
    % \item 12.158 -- problem \ref{banas_12_158}
    % \item 12.159 -- problem \ref{banas_12_159}
    \item 12.160 -- problem \ref{banas_12_160}
    % \item 12.161 -- problem \ref{banas_12_161}
    % \item 12.162 -- problem \ref{banas_12_162}
    % \item 12.163 -- problem \ref{banas_12_163}
    % \item 12.164 -- problem \ref{banas_12_164}
    % \item 12.165 -- problem \ref{banas_12_165}
    % \item 12.166 -- problem \ref{banas_12_166}
    % \item 12.167 -- problem \ref{banas_12_167}
    % \item 12.168 -- problem \ref{banas_12_168}
    % \item 12.169 -- problem \ref{banas_12_169}
    % \item 12.170 -- problem \ref{banas_12_170}
    % \item 12.171 -- problem \ref{banas_12_171}
    % \item 12.172 -- problem \ref{banas_12_172}
    % \item 12.173 -- problem \ref{banas_12_173}
    % \item 12.174 -- problem \ref{banas_12_174}
    % \item 12.175 -- problem \ref{banas_12_175}
    % \item 12.176 -- problem \ref{banas_12_176}
    % \item 12.177 -- problem \ref{banas_12_177}
    % \item 12.178 -- problem \ref{banas_12_178}
    % \item 12.179 -- problem \ref{banas_12_179}
    % \item 12.180 -- problem \ref{banas_12_180}
    % \item 12.181 -- problem \ref{banas_12_181}
    % \item 12.182 -- problem \ref{banas_12_182}
    % \item 12.183 -- problem \ref{banas_12_183}
    % \item 12.184 -- problem \ref{banas_12_184}
    % \item 12.185 -- problem \ref{banas_12_185}
    % \item 12.186 -- problem \ref{banas_12_186}
    % \item 12.187 -- problem \ref{banas_12_187}
    % \item 12.188 -- problem \ref{banas_12_188}
    % \item 12.189 -- problem \ref{banas_12_189}
    % \item 12.190 -- problem \ref{banas_12_190}
    % \item 12.191 -- problem \ref{banas_12_191}
    % \item 12.192 -- problem \ref{banas_12_192}
    % \item 12.193 -- problem \ref{banas_12_193}
    % \item 12.194 -- problem \ref{banas_12_194}
    % \item 12.195 -- problem \ref{banas_12_195}
    % \item 12.196 -- problem \ref{banas_12_196}
    % \item 12.197 -- problem \ref{banas_12_197}
    % \item 12.198 -- problem \ref{banas_12_198}
    % \item 12.199 -- problem \ref{banas_12_199}
\end{itemize}
\end{multicols}

%

\indexprologue{\small Tekst prologu I.}
\printindex

\indexprologue{\small Tekst prologu II.}
\printindex[persons]

\raggedright
\bibliography{integrals}{}
\bibliographystyle{plain}

\end{document}

% TODO: https://www.amazon.com/dp/0867202939
% TODO: $$\int_0^\infty \log(x) / (1+x^2) dx = 0$$ (Euler?)
% TODO: $$\int_0^\infty 1/(x^3 - 1) dx = -pi sqrt 3 / 9$$

% TODO: https://math.stackexchange.com/questions/tagged/integration?tab=votes&page=2&pagesize=15

% TODO: https://www.youtube.com/watch?v=p75k1DbuS0o
% TODO: https://www.youtube.com/watch?v=lXOLcLd4-BA
% TODO: https://www.youtube.com/watch?v=cp_jmqfrcrc
% TODO: https://www.youtube.com/watch?v=SQ-CY0kciE0

% TODO: https://www.youtube.com/watch?v=bmKj25nwD0I 1/(1+x^4)
% TODO: https://www.youtube.com/watch?v=Ups1CVMwm9g sin x / sqrt x
% TODO: https://www.youtube.com/watch?v=yLOuddmTSUk x^m/(1+x^n)
% TODO: https://www.youtube.com/watch?v=J6JlWjOw5rk cos (x^2) https://www.youtube.com/watch?v=s5FysKZAV2o
% TODO: https://www.youtube.com/watch?v=qm_-tmTEGKg
% TODO: https://www.youtube.com/watch?v=7M6R9WRUMiA 1/(1+x^2)
% TODO: https://www.youtube.com/watch?v=CT1P2z8xNgk 1/(cos x)^2
% TODO: https://www.youtube.com/watch?v=X7OpiWaaArY 1/(1-x^2)
% TODO: https://www.youtube.com/watch?v=O1YQF9Xb_SY x / sqrt(1-x^2)
% TODO: https://www.youtube.com/watch?v=9r5RF4EFSbI 1 / (sin x + cos x)
% TODO: https://www.youtube.com/watch?v=UZYAcfwJ8JE 1 / sqrt(tan x)
% TODO: https://www.youtube.com/watch?v=RP-gwuItggU 1 / (a^2 + x^2)
% TODO: https://www.youtube.com/watch?v=iwL60KHp0Po 1 / (a^2 + x^2)^3/2
% TODO: https://www.youtube.com/watch?v=uYNjW5Peb5A (n \choose x)

% TODO: https://www.youtube.com/watch?v=wIDOMgBFZ_Q exp(-x*x) sin(x*x)
% TODO: https://www.youtube.com/watch?v=jdL-irXyZLw sqrt(x/(1-x^3)) -> pi/3

% https://www.youtube.com/watch?v=BmdAjo1nQl4 (x-1)/(log x )(1+x^3)
% http://x/(1+2^X) https://www.youtube.com/watch?v=c4nfc5xZtgs
% https://www.youtube.com/watch?v=Sg7lhOPd548
% https://www.youtube.com/watch?v=e0xLorTapLc
% https://www.youtube.com/watch?v=55gLh8VBOBI ARCTAN ^2(1/X)

% TODO: https://www.youtube.com/watch?v=VZVCOpcK_sE MIT Bee 2025 Finals
% TODO: https://www.youtube.com/watch?v=UM13L_4mWfA Cleo
% TODO: https://www.youtube.com/watch?v=J20crVcWra0 MIT Bee 2023 Finals
% TODO: https://www.youtube.com/watch?v=Xmro-XuDhOw Feynman
% TODO: https://www.youtube.com/watch?v=oiQQx3SqYbI
% TODO: https://www.youtube.com/watch?v=QaI38XOsqL0 MIT Bee 2024 Finals
% TODO: https://www.youtube.com/watch?v=oT20UwCQ-Vg Fresnel
% TODO: https://www.youtube.com/watch?v=XnvFr2w2gUI Feynman's technique
% TODO: https://www.youtube.com/watch?v=UYAN6JNuzxo Feynman sin x / x
% TODO: https://www.youtube.com/watch?v=Mbcp6qbP04M
% TODO: https://www.youtube.com/watch?v=DbSgfJtcjgw
% TODO: https://www.youtube.com/watch?v=nA3w7iNEgRI całka konturowa
% TODO: https://www.youtube.com/watch?v=81qExKEYzo0 taka sobie z szeregeim
% TODO: https://www.youtube.com/watch?v=83mUOaF7G9A logarytm z ułamka
% TODO: https://www.youtube.com/watch?v=631elWkBn3I feynman

% https://math.stackexchange.com/questions/5052075/compute-int-11-left-frac-sin-1x1x-sqrt1-x4-right-ln-left-fra
% https://math.stackexchange.com/questions/4981343/does-int-01-x2-sin-pi-x-xx-1-x1-x-mathrmdx-frac735760-pi
% zespolona: https://math.stackexchange.com/questions/5053363/the-integral-int-mathbb-r-fracdxex-x12-pi2-frac12
% https://math.stackexchange.com/questions/4889880/prove-int-01-frac1-sqrt1-x2-arccos-left-frac3x3-3x4x2-sqrt2-x2
% https://math.stackexchange.com/questions/4977562/show-that-int-0-pi-3-arccos22-sin2-x-cos-x-mathrm-dx-frac19-pi313
%. https://math.stackexchange.com/questions/5060946/the-integral-has-a-closed-form-solution-int-01-frac1-sqrt-frac1xx-sqrt
% https://math.stackexchange.com/questions/5043189/evaluating-int-alpha-sqrt-alpha2-1-alpha-sqrt-alpha2-1-frac-ln
% https://math.stackexchange.com/questions/5039212/evaluate-int-01-frac-ln-x-sin-1xx1-x2dx
% https://math.stackexchange.com/questions/1015462/a-strange-integral-int-infty-infty-dx-over-1-leftx-tan-x-rig
% https://math.stackexchange.com/questions/4977562/show-that-int-0-pi-3-arccos22-sin2-x-cos-x-mathrm-dx-frac19-pi313
% https://math.stackexchange.com/questions/4099830/what-other-tricks-and-techniques-can-i-use-in-integration

% https://www.youtube.com/watch?v=l5uzFCL8Rk8 sqrt cos x

% https://www.youtube.com/watch?v=AqSvObSUU-Q
% https://www.youtube.com/watch?v=sUTg7PqlhvY feynman
% https://www.youtube.com/watch?v=iQW8zkenHkQ feynman?
% https://www.youtube.com/watch?v=Yf57U4_8SVo raabe log Gamma
% https://www.youtube.com/watch?v=4pvK0DTPPg8 cos cos + sin sin
% https://www.youtube.com/watch?v=YJ6j1DXS4AU berkeley
% https://www.youtube.com/watch?v=xOtQ8Mh0cvg feynman
% https://www.youtube.com/watch?v=_VkRvuSxF18 ramanujan

% https://www.youtube.com/watch?v=UTKJi9LGAZQ gaussian, feynman
% https://www.youtube.com/watch?v=LxVTHBeXss0 apery?
% https://www.youtube.com/watch?v=gO9eApg8Kfo trygonometryczna
% https://www.youtube.com/watch?v=IHjGVTeFdFA nested trig
% https://www.youtube.com/watch?v=Ebn2enuBr60