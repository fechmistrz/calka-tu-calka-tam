
% strona piąta

\tableofcontents

\chapter*{Przedmowa}
Wiele z zaprezentowanych tu problemów, ich rozwiązań (lub jednego i drugiego) zostało zaczerpnięte z popularnych zbiorów, takich jak \cite{wedrychowicz12} Banasia, Wędrychowicza.
Czasami autorzy proszą o zastosowanie wzoru na całkowanie przez części, kiedy zmyślne podstawienie prowadzi do celu krótszą drogą.
Nie przejmowaliśmy się zbytnio poleceniami z ich książki, ale żeby się w tym wszystkim nie pogubić, zamieszczamy na ostatnich stronach ,,tłumaczenie'' naszej numeracji na numerację z rozdziałów 12, 13 \cite{wedrychowicz12}.
Niektóre zadania nie zasługują na szczególną uwagę i choć ich treść została pominięta, nie brakuje ich we wspomniamym przed chwilą ,,tłumaczeniu''.

Gdzie to możliwe, staram się też proponować rozwiązania alternatywne, aby ukazać różnorodność narzędzi, które mogą być użyteczne dla osób całkujących.

TODO
$G$ oznacza...
Jeśli nie zaznaczono inaczej, $n \in \N$.

% koniec strony piątej

