%

\begin{problem_with_solution}
    \label{reuleaux_tetrahedron}%
    Czworościan Reuleaux to bryła będąca częścią wspólną czterech kul, których środki leżą w wierzchołkach czworościanu foremnego, a promienie są tej samej długości, co krawędzie tego czworościanu.
    Znaleźć objętość tej bryły,
    \begin{equation}
        V = \int_0^1
        \frac{
            8\sqrt{3}
        }{
            1 + 3t^2
        } - \frac{
            16 \sqrt{2} (3t+1) (4t^2 +t+1)^{3/2}
        }{
            (3t^2+1)(11t^2 + 2t + 3)^2
        } - \frac{
            \sqrt{2} (249 t^2 + 54t + 65)
        }{
            (11t^2 + 2t +3)^2
        } \,\mathrm{d} t.
    \end{equation}
\end{problem_with_solution}

% SOLUTION
\textbf{Problem \ref{reuleaux_tetrahedron}} -- patrz \url{https://mathworld.wolfram.com/ReuleauxTetrahedron.html}.
% SOLUTION

% TODO: https://mathworld.wolfram.com/images/gifs/FoxTrotMathTest.jpg
% TODO https://mathworld.wolfram.com/DefiniteIntegral.html




\begin{problem_with_solution}
    \label{schuster_integral}%
    Niech $S, C$ oznaczają całki Fresnela.
    % TODO: https://en.wikipedia.org/wiki/Fresnel_integral
    \begin{equation}
        \int_0^\infty (S(x)^2 + C(x)^2) \,\mathrm{d}x = \sqrt{\frac{\pi}{8}} \approx  0.62665\,70686\ldots
    \end{equation}
\end{problem_with_solution}

W 1925 roku brytyjski fizyk Arthur Schuster opublikował pracę na temat teorii światła, w której natknął się na powyższą całkę.
Sam nie potrafił jej wyznaczyć, ale zajął się tym Hardy -- dostał ten samą wartość, którą Schuster przypuszczał.

% SOLUTION
\textbf{Problem \ref{schuster_integral}} -- patrz Nahin \cite[s. 201-205]{nahin15}.
% SOLUTION

\begin{problem_with_solution}
    \label{watson_integrals1}%
    \begin{equation}
        I_1 = \frac{1}{\pi^3} \int_0^\pi\int_0^\pi\int_0^\pi \frac{\mathrm{d}u \, \mathrm{d}v \, \mathrm{d}w}{1 - \cos u \cos v \cos w} = \frac{\Gamma(1/4)^4}{4 \pi^3} = 1.39320\,39296\ldots
    \end{equation}
\end{problem_with_solution}

\begin{problem_with_solution}
    \label{watson_integrals2}%
    \begin{equation}
        I_2 = \frac{1}{\pi^3} \int_0^\pi\int_0^\pi\int_0^\pi \frac{\mathrm{d}u \, \mathrm{d}v \, \mathrm{d}w}{3 - \cos u \cos v - \cos u \cos w - \cos v \cos w} = \frac{3 \Gamma(1/3)^6}{2^{14/3} \pi^4} = 0.44822\,03943\ldots
    \end{equation}
\end{problem_with_solution}

\begin{problem_with_solution}
    \label{watson_integrals3}%
    \begin{equation}
        I_3 = \frac{1}{\pi^3} \int_0^\pi\int_0^\pi\int_0^\pi \frac{\mathrm{d}u \, \mathrm{d}v \, \mathrm{d}w}{3 - \cos u - \cos v - \cos w} = \frac{\Gamma(1/24) \Gamma(5/24) \Gamma(7/24) \Gamma(11/24)}{16 \sqrt{6} \pi^3} = 0.50546\,20197\ldots
    \end{equation}
\end{problem_with_solution}

W 1938 roku van Peype, student holenderskiego fizyka Kramnersa, napisał pracę, gdzie pojawiły się te trzy całki.
Chociaż van Peype znał wartość $I_1$, nie mógł sobie poradzić z $I_2$ oraz $I_3$, więc wysłał je do brytyjskiego fizyka Ralpha Fowlera, który przekazał je Hardy'emu, który nie poradził sobie z nimi.
Wynik jako pierwszy uzyskał George Watson, matematyk angielski.

% SOLUTION
\textbf{Problemy \ref{watson_integrals1}, \ref{watson_integrals2}, \ref{watson_integrals3}} -- patrz Nahin \cite[s. 206-212]{nahin15}.
% SOLUTION


%