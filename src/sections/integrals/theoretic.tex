%

\section{Teoretyczna teoria}
\section{Teoretyczna teoria} % SOLUTION

\begin{problem}[problem B4 na egzaminie Putnam 1968]
    \label{putnam_1968_b4}%
    Niech $f \colon \R \to \R$ będzie ciągłą funkcją taką, że całka $\int_\R f(x)\,\mathrm{d}x$ istnieje.
    Pokazać, że całka
    \begin{equation}
        \int_\R f\left(x - \frac 1 x\right)\,\mathrm{d}x
    \end{equation}
    też istnieje i przyjmuje tę samą wartość.
\end{problem}

% SOLUTION
\begin{solution}[do problemu \ref{putnam_1968_b4}]
    Będziemy całkować przez podstawienie, $x = \exp \theta$ (i potem $x = - \exp -\theta$):
    \begin{align}
        \int_{-\infty}^{\infty}f\left(x-x^{-1}\right)dx&=\int_{0}^{\infty}f\left(x-x^{-1}\right)dx+\int_{-\infty}^{0}f\left(x-x^{-1}\right)dx=\\
        &=\int_{-\infty}^{\infty}f(2\sinh\theta)\,e^{\theta}d\theta+\int_{-\infty}^{\infty}f(2\sinh\theta)\,e^{-\theta}d\theta=\\
        &=\int_{-\infty}^{\infty}f(2\sinh\theta)\,2\cosh\theta\,d\theta=\\
        &=\int_{-\infty}^{\infty}f(x)\,dx.
    \end{align}
\end{solution}
% SOLUTION

%