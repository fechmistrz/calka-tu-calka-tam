%

% Harvard:
% TODO: https://www.youtube.com/watch?v=hxAUEat_04o

% 2025
% TODO: https://www.youtube.com/watch?v=rGolSnrWc7s


\subsection{MIT Integration BEE 2024}
% kwalifikacje, problem 2: \frac{(x-1)^{\log (x+1)}}{(x+1)^{\log (x-1)}} = 1
% https://www.youtube.com/watch?v=p75k1DbuS0o

\begin{problem_with_solution}[ćwierćfinał 2, problem 2]
    \label{bee_mit_2024_q2_p2}%
    \begin{equation}
        I = \int_0^1 \frac 1 x \log (1 + x^2 + x^3 + x^4 + x^5 + x^6 + x^7 + x^9) \,\mathrm{d}x
    \end{equation}
\end{problem_with_solution}

% SOLUTION
% https://math.stackexchange.com/questions/1617081/proving-an-integration-equality
\textbf{Problem \ref{bee_mit_2024_q2_p2}} -- łatwo widać, że szukana całka jest równa $I_2 + I_3 + I_4$, gdzie
\begin{align}
    I_n & = \int_0^1 \frac {\log (1 + x^n)}{x} \,\mathrm{d}x \\
        & = \frac 1 n \int_0^1 \frac {\log (1 + x)}{x} \,\mathrm{d}x \\
        & = \frac 1 n \int_0^1 \frac 1 x \sum_{k=1}^\infty \frac{(-1)^{k-1}x^k}{k} \,\mathrm{d}x \\
        & = \frac 1 n \sum_{k=1}^\infty \frac{(-1)^{k-1}}{k^2} \\
        & = \frac {\pi^2}{12n}.
\end{align}
% SOLUTION

% qualifier q7 https://www.youtube.com/watch?v=YI-c9b7vNEs
% q9 https://www.youtube.com/watch?v=hzWtJ-qadws
% q15 https://www.youtube.com/watch?v=mxoyNN5g3hE
% q19 https://www.youtube.com/watch?v=7JwGegZDz4Y

% 2023
% https://math.stackexchange.com/questions/4642139/question-from-mit-integration-bee-2023-final-evaluate-int1-0-sum-infty-n
% https://www.youtube.com/watch?v=gn6LOe_bgw4

% 2022
% regular q8 https://www.youtube.com/watch?v=MlyvsB7PBlI

% 2020
% TODO: https://www.youtube.com/watch?v=oZWqG4IIHc0

% 2012
% q10, qualifier https://www.youtube.com/watch?v=HV489m57f2s

%