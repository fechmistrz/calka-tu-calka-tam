\subsection{Oszacowania liczby $\pi$}
\begin{problem}[całka Dalzella]
\label{22_7_pi}%
\begin{equation}
    \int_0^1 \frac{x^4(1-x)^4}{1 + x^4} \,\mathrm{d}x = \frac{22}{7} - \pi.
\end{equation}
\end{problem}

Donald Percy Dalzell \cite{dalzell44} jako pierwszy opublikował to cudo.
Ograniczając mianownik z dołu oraz góry przez $1$ oraz $2$ możemy dojść do wniosku, że
\begin{equation}
    \frac{22}{7} - \frac {1}{630} < \pi < \frac{22}{7} - \frac{1}{1260},
\end{equation}
a więc pomylić się o mniej niż $0.015\%$!

% TODO - na en.wiki jest łatwe w przepisaniu rozwiązanie
% SOLUTION
\textbf{Problem \ref{22_7_pi}} -- patrz \cite[s. 24]{nahin15}
% SOLUTION

% TODO: patrz też https://math.stackexchange.com/questions/1956/is-there-an-integral-that-proves-pi-333-106
Istnieje wiele uogólnień powyższego wyniku do innych przybliżeń liczby $\pi$, dobrym źródłem dalszych informacji, a może nawet inspiracji jest artykuł Lucasa \cite{lucas05}.
Na przykład:

\begin{problem}
\begin{equation}
    \int_0^1 \frac {x^8(1-x)^8 (25+816x^2)}{3164 (1+x^2)} \,\mathrm{d} x = \frac {355}{113} - \pi.
\end{equation}
\end{problem}

albo:

\begin{problem}
\begin{equation}
    \int_0^1 \frac{x^5 ( 1-x)^6 (197 + 462 x^2)}{530 (1+x^2)} \,\mathrm{d}x = \pi - \frac{333}{106}.
\end{equation}
\end{problem}