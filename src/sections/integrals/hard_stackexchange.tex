%

\subsection{Znalezione na math.stackexchange.com}
% SOLUTION
\subsection{Znalezione na math.stackexchange.com}
% SOLUTION
Wszystkie poniższe całki pojawiają się na stronie na math.stackexchange.com.

%%

% https://math.stackexchange.com/q/1653979
\begin{problem}[pytanie 9402]
    \label{stack_9402}%
    % TODO: https://math.stackexchange.com/questions/9402/calculating-the-integral-int-0-infty-frac-cos-x1x2-mathrmdx-with
    \begin{equation}
        \int_0^\infty \frac {\cos b x}{1+x^2} \,\mathrm{d}x = \frac{\pi} 2 \exp(-b).
    \end{equation}
\end{problem}

%%

\begin{problem}[pytanie 15719]
    \label{stack_15719}%
    \begin{equation}
        \int \frac{1 + x^2}{(1 - x^2) \sqrt{1 + x^4}} \,\mathrm{d}x = \frac{1}{\sqrt 2} \log \frac{2x + \sqrt{2x^4 + 2}}{x^2 - 1}.
    \end{equation}
\end{problem}

%%

% https://math.stackexchange.com/questions/110457/closed-form-for-int-0-infty-fracxn1-xmdx
\begin{problem_with_solution}[pytanie 110457]
    \label{stack_110457}%
    Niech $0 < n < m$, wtedy
    \begin{equation}
        \int_0^\infty \frac{n x^{n-1}}{1 + x^m} \,\mathrm{d} x = \frac {\pi n} m \operatorname{csc} \frac {\pi n}{m}.
    \end{equation}
\end{problem_with_solution}

% SOLUTION
\textbf{Problem \ref{stack_110457}} -- podstawiamy $x = \tan^{2/m} \theta$, co prowadzi do całki
\begin{align}
    I & = \int_0^\infty \frac{x^{n-1}}{1 + x^m} \,\mathrm{d} x \\
      & = \int_0^{\pi/2} \frac 2 m \tan^{2n/m - 1} \theta \,\mathrm{d}\theta \\
      & = \frac 1 m \beta\left( \frac nm, 1 - \frac nm \right) \\
      & = \frac 1 m \Gamma \left(\frac nm\right) \Gamma \left(1 - \frac nm\right) \\
      & = \frac \pi m \operatorname{csc} \frac {\pi n}{m}.
\end{align}
% SOLUTION

%%

% https://math.stackexchange.com/questions/155941/evaluate-the-integral-int-01-frac-lnx1x21-mathrm-dx
\begin{problem_with_solution}[pytanie 155941]
    \label{stack_155941}%
    \begin{equation}
        \int_0^1 \frac{\log (1+x)}{1 + x^2} \,\mathrm{d}x = \frac \pi 8  \log 2.
    \end{equation}
\end{problem_with_solution}

% SOLUTION
\textbf{Problem \ref{stack_155941}} -- podstawiamy $x = \tan \theta$.
% SOLUTION

%%

% https://math.stackexchange.com/q/178790
\begin{problem}[pytanie 178790]
    \label{stack_178790}%
    \begin{equation}
        \int_0^{\pi/2} \frac{x^2}{x^2 + [\log (2 \cos x)]^2} \,\mathrm{d}x = \frac{\pi}{8} (1 - \gamma + \log (2 \pi)).
    \end{equation}
\end{problem}

%%

% https://math.stackexchange.com/questions/187729/evaluating-int-0-infty-sin-x2-dx-with-real-methods
\begin{problem_with_solution}[pytanie 187729, całka Fresnela]
    \label{stack_187729}%
    \begin{equation}
        I = \int_0^\infty \sin (x^2) \,\mathrm{d} x = \sqrt{\frac \pi 8}.
    \end{equation}
\end{problem_with_solution}

Całki Fresnela mają praktyczne zastosowanie, historycznie pierwszym było obliczenie natężenia pola elektromagnetycznego w środowisku, gdzie światło ugina się wokół nieprzezroczystych obiektów.
\index{całka Fresnela}%

% SOLUTION
\textbf{Problem \ref{stack_187729}} -- znajdziemy ogólniejszą całkę $I_\lambda$ funkcji $\sin(x^2) e^{-\lambda x^2}$ nad zbiorem $[0, \infty)$.
\begin{align}
    I_\lambda^2 & = \left(\int_0^\infty \sin(x^2) e^{-\lambda x^2} \,\mathrm{d}x \right)^2 \\
    & = \int_0^\infty \int_0^\infty \sin(x^2)\sin(y^2) e^{- \lambda(x^2+y^2)}\,\mathrm{d}y\,\mathrm{d}x \\
    & = \frac12 \int_0^\infty \int_0^\infty \left(\cos(x^2-y^2)-\cos(x^2+y^2)\right) e^{- \lambda(x^2+y^2)}\,\mathrm{d}y\,\mathrm{d}x \\
    & = \frac12 \int_0^{\pi/2} \int_0^\infty \left(\cos(r^2\cos(2\phi))-\cos(r^2)\right)e^{- \lambda r^2} \,r\,\mathrm{d}r\,\mathrm{d}\phi \\
    & = \frac14 \int_0^{\pi/2} \int_0^\infty \left(\cos(s\cos(2\phi))-\cos(s)\right) e^{- \lambda s} \,\mathrm{d}s\,\mathrm{d}\phi \\
    & = \frac14 \int_0^{\pi/2} \left( \frac{ \lambda}{\cos^2(2\phi)+ \lambda^2} - \frac{ \lambda}{1+ \lambda^2}\right)\,\mathrm{d}\phi \\
    & = \frac12 \int_0^{\pi/4} \frac{ \lambda\,\mathrm{d}\phi}{\cos^2(2\phi)+ \lambda^2} - \frac{ \lambda\pi/8}{1+ \lambda^2} \\
    & = \frac14 \int_0^{\pi/4} \frac{ \lambda\,\mathrm{d} \tan(2\phi)} {1+ \lambda^2+ \lambda^2 \tan^2(2\phi)} - \frac{ \lambda\pi/8}{1+ \lambda^2} \\
    & = \frac14 \int_0^\infty \frac1{1+ \lambda^2+t^2}\,\mathrm{d}t - \frac{ \lambda\pi/8}{1+ \lambda^2} \\
    & = \frac{\pi/8}{\sqrt{1+ \lambda^2}} - \frac{ \lambda\pi/8}{1+ \lambda^2}
\end{align}
% SOLUTION

%%

% https://math.stackexchange.com/questions/426325/evaluate-int-01-frac-log-left-1x2-sqrt3-right1x-mathrm-dx
\begin{problem}[pytanie 426325]
    \label{stack_426325}%
    \begin{equation}
        \int_0^1 \frac{\log \left(1 + x^{2 + \sqrt 3}\right)}{1 + x} \,\mathrm{d} x = \frac{\pi^2}{12} (1 - \sqrt 3) + \log (2) \log(1 + \sqrt 3).
    \end{equation}
\end{problem}

%%

% https://math.stackexchange.com/questions/464769/how-to-prove-int-01-tan-1-left-frac-tanh-1x-tan-1x-pi-tanh-1
\begin{problem}[pytanie 464769]
    \label{stack_464769}%
    \begin{equation}
        \int_0^1 \arctan \frac { \operatorname{artanh} x - \arctan x} {\pi + \operatorname{artanh} x - \arctan x}  \, \frac{\mathrm{d}x}{x} = \frac \pi 4 \log \frac{\pi}{2 \sqrt{2}}.
    \end{equation}
\end{problem}

%%

% https://math.stackexchange.com/questions/507425/an-integral-involving-airy-functions-int-0-infty-fracxp-operatornameai
\begin{problem_with_solution}[pytanie 507425]
    \label{stack_507425}%
    Niech
    \begin{align}
        \operatorname{Ai} (x) & = \frac 1 \pi \int_0^\infty \cos \left( x z + \frac {z^3} 3 \right) \,\mathrm{d}z, \\
        \operatorname{Bi} (x) & = \frac 1 \pi \int_0^\infty \sin \left( x z + \frac {z^3} 3 \right) + \exp \left( x z - \frac {z^3} 3 \right) \,\mathrm{d}z
    \end{align}
    będą funkcjami Airy'ego, zaś $n \in \mathbb N$ parametrem.
    Znaleźć
    \begin{equation}
        I_n = \int_0^\infty \frac{x^{3n} \,\mathrm{d} x}{(\operatorname{Ai} x)^2 + (\operatorname{Bi} x)^2}.
    \end{equation}
\end{problem_with_solution}

% SOLUTION
\textbf{Problem \ref{stack_507425}} -- $I_n = \pi^2 a_{2n} / (6 \cdot 2^{7n})$, gdzie $a_0 = 1$, $a_{n+1} = (6n+4)a_n \sum_{k=0}^n a_k a_{n-k}$.
% SOLUTION


%%

% https://math.stackexchange.com/questions/523027/a-math-contest-problem-int-01-ln-left1-frac-ln2x4-pi2-right-frac
\begin{problem}[pytanie 523027]
    \label{stack_523027}%
    \begin{equation}
        \int_0^1 \log\left(1+\left(\frac{\log x}{2\pi}\right)^2 \right)\frac{\log(1-x)}x \,\mathrm{d} x=-\pi^2\left(4\zeta'(-1)+\frac23\right).
    \end{equation}
\end{problem}

%%

% https://math.stackexchange.com/questions/541751/how-prove-this-i-int-0-infty-frac1x-ln-left-frac1x1-x-right2/541861#541861
\begin{problem}[pytanie 541751]
    \label{stack_541751}%
    \begin{equation}
        I = \int_0^\infty \frac{1}{x} \log \left(\frac{1+x}{1-x}\right)^2 \,\mathrm{d}x = \pi^2.
    \end{equation}
\end{problem}

%%

% https://math.stackexchange.com/q/562694
\begin{problem}[pytanie 562694]
    \label{stack_562694}%
    \begin{equation}
        \int_{-1}^1 \frac{1}{x} \sqrt{\frac{1+x}{1-x}} \log \frac{2x^2+2x+1}{2x^2-2x+1} \,\mathrm{d}x = 4 \pi \operatorname{arccot} \sqrt{\phi}.
    \end{equation}
\end{problem}

%%

\begin{problem}[pytanie 570997]
    \label{stack_570997}%
    \begin{equation}
        \int_0^1 \frac{\log (x + \sqrt 2)}{\sqrt{x(1-x)(2-x)}} \,\mathrm{d}x = \frac{\pi^{3/2}}{8\sqrt{2} \Gamma(3/4)^2 } \left(7 \log 2 + 4 \log (1 + \sqrt 2) - \pi \right).
    \end{equation}
\end{problem}

%%

% https://math.stackexchange.com/questions/815863/compute-int-0-pi-4-frac1-x2-ln1x21x2-1-x2-ln1-x21-x4
\begin{problem}[pytanie 815863]
    \label{stack_815863}%
    \begin{align}
        I & = \int_0^{\pi/4} \frac{ (1-x^2) [ \log(1+x^2) - \log(1 - x^2)] + 1 + x^2}{(1-x^4)(1+x^2)} x \exp \frac {x^2 - 1}{x^2 + 1} \,\mathrm{d} x \\
        & =  - \frac 1 4  \exp \frac{\pi^2 - 16}{\pi^2 + 16} \log \frac {16 - \pi^2}{16 + \pi^2}.
    \end{align}
\end{problem}

%%

% TODO: https://math.stackexchange.com/a/942440

%%

\begin{problem_with_solution}[pytanie 1582943]
    \label{stack_1582943}%
    \begin{equation}
        \int_0^\infty \left(\frac{\tanh x}{x}\right)^2 \,\mathrm{d}x = \frac{14 \zeta (3)}{\pi^2}.
    \end{equation}
\end{problem_with_solution}

% SOLUTION
\textbf{Problem \ref{stack_1582943}} -- dowodzimy najpierw, że poniższe całki są równe:
\begin{equation}
    \frac{\pi^2}{4} \int_0^\infty \frac{\tanh x \cdot \tanh xs}{x^2} \,\mathrm{d}x = s \int_0^1 \log \frac{1-x}{1+x} \log \frac{1-x^s}{1+x^s} \frac{\mathrm{d}x}{x}.
\end{equation}
% SOLUTION

%%

% https://math.stackexchange.com/q/1653979
\begin{problem}[pytanie 1653979]
    \label{stack_1653979}%
    % Niech $\phi = \frac 1 2 (1 + \sqrt 5)$ oznacza złotą liczbę.
    \begin{equation}
        \int_0^\infty \frac{5x^2}{1  + x^{10}} \,\mathrm{d}x = \frac{\pi}{\phi}.
    \end{equation}
\end{problem}

%%

% https://math.stackexchange.com/q/2529614
\begin{problem}[pytanie 2529614]
    Korzystając z technik analizy zespolonej pokazać, że
    \label{stack_2529614}%
    \begin{equation}
        \int_0^\infty \frac{x^{-\mathrm{i}a}}{x^2+bx+1} \,\mathrm{d}x = \frac{2\pi}{\sqrt{4-b^2}} \cdot \frac{\sinh (a \arccos (b/2))}{\sinh (a \pi)}.
    \end{equation}
\end{problem}

%%

% https://math.stackexchange.com/q/2826571
\begin{problem}[pytanie 2826571]
    \label{stack_2826571}%
    Niech $(m, n)$ oznacza największy wspólny dzielnik liczb $m, n$.
    Wtedy
    \begin{equation}
        \int_0^{\pi/2} \log \lvert \sin(mx) \rvert \cdot \log \lvert\sin(nx)\rvert \, dx = \frac{\pi^3}{24} \frac{(m,n)^2}{mn}+\frac{\pi\log (2)^2}{2}.
    \end{equation}
\end{problem}

%%

% https://math.stackexchange.com/questions/3490404/
\begin{problem}[pytanie 3490404]
    \label{stack_3490404}%
    Niech $f(x) = x \log (1 - \sin x) / \sin x$.
    Udowodnić, że
    \begin{align}
        2\int_0^{\pi/2} f(x) \,\mathrm{d}x =
        \int_{\pi/2}^\pi\ f(x) \,\mathrm{d}x,
    \end{align}
    bez uprzedniego znajdowania funkcji pierwotnej $f$.
\end{problem}

%%

% https://math.stackexchange.com/q/3981861
\begin{problem}[pytanie 3981861]
    Udowodnić lub podać kontrprzykład: niech $f \colon [0, 2\pi] \to [0, 2\pi]$ będzie różniczkowalną funkcją taką, że $f(0) = f(2\pi)$. Wtedy
    \label{stack_3981861}%
    \begin{equation}
        \left(\int_0^{2 \pi} \cos f(x) \,d x\right)^2
        +
        \left(\int_0^{2 \pi} \sqrt{(f'(x))^2+\sin^2 f(x)} \, \mathrm{d}x\right)^2 \ge (2 \pi)^2.
    \end{equation}
\end{problem}

%%

% https://math.stackexchange.com/q/4568778
\begin{problem}[pytanie 4568778]
    \label{stack_4568778}%
    \begin{equation}
        \int_0^1 \frac{\log (x+1) - \log(2x^2)}{\sqrt{x^2 + 2x}}\,\mathrm{d}x = \frac{\pi^2}{2}.
    \end{equation}
\end{problem}

%