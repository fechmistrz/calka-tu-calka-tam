%

\begin{problem_with_solution}
    \label{valean_1_10}%
    Niech $n \ge 1$ będzie liczbą naturalną.
    Znaleźć
    \begin{equation}
        I_n = \int_0^1 \frac 1 x \log(1-x) \log^{2n} x \log (1+x) \,\mathrm{d}x.
    \end{equation}
    Jeśli jest to za trudne, pokazać, że
    \begin{align}
        I_1 & = \frac 3 4 \zeta (2) \zeta (3) - \frac {27}{16} \zeta(5), \\
        I_2 & = \frac 9 4 \zeta (3) \zeta (4) + \frac{45}{4} \zeta(2) \zeta(5) - \frac{363}{16} \zeta (7), \\
        I_3 & = \frac{2835}{8} \zeta(2) \zeta (7) + \frac {135}{8} \zeta (3) \zeta (6) + \frac {675}{8} \zeta (4) \zeta (5) - \frac {22635}{32} \zeta (9).
    \end{align} 
\end{problem_with_solution}

% the evaluation of a class of logarithmic integrals using a slightly modified result from ,,Table of Integrals, Series and Products'' by I. S. Gradshteyn and I. M. Ryzhik together with a series result elementarily proved by Guy Bastien

% SOLUTION
\begin{solution}[do problemu \ref{valean_1_10}]
    Patrz \cite[s. 6, 7]{valean19}.
\end{solution}
% SOLUTION

%