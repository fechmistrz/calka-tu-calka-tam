%

\begin{problem_with_solution}
    \label{valean_1_5}%
    Niech $s > 0$ będzie liczbą rzeczywistą, zaś $\psi$ oznacza funkcję digamma.
    Wtedy
    \begin{align}
        \int_0^1 \frac{x^{s-1}}{x+1} \,\mathrm{d} x & = \psi(s) - \psi\left(\frac s2\right) - \log 2 \\
        \int_0^\infty e^{-sx} \tanh x \,\mathrm{d} x & = \frac 1 2 \left[\psi\left(\frac{s+2}{4}\right) - \psi \left(\frac s4 \right) - \frac 2 s\right]. 
    \end{align}
\end{problem_with_solution}

% (Valean nazywa to ,,a couple of practical definite integrals expressed in terms of the digamma function'').

% SOLUTION
\begin{solution}[do problemu \ref{valean_1_5}]
    Patrz \cite[s. 3]{valean19}.
\end{solution}
% SOLUTION

%s