%

\begin{problem_with_solution}
    \label{valean_1_5}%
    Niech $s > 0$ będzie liczbą rzeczywistą.
    % , zaś $\psi$ oznacza funkcję digamma.
    Wtedy
    \begin{equation}
        \int_0^1 \frac{x^{s-1}}{x+1} \,\mathrm{d} x = \frac 1 2 \left[\psi \left(\frac{s+1}{2}\right) - \psi \left(\frac s 2 \right) \right].
    \end{equation}
    Przy tych samych założeniach
    \begin{equation}
        \int_0^\infty \frac{\tanh x}{ \exp (sx)} \,\mathrm{d} x = \frac 1 2 \left[\psi\left(\frac{s+2}{4}\right) - \psi \left(\frac s4 \right) - \frac 2 s\right]. 
    \end{equation}
\end{problem_with_solution}

% SOLUTION
\textbf{Problem \ref{valean_1_5}}.
Vălean \cite[s. 3]{nahin15}.
% SOLUTION

%