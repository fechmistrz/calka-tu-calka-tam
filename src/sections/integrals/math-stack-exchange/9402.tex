%

% https://math.stackexchange.com/q/1653979
\begin{problem_with_solution}[pytanie 9402]
    \label{stack_9402}%
    % TODO: https://math.stackexchange.com/questions/9402/calculating-the-integral-int-0-infty-frac-cos-x1x2-mathrmdx-with
    \begin{equation}
        \int_0^\infty \frac {\cos ax}{x^2 + b^2} \,\mathrm{d}x = \frac{\pi}{2b} \exp(-ab).
    \end{equation}
\end{problem_with_solution}

% SOLUTION

\textbf{Problem \ref{stack_9402}}.
Wskazówka: rozpatrzeć funkcję
\begin{equation}
    F(t) = \int_0^\infty \frac{\sin (xt)}{x(1+x^2)} \,\mathrm{d} x,
\end{equation}
która spełnia równanie $F''(t) - F(t) + \pi/2 = 0$, zatem $F(t) = \frac \pi 2 (1 - e^{-t})$.
Teraz zróżniczkować pod znakiem całki.
Inne rozwiązanie (dla $b = 1$) zaczyna się od zauważenia, że szukana całka to
\begin{equation}
    \int_0^\infty \frac{\cos ax}{x} \left(\frac{x}{1+x^2}\right) \mathrm{d}x = 
    \int_0^\infty \frac{\cos ax}{x} \left(\int_0^\infty \frac{\sin xt}{e^t} \,\mathrm{d}t \right) \mathrm{d}x,
\end{equation}
gdzie wystarczy zmienić kolejność całkowania. % => https://en.wikipedia.org/wiki/Dirichlet_integral, gdzie to jest w tej książce?
Jeszcze inne rozwiązania korzystają z transformaty Fouriera albo Ramanujan's master theorem.

% SOLUTION

%