
\emph{Polilogarytm} (funkcja Jonquière’a) jest zdefiniowany szeregiem
\begin{equation}
    \polylog_n(z) = \sum_{k=1}^\infty \frac{z^k}{k^n},
\end{equation}
który jest zbieżny dla $|z| < 1$.
Spełnia rekurencję
\begin{equation}
    \polylog_n(z) = \int_0^z \frac{\polylog_{n-1}(t)}{t} \,\mathrm{d}t
\end{equation}
z warunkiem brzegowym $\polylog_1 (z) = - \log (1 - z)$.

\emph{Funkcja digamma} to pochodna logarytmiczna funkcji Gamma, co tłumaczy nazwę, ale nie symbol:
\begin{equation}
    \Psi (z) := \frac{\Gamma'(z)}{\Gamma(z)}.
\end{equation}